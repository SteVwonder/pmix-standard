%%%%%%%%%%%%%%%%%%%%%%%%%%%%%%%%%%%%%%%%%%%%%%%%%
% Chapter: Data Structures
%%%%%%%%%%%%%%%%%%%%%%%%%%%%%%%%%%%%%%%%%%%%%%%%%
\chapter{Data Structures and Types}
\label{chap:struct}

This chapter defines PMIx standard data structures, types, and constants.
These apply to all consumers of the PMIx interface.
Where necessary for clarification, the description of, for example, an attribute may be copied from this chapter into a section where it is used.

A PMIx implementation may define additional attributes beyond those specified in this document.

\adviceimplstart
If a PMIx implementation chooses to define additional attributes they should avoid using the \code{PMIX} prefix.
This helps the end user distinguish between what is defined by the PMIx standard and what is specific to that PMIx implementation.
\adviceimplend


%%%%%%%%%%%
\section{Constants}

\ac{PMIx} defines a few values that are used throughout the standard to set the size of fixed arrays or as a means of identifying values with special meaning.
The community makes every attempt to minimize the number of such definitions.
The constants defined in this section may be used before calling any \ac{PMIx} library initialization routine.
Additional constants associated with specific data structures or types are defined in the section describing that data structure or type.

\begin{constantdesc}
%
\declareconstitemvalue{PMIX_MAX_NSLEN}{255}
Maximum namespace string length as an integer.
%
\declareconstitemvalue{PMIX_MAX_KEYLEN}{511}
Maximum key string length as an integer.
%
\end{constantdesc}

%%%%%%%%%%%
\subsection{Error Constants}
\label{api:struct:errors}
\declarestruct{pmix_status_t}

The \refstruct{pmix_status_t} structure is an \code{int} type for return status.

The tables shown in this section define the possible values for \refstruct{pmix_status_t}.
PMIx errors are required to always be negative, with 0 reserved for \refconst{PMIX_SUCCESS}.

A PMIx implementation must define all of the constants defined in this section, even if they will never return the specific value to the caller.

\adviceuserstart
Other than \refconst{PMIX_SUCCESS} (which is required to be zero), the actual value of any \ac{PMIx} error constant is left to the \ac{PMIx} library implementor. Thus, users are advised to always refer to constant by name, and not a specific implementation's value, for portability and compatibility across library versions.
\adviceuserend

%%%%%%%%%%%
\subsubsection{PMIx v1 Error Constants}

The following list contains those constants defined in the PMIx v1 standard.
Those values in the list that were deprecated in later standards are denoted as such.
\ac{PMIx} errors are always negative, with \code{0} reserved for success.

\begin{constantdesc}
%
\declareconstitem{PMIX_SUCCESS}
Success
%
\declareconstitem{PMIX_ERROR}
General Error
%
\declareconstitem{PMIX_ERR_SILENT}
Silent error
%
\declareconstitem{PMIX_ERR_DEBUGGER_RELEASE}
Error in debugger release
%
\declareconstitem{PMIX_ERR_PROC_RESTART}
Fault tolerance: Error in process restart
%
\declareconstitem{PMIX_ERR_PROC_CHECKPOINT}
Fault tolerance: Error in process checkpoint
%
\declareconstitem{PMIX_ERR_PROC_MIGRATE}
Fault tolerance: Error in process migration
%
\declareconstitem{PMIX_ERR_PROC_ABORTED}
Process was aborted
%
\declareconstitem{PMIX_ERR_PROC_REQUESTED_ABORT}
Process is already requested to abort
%
\declareconstitem{PMIX_ERR_PROC_ABORTING}
Process is being aborted
%
\declareconstitem{PMIX_ERR_SERVER_FAILED_REQUEST}
Failed to connect to the server
%
\declareconstitem{PMIX_EXISTS}
Requested operation would overwrite an existing value
%
\declareconstitem{PMIX_ERR_INVALID_CRED}
Invalid security credentials
%
\declareconstitem{PMIX_ERR_HANDSHAKE_FAILED}
Connection handshake failed
%
\declareconstitem{PMIX_ERR_READY_FOR_HANDSHAKE}
Ready for handshake
%
\declareconstitem{PMIX_ERR_WOULD_BLOCK}
Operation would block
%
\declareconstitem{PMIX_ERR_UNKNOWN_DATA_TYPE}
Unknown data type
%
\declareconstitem{PMIX_ERR_PROC_ENTRY_NOT_FOUND}
Process not found
%
\declareconstitem{PMIX_ERR_TYPE_MISMATCH}
Invalid type
%
\declareconstitem{PMIX_ERR_UNPACK_INADEQUATE_SPACE}
Inadequate space to unpack data
%
\declareconstitem{PMIX_ERR_UNPACK_FAILURE}
Unpack failed
%
\declareconstitem{PMIX_ERR_PACK_FAILURE}
Pack failed
%
\declareconstitem{PMIX_ERR_PACK_MISMATCH}
Pack mismatch
%
\declareconstitem{PMIX_ERR_NO_PERMISSIONS}
No permissions
%
\declareconstitem{PMIX_ERR_TIMEOUT}
Timeout expired
%
\declareconstitem{PMIX_ERR_UNREACH}
Unreachable
%
\declareconstitem{PMIX_ERR_IN_ERRNO}
Error defined in \code{errno}
%
\declareconstitem{PMIX_ERR_BAD_PARAM}
Bad parameter
%
\declareconstitem{PMIX_ERR_RESOURCE_BUSY}
Resource busy
%
\declareconstitem{PMIX_ERR_OUT_OF_RESOURCE}
Resource exhausted
%
\declareconstitem{PMIX_ERR_DATA_VALUE_NOT_FOUND}
Data value not found
%
\declareconstitem{PMIX_ERR_INIT}
Error during initialization
%
\declareconstitem{PMIX_ERR_NOMEM}
Out of memory
%
\declareconstitem{PMIX_ERR_INVALID_ARG}
Invalid argument
%
\declareconstitem{PMIX_ERR_INVALID_KEY}
Invalid key
%
\declareconstitem{PMIX_ERR_INVALID_KEY_LENGTH}
Invalid key length
%
\declareconstitem{PMIX_ERR_INVALID_VAL}
Invalid value
%
\declareconstitem{PMIX_ERR_INVALID_VAL_LENGTH}
Invalid value length
%
\declareconstitem{PMIX_ERR_INVALID_LENGTH}
Invalid argument length
%
\declareconstitem{PMIX_ERR_INVALID_NUM_ARGS}
Invalid number of arguments
%
\declareconstitem{PMIX_ERR_INVALID_ARGS}
Invalid arguments
%
\declareconstitem{PMIX_ERR_INVALID_NUM_PARSED}
Invalid number parsed
%
\declareconstitem{PMIX_ERR_INVALID_KEYVALP}
Invalid key/value pair
%
\declareconstitem{PMIX_ERR_INVALID_SIZE}
Invalid size
%
\declareconstitem{PMIX_ERR_INVALID_NAMESPACE}
Invalid namespace
%
\declareconstitem{PMIX_ERR_SERVER_NOT_AVAIL}
Server is not available
%
\declareconstitem{PMIX_ERR_NOT_FOUND}
Not found
%
\declareconstitem{PMIX_ERR_NOT_SUPPORTED}
Not supported
%
\declareconstitem{PMIX_ERR_NOT_IMPLEMENTED}
Not implemented
%
\declareconstitem{PMIX_ERR_COMM_FAILURE}
Communication failure
%
\declareconstitem{PMIX_ERR_UNPACK_READ_PAST_END_OF_BUFFER}
Unpacking past the end of the buffer provided
%
\end{constantdesc}

%%%%%%%%%%%
\subsubsection{PMIx v2 Error Constants}

The following list contains constants added in the PMIx v2 standard.

\begin{constantdesc}
%
\declareconstitem{PMIX_ERR_LOST_CONNECTION_TO_SERVER}
Lost connection to server
%
\declareconstitem{PMIX_ERR_LOST_PEER_CONNECTION}
Lost connection to peer
%
\declareconstitem{PMIX_ERR_LOST_CONNECTION_TO_CLIENT}
Lost connection to client
%
\declareconstitem{PMIX_QUERY_PARTIAL_SUCCESS}
Query partial success (used by query system)
%
\declareconstitem{PMIX_NOTIFY_ALLOC_COMPLETE}
Notify that allocation is complete
%
\declareconstitem{PMIX_JCTRL_CHECKPOINT}
Job control: Monitored by PMIx client to trigger checkpoint operation
%
\declareconstitem{PMIX_JCTRL_CHECKPOINT_COMPLETE}
Job control: Sent by PMIx client and monitored by PMIx server to notify that requested checkpoint operation has completed.
%
\declareconstitem{PMIX_JCTRL_PREEMPT_ALERT}
Job control: Monitored by PMIx client to detect an \ac{RM} intending to preempt the job.
%
\declareconstitem{PMIX_MONITOR_HEARTBEAT_ALERT}
Job monitoring: Heartbeat alert
%
\declareconstitem{PMIX_MONITOR_FILE_ALERT}
Job monitoring: File alert
%
\declareconstitem{PMIX_PROC_TERMINATED}
Process terminated - can be either normal or abnormal termination
%
\end{constantdesc}

The following list contains operational error constants introduced in the v2 standard.

\begin{constantdesc}
%
\declareconstitem{PMIX_ERR_EVENT_REGISTRATION}
Error in event registration
%
\declareconstitem{PMIX_ERR_JOB_TERMINATED}
Error job terminated
%
\declareconstitem{PMIX_ERR_UPDATE_ENDPOINTS}
Error updating endpoints
%
\declareconstitem{PMIX_MODEL_DECLARED}
Model declared
%
\declareconstitem{PMIX_GDS_ACTION_COMPLETE}
The \ac{GDS} action has completed
%
\end{constantdesc}

The following list contains system error constants introduced in the v2 standard.

\begin{constantdesc}
%
\declareconstitem{PMIX_ERR_NODE_DOWN}
Node down
%
\declareconstitem{PMIX_ERR_NODE_OFFLINE}
Node is marked as offline
%
\end{constantdesc}

The following list contains event handler error constants introduced in the v2 standard.

\begin{constantdesc}
%
\declareconstitem{PMIX_EVENT_NO_ACTION_TAKEN}
Event handler: No action taken
%
\declareconstitem{PMIX_EVENT_PARTIAL_ACTION_TAKEN}
Event handler: Partial action taken
%
\declareconstitem{PMIX_EVENT_ACTION_DEFERRED}
Event handler: Action deferred
%
\declareconstitem{PMIX_EVENT_ACTION_COMPLETE}
Event handler: Action complete
%
\end{constantdesc}

%%%%%%%%%%%
\subsubsection{User-Defined Error Constants}

PMIx establishes an error code boundary for constants defined in the PMIx standard. Negative values larger than this (and any positive values greater than zero) are guaranteed not to conflict with PMIx values.

\begin{constantdesc}
%
\declareconstitemvalue{PMIX_EXTERNAL_ERR_BASE}
A starting point for user-level defined error constants.
Negative values lower than this are guaranteed not to conflict with PMIx values.
Definitions should always be based on the \refconst{PMIX_EXTERNAL_ERR_BASE} constant and \emph{not} a specific value as the value of the constant may change.
%
\end{constantdesc}



%%%%%%%%%%%
\section{Data Types}

This section defines various data types used by the PMIx APIs.

%%%%%%%%%%%
\subsection{Rank Structure}
\declarestruct{pmix_rank_t}

The \refstruct{pmix_rank_t} structure is a \code{uint32_t} type for rank values.

The following constants can be used to set a variable of the type \refstruct{pmix_rank_t}.
Valid rank values start at \code{0}.

\begin{constantdesc}
%
\declareconstitemvalue{PMIX_RANK_UNDEF}{UINT32_MAX}
Define a value for requests for job-level data where the information itself is not associated with any specific rank, or when a request involves a rank that is not known.
For example, when someone requests information thru one of the PMIx v1 legacy interfaces where the rank is typically encoded into the key itself, since there is no rank parameter in the PMIx v1 \refapi{PMIx_Get} API in that version.
%
\declareconstitemvalue{PMIX_RANK_WILDCARD}{UINT32_MAX-1}
Define a value to indicate that the user wants the data for the given key from every rank that posted that key.
%
\declareconstitemvalue{PMIX_RANK_LOCAL_NODE}{UINT32_MAX-2}
Special rank value used to define groups of ranks for use in collectives.
This constant defines the group of all ranks on a local node.
%
\end{constantdesc}


%%%%%%%%%%%
\subsection{Process Structure}
\declarestruct{pmix_proc_t}

The \refstruct{pmix_proc_t} structure is used to identify a single process in the PMIx universe.
It contains a reference to the namespace and the \refstruct{pmix_rank_t} within that namespace.

\cspecificstart
\begin{codepar}
typedef struct pmix_proc \{
    char nspace[PMIX_MAX_NSLEN+1];
    pmix_rank_t rank;
\} pmix_proc_t;
\end{codepar}
\cspecificend


%%%%%%%%%%%
\subsection{Process State Structure}
\label{api:struct:processstate}
\declarestruct{pmix_proc_state_t}

The \refstruct{pmix_proc_state_t} structure is a \code{uint8_t} type for process state values.

The following constants can be used to set a variable of the type \refstruct{pmix_proc_state_t}.


\begin{constantdesc}
%
\declareconstitemvalue{PMIX_PROC_STATE_UNDEF}{0}
Undefined process state
%
\declareconstitemvalue{PMIX_PROC_STATE_PREPPED}{1}
Process is ready to be launched
%
\declareconstitemvalue{PMIX_PROC_STATE_LAUNCH_UNDERWAY}{2}
Process launch is underway
%
\declareconstitemvalue{PMIX_PROC_STATE_RESTART}{3}
Process is ready for restart
%
\declareconstitemvalue{PMIX_PROC_STATE_TERMINATE}{4}
Process is marked for termination
%
\declareconstitemvalue{PMIX_PROC_STATE_RUNNING}{5}
Process has been locally \code{fork}'ed by the \ac{RM}
%
\declareconstitemvalue{PMIX_PROC_STATE_CONNECTED}{6}
Process has connected to PMIx server
%
\declareconstitemvalue{PMIX_PROC_STATE_UNTERMINATED}{15}
Define a ``boundary'' between this constant and \refconst{PMIX_PROC_STATE_CONNECTED} so users can easily and quickly determine if a process is still running or not.
Any value less than this constant means that the process has not terminated.
%
\declareconstitemvalue{PMIX_PROC_STATE_TERMINATED}{20}
Process has terminated and is no longer running
%
\declareconstitemvalue{PMIX_PROC_STATE_ERROR}{50}
Define a boundary so users can easily and quickly determine if a process abnormally terminated.
Any value above this constant means that the process has terminated abnormally.
%
\declareconstitemvalue{PMIX_PROC_STATE_KILLED_BY_CMD}{PMIX_PROC_STATE_ERROR+1}
Process was killed by a command
%
\declareconstitemvalue{PMIX_PROC_STATE_ABORTED}{PMIX_PROC_STATE_ERROR+2}
Process was aborted
%
\declareconstitemvalue{PMIX_PROC_STATE_FAILED_TO_START}{PMIX_PROC_STATE_ERROR+3}
Process failed to start
%
\declareconstitemvalue{PMIX_PROC_STATE_ABORTED_BY_SIG}{PMIX_PROC_STATE_ERROR+4}
Process aborted by a signal
%
\declareconstitemvalue{PMIX_PROC_STATE_TERM_WO_SYNC}{PMIX_PROC_STATE_ERROR+5}
Process exited without calling \refapi{PMIx_Finalize}
%
\declareconstitemvalue{PMIX_PROC_STATE_COMM_FAILED}{PMIX_PROC_STATE_ERROR+6}
Process communication has failed
%
\declareconstitemvalue{PMIX_PROC_STATE_CALLED_ABORT}{PMIX_PROC_STATE_ERROR+7}
Process called \refapi{PMIx_Abort}
%
\declareconstitemvalue{PMIX_PROC_STATE_MIGRATING}{PMIX_PROC_STATE_ERROR+8}
Process failed and is waiting for resources before restarting
%
\declareconstitemvalue{PMIX_PROC_STATE_CANNOT_RESTART}{PMIX_PROC_STATE_ERROR+9}
Process failed and cannot be restarted
%
\declareconstitemvalue{PMIX_PROC_STATE_TERM_NON_ZERO}{PMIX_PROC_STATE_ERROR+10}
Process exited with a non-zero status
%
\declareconstitemvalue{PMIX_PROC_STATE_FAILED_TO_LAUNCH}{PMIX_PROC_STATE_ERROR+11}
Unable to launch process
%
\end{constantdesc}


%%%%%%%%%%%
\subsection{Process Information Structure}
\declarestruct{pmix_proc_info_t}

The \refstruct{pmix_proc_info_t} structure defines a set of information about a specific process including it's name, location, and state.

\cspecificstart
\begin{codepar}
typedef struct pmix_proc_info \{
    /** Process structure */
    pmix_proc_t proc;
    /** Hostname where process resides */
    char *hostname;
    /** Name of the executable */
    char *executable_name;
    /** Process ID on the host */
    pid_t pid;
    /** Exit code of the process. Default: 0 */
    int exit_code;
    /** Current state of the process */
    pmix_proc_state_t state;
\} pmix_proc_info_t;
\end{codepar}
\cspecificend


%%%%%%%%%%%
\subsection{Scope of Put Data}
\declarestruct{pmix_scope_t}

The \refstruct{pmix_scope_t} structure is a \code{uint8_t} type that defines the scope for data passed to \refapi{PMIx_Put}.
The following constants can be used to set a variable of the type \refstruct{pmix_scope_t}.

Specific implementations may support different scope values, but all implementations must support at least \refconst{PMIX_GLOBAL}.
If a scope value is not supported, then the \refapi{PMIx_Put} call must return \refconst{PMIX_ERR_NOT_SUPPORTED}.

\begin{constantdesc}
%
\declareconstitem{PMIX_SCOPE_UNDEF}
Undefined scope
%
\declareconstitem{PMIX_LOCAL}
The data is intended only for other application processes on the same node.
Data marked in this way will not be included in data packages sent to remote requestors --- i.e., it is only available to processes on the local node.
%
\declareconstitem{PMIX_REMOTE}
The data is intended solely for applications processes on remote nodes.
Data marked in this way will not be shared with other processes on the same node --- i.e., it is only available to  processes on remote nodes.
%
\declareconstitem{PMIX_GLOBAL}
The data is to be shared with all other requesting processes, regardless of location.
%
\declareconstitem{PMIX_INTERNAL}
The data is intended solely for this process and is not shared with other processes.
%
\end{constantdesc}



%%%%%%%%%%%
\subsection{Scope of Published Data}
\declarestruct{pmix_data_range_t}

The \refstruct{pmix_data_range_t} structure is a \code{uint8_t} type that defines a range for ``published'' data.
The following constants can be used to set a variable of the type \refstruct{pmix_data_range_t}.

\begin{constantdesc}
%
\declareconstitem{PMIX_RANGE_UNDEF}
Undefined range
%
\declareconstitem{PMIX_RANGE_RM}
Data is intended for the host resource manager.
%
\declareconstitem{PMIX_RANGE_LOCAL}
Data is only available to processes on the local node.
%
\declareconstitem{PMIX_RANGE_NAMESPACE}
Data is only available to processes in the same namespace.
%
\declareconstitem{PMIX_RANGE_SESSION}
Data is only available to all processes in the session.
%
\declareconstitem{PMIX_RANGE_GLOBAL}
Data is available to all processes.
%
\declareconstitem{PMIX_RANGE_CUSTOM}
Range is specified in the \refstruct{pmix_info_t} associated with this call.
%
\declareconstitem{PMIX_RANGE_PROC_LOCAL}
Data is only available to this process.
%
\end{constantdesc}



%%%%%%%%%%%
\subsection{Data Persistence Structure}
\declarestruct{pmix_persistence_t}

The \refstruct{pmix_persistence_t} structure is a \code{uint8_t} type that defines the policy for data published by clients via the \refapi{PMIx_Publish} \ac{API}.
The following constants can be used to set a variable of the type \refstruct{pmix_persistence_t}.

\begin{constantdesc}
%
\declareconstitem{PMIX_PERSIST_INDEF}
Retain data until specifically deleted.
%
\declareconstitem{PMIX_PERSIST_FIRST_READ}
Retain data until the first access, then the data is deleted.
%
\declareconstitem{PMIX_PERSIST_PROC}
Retain data until the publishing process terminates.
%
\declareconstitem{PMIX_PERSIST_APP}
Retain data until the application terminates.
%
\declareconstitem{PMIX_PERSIST_SESSION}
Retain data until the session/allocation terminates.
%
\end{constantdesc}



%%%%%%%%%%%
\subsection{Info Type Directives}
\declarestruct{pmix_info_directives_t}

The \refstruct{pmix_info_directives_t} structure is a \code{uint32_t} type that defines the behavior of command directives via \refstruct{pmix_info_t} arrays.
By default, the values in the \refstruct{pmix_info_t} array passed to a PMIx are \emph{optional}.

\adviceuserstart
A PMIx implementation or PMIx-enabled \ac{RM} may ignore any \refstruct{pmix_info_t} value passed to a PMIx API if it is not explicitly marked as \refconst{PMIX_INFO_REQD}.
This is because the values specified default to optional, meaning they can be ignored.
This may lead to unexpected behavior if the user is relying on the behavior specified by the \refstruct{pmix_info_t} value.
If the user relies on the behavior defined by the \refstruct{pmix_info_t} then they must pass \refconst{PMIX_INFO_REQD} --- the PMIx implementation or \ac{RM} must return \refconst{PMIX_ERR_NOT_SUPPORTED} as soon as possible to the caller if the requested behavior is not supported.
\adviceuserend

The following constants can be used to set a variable of the type \refstruct{pmix_info_directives_t}.

\begin{constantdesc}
%
\declareconstitemvalue{PMIX_INFO_REQD}{0x0001}
The behavior defined in the \refstruct{pmix_info_t} array is required, and not optional.
%
\end{constantdesc}



%%%%%%%%%%%
\subsection{Job Allocation Directives}
\declarestruct{pmix_alloc_directive_t}

The \refstruct{pmix_alloc_directive_t} structure is a \code{uint8_t} type that defines the behavior of allocation requests.
The following constants can be used to set a variable of the type \refstruct{pmix_alloc_directive_t}.

\begin{constantdesc}
%
\declareconstitem{PMIX_ALLOC_NEW}
A new allocation is being requested.
The resulting allocation will be disjoint (i.e., not connected in a job sense) from the requesting allocation.
%
\declareconstitem{PMIX_ALLOC_EXTEND}
Extend the existing allocation, either in time or as additional resources.
%
\declareconstitem{PMIX_ALLOC_RELEASE}
Release part of the existing allocation.
Attributes in the accompanying \refstruct{pmix_info_t} array may be used to specify permanent release of the identified resources, or ``lending'' of those resources for some period of time.
%
\declareconstitem{PMIX_ALLOC_REAQUIRE}
Reacquire resources that were previously ``lent'' back to the scheduler.
%
\declareconstitemvalue{PMIX_ALLOC_EXTERNAL}{128}
A value boundary above which implementers are free to define their own directive values.
%
\end{constantdesc}



%%%%%%%%%%%
\subsection{Lookup Returned Data Structure}
\declarestruct{pmix_pdata_t}

The \refstruct{pmix_pdata_t} structure is used by \refapi{PMIx_Lookup} to describe the data being accessed.

\cspecificstart
\begin{codepar}
typedef struct pmix_pdata \{
    pmix_proc_t proc;
    char key[PMIX_MAX_KEYLEN+1];
    pmix_value_t value;
\} pmix_pdata_t;
\end{codepar}
\cspecificend


%%%%%%%%%%%
\subsection{Application Structure}
\declarestruct{pmix_app_t}

The \refstruct{pmix_app_t} structure describes the application context for the \refapi{PMIx_Spawn} and \refapi{PMIx_Spawn_nb} operations.

\cspecificstart
\begin{codepar}
typedef struct pmix_app \{
    /** Executable */
    char *cmd;
    /** Argument set, NULL terminated */
    char **argv;
    /** Environment set, NULL terminated */
    char **env;
    /** Current working directory */
    char *cwd;
    /** Maximum processes with this profile */
    int maxprocs;
    /** Array of info keys describing this application*/
    pmix_info_t *info;
    /** Number of info keys in 'info' array */
    size_t ninfo;
\} pmix_app_t;
\end{codepar}
\cspecificend


%%%%%%%%%%%
\subsection{Query Structure}
\declarestruct{pmix_query_t}

The \refstruct{pmix_query_t} structure is used by \refapi{PMIx_Query_info_nb} to describe a single query operation.

\cspecificstart
\begin{codepar}
typedef struct pmix_query \{
    char **keys;
    pmix_info_t *qualifiers;
    size_t nqual;
\} pmix_query_t;
\end{codepar}
\cspecificend


%%%%%%%%%%%
\subsection{Modex Structure}
\declarestruct{pmix_modex_data_t}

The \refstruct{pmix_modex_data_t} structure describes the \ac{BCX} information.

\notestart
\noteheader
This structure has been deprecated and will be removed in future versions of the \ac{PMIx} Standard.
\noteend

\cspecificstart
\begin{codepar}
typedef struct pmix_modex_data \{
    char nspace[PMIX_MAX_NSLEN+1];
    int rank;
    uint8_t *blob;
    size_t size;
\} pmix_modex_data_t;
\end{codepar}
\cspecificend


%%%%%%%%%%%
\section{Data Packing/Unpacking Types and Structures}

This section defines types and structures used to pack and unpack data passed through the PMIx API.

%%%%%%%%%%%
\subsection{Byte Object Type}
\declarestruct{pmix_byte_object_t}

The \refstruct{pmix_byte_object_t} structure describes a raw byte sequence.

\cspecificstart
\begin{codepar}
typedef struct pmix_byte_object \{
    char *bytes;
    size_t size;
\} pmix_byte_object_t;
\end{codepar}
\cspecificend


%%%%%%%%%%%
\subsection{Data Buffer Type}
\declarestruct{pmix_data_buffer_t}

The \refstruct{pmix_data_buffer_t} structure describes a data buffer used for packing and unpacking.

\cspecificstart
\begin{codepar}
typedef struct pmix_data_buffer \{
    /** Start of my memory */
    char *base_ptr;
    /** Where the next data will be packed to (within the allocated
        memory starting at base_ptr) */
    char *pack_ptr;
    /** Where the next data will be unpacked from (within the
        allocated memory starting as base_ptr) */
    char *unpack_ptr;
    /** Number of bytes allocated (starting at base_ptr) */
    size_t bytes_allocated;
    /** Number of bytes used by the buffer (i.e., amount of data --
        including overhead -- packed in the buffer) */
    size_t bytes_used;
\} pmix_data_buffer_t;
\end{codepar}
\cspecificend



%%%%%%%%%%%
\subsection{Data Array Structure}
\declarestruct{pmix_data_array}

The \refstruct{pmix_data_array} structure defines an array data structure.

\cspecificstart
\begin{codepar}
typedef struct pmix_data_array \{
    pmix_data_type_t type;
    size_t size;
    void *array;
\} pmix_data_array_t;
\end{codepar}
\cspecificend



%%%%%%%%%%%
\subsection{Info and Info Array Structures}
\declarestruct{pmix_info_t}
\declarestruct{pmix_info_array}

The \refstruct{pmix_info_t} structure defines a key/value pair with associated directive.

\cspecificstart
\begin{codepar}
typedef struct pmix_info_t \{
    char key[PMIX_MAX_KEYLEN+1];
    pmix_info_directives_t flags;
    pmix_value_t value;
\} pmix_info_t;
\end{codepar}
\cspecificend

The \refstruct{pmix_info_array} structure defines an array of \refstruct{pmix_info_t} structures.

\notestart
\noteheader
The \refstruct{pmix_info_array} structure has been deprecated and will be removed in future versions of the \ac{PMIx} Standard.
\noteend

\cspecificstart
\begin{codepar}
typedef struct pmix_info_array \{
    size_t size;
    pmix_info_t *array;
\} pmix_info_array_t;
\end{codepar}
\cspecificend


%%%%%%%%%%%
\subsection{Generalized Data Types Used for Packing/Unpacking}
\declarestruct{pmix_data_type_t}

The \refstruct{pmix_data_type_t} structure is a \code{uint16_t} type for identifying the data type for packing/unpacking purposes.

The following constants can be used to set a variable of the type \refstruct{pmix_data_type_t}.
The C language type is specified for clarity of intent.
A PMIx implementation, especially those in other languages, may choose types more appropriate for their implementation.
Additionally, a PMIx implementation may choose to add additional types.

\begin{constantdesc}
%
\declareconstitem{PMIX_UNDEF}
Undefined
%
\declareconstitem{PMIX_BOOL}
Boolean (converted to/from native \code{true}/\code{false}) (\code{bool})
%
\declareconstitem{PMIX_BYTE}
A byte of data (\code{uint8_t})
%
\declareconstitem{PMIX_STRING}
\code{NUL} terminated string (\code{char*})
%
\declareconstitem{PMIX_SIZE}
Size \code{size_t}
%
\declareconstitem{PMIX_PID}
Operating \ac{PID} (\code{pid_t})
%
\declareconstitem{PMIX_INT}
Integer (\code{int})
%
\declareconstitem{PMIX_INT8}
8-byte integer (\code{int8_t})
%
\declareconstitem{PMIX_INT16}
16-byte integer (\code{int16_t})
%
\declareconstitem{PMIX_INT32}
32-byte integer (\code{int32_t})
%
\declareconstitem{PMIX_INT64}
64-byte integer (\code{int64_t})
%
\declareconstitem{PMIX_UINT}
Unsigned integer (\code{unsigned int})
%
\declareconstitem{PMIX_UINT8}
Unsigned 8-byte integer (\code{uint8_t})
%
\declareconstitem{PMIX_UINT16}
Unsigned 16-byte integer (\code{uint16_t})
%
\declareconstitem{PMIX_UINT32}
Unsigned 32-byte integer (\code{uint32_t})
%
\declareconstitem{PMIX_UINT64}
Unsigned 64-byte integer (\code{uint64_t})
%
\declareconstitem{PMIX_FLOAT}
Float (\code{float})
%
\declareconstitem{PMIX_DOUBLE}
Double (\code{double})
%
\declareconstitem{PMIX_TIMEVAL}
Time value (\code{struct timeval})
%
\declareconstitem{PMIX_TIME}
Time (\code{time_t})
%
\declareconstitem{PMIX_STATUS}
Status (\refstruct{pmix_status_t}).
Note that this needs to be tracked separately from integer for those times when we are embedded and it needs to be converted to the host error definitions.
%
\declareconstitem{PMIX_VALUE}
Value (\refstruct{pmix_value_t})
%
\declareconstitem{PMIX_PROC}
Process (\refstruct{pmix_proc_t})
%
\declareconstitem{PMIX_APP}
Application context
%
\declareconstitem{PMIX_INFO}
Info object
%
\declareconstitem{PMIX_PDATA}
Pointer to data
%
\declareconstitem{PMIX_BUFFER}
Buffer
%
\declareconstitem{PMIX_BYTE_OBJECT}
Byte object (\refstruct{pmix_byte_object_t})
%
\declareconstitem{PMIX_KVAL}
Key/value pair
%
\declareconstitemDEP{PMIX_MODEX}{2.0}
Modex
%
\declareconstitem{PMIX_PERSIST}
Persistance (\refstruct{pmix_persistence_t})
%
\declareconstitem{PMIX_POINTER}
Pointer (\code{void*})
%
\declareconstitem{PMIX_SCOPE}
Scope (\refstruct{pmix_scope_t})
%
\declareconstitem{PMIX_DATA_RANGE}
Data range (\refstruct{pmix_data_range_t})
%
\declareconstitem{PMIX_COMMAND}
Command
%
\declareconstitem{PMIX_INFO_DIRECTIVES}
Info directives
%
\declareconstitem{PMIX_DATA_TYPE}
Data type
%
\declareconstitem{PMIX_PROC_STATE}
Process state (\refstruct{pmix_proc_state_t})
%
\declareconstitem{PMIX_PROC_INFO}
Process info (\refstruct{pmix_proc_info_t})
%
\declareconstitem{PMIX_DATA_ARRAY}
Data array (\refstruct{pmix_data_array_t})
%
\declareconstitem{PMIX_PROC_RANK}
Process rank (\refstruct{pmix_rank_t})
%
\declareconstitem{PMIX_QUERY}
Query
%
\declareconstitem{PMIX_COMPRESSED_STRING}
Compressed string (with zlib)
%
\declareconstitem{PMIX_ALLOC_DIRECTIVE}
Allocation directive (\refstruct{pmix_alloc_directive_t})
%
\declareconstitemDEP{PMIX_INFO_ARRAY}{2.0}
Info array
%
\declareconstitemvalue{PMIX_DATA_TYPE_MAX}{500}
A boundary for implementers so they can add their own data types.
%
\end{constantdesc}


%%%%%%%%%%%
\subsection{Value Structure}
\declarestruct{pmix_value_t}

The \refstruct{pmix_value_t} structure is used to represent the value passed to \refapi{PMIx_Put} and retrieved by \refapi{PMIx_Get}, as well as many of the other \ac{PMIx} functions.

A collection of values may be specified under a single key by passing a \refstruct{pmix_value_t} containing an array of type \refstruct{pmix_data_array_t}, with each array element containing its own object.

\cspecificstart
\begin{codepar}
typedef struct pmix_value \{
    pmix_data_type_t type;
    union \{
        bool flag;
        uint8_t byte;
        char *string;
        size_t size;
        pid_t pid;
        int integer;
        int8_t int8;
        int16_t int16;
        int32_t int32;
        int64_t int64;
        unsigned int uint;
        uint8_t uint8;
        uint16_t uint16;
        uint32_t uint32;
        uint64_t uint64;
        float fval;
        double dval;
        struct timeval tv;
        time_t time;
        pmix_status_t status;
        pmix_rank_t rank;
        pmix_proc_t *proc;
        pmix_byte_object_t bo;
        pmix_persistence_t persist;
        pmix_scope_t scope;
        pmix_data_range_t range;
        pmix_proc_state_t state;
        pmix_proc_info_t *pinfo;
        pmix_data_array_t *darray;
        void *ptr;
        pmix_alloc_directive_t adir;
        /**** DEPRECATED in PMIx 2 ****/
        pmix_info_array_t *array;
        /******************************/
    \} data;
\} pmix_value_t;
\end{codepar}
\cspecificend


%%%%%%%%%%%
\subsubsection{Value Structure Helper Functions}
\declareapi{pmix_value_load}

The functions defined in this section are convenience operations to help in loading data into, and copying data between \refstruct{pmix_value_t} structures.

%%%%
\format

\cspecificstart
\begin{codepar}
void
pmix_value_load(pmix_value_t *v,
                const void *data, pmix_data_type_t type);
\end{codepar}
\cspecificend

\begin{arglist}
\arginout{v}{value type \refstruct{pmix_value_t} (handle)}
\argin{data}{pointer to data (pointer)}
\argin{type}{Type of the specified \refarg{data} defined by \refstruct{pmix_data_type_t} (handle)}
\end{arglist}

%%%%
\descr

Load \refarg{data} into the value type \refarg{v}.

%%%%
\format

\cspecificstart
\begin{codepar}
pmix_status_t
pmix_value_xfer(pmix_value_t *kv, pmix_value_t *src);
\end{codepar}
\cspecificend

\begin{arglist}
\argout{kv}{key/value type \refstruct{pmix_value_t} destination (handle)}
\argin{src}{value type \refstruct{pmix_value_t} source (handle)}
\end{arglist}

%%%%
\descr

Copy the values stored in \refarg{src} into \refarg{kv}.



%%%%%%%%%%%
\section{Reserved attributes}
\label{api:struct:attributes}

The PMIx standard defines a relatively small set of APIs and the caller may customize the behavior of the API by passing one or more attributes to that API.
Additionally, attributes may be keys passed to \refapi{PMIx_Get} calls to access the specified values from the system.

Each attribute is represented by a \var{key} string, and a type for the associated \var{value}.
This section defines a set of \textbf{reserved} keys which are prefixed with \code{pmix.} to designate them as PMIx standard reserved keys.
The application or associated libraries (e.g., \ac{MPI}) may choose to define additional attributes
The attributes defined in this section are of the system and job as opposed to the attributes that the application (or associated libraries) might choose to expose.
Due to this extensibility the \refapi{PMIx_Get} API will return \refconst{PMIX_ERR_NOT_FOUND} if the provided \var{key} cannot be found.


\declareAttribute{PMIX_ATTR_UNDEF}{NULL}{NULL}{
Constant representing an undefined attribute.
}

%%%%%%%%%%%
\subsection{Initialization attributes}
\label{api:struct:attributes:init}

These attributes are defined to assist the caller with initialization.

%
\declareAttribute{PMIX_EVENT_BASE}{"pmix.evbase"}{struct event_base *}{
Pointer to libevent\footnote{\url{http://libevent.org/}} \code{event_base} to use in place of the internal progress thread.
}

%
\declareAttribute{PMIX_SERVER_TOOL_SUPPORT}{"pmix.srvr.tool"}{bool}{
The host \ac{RM} wants to declare itself as willing to accept tool connection requests.
}

%
\declareAttribute{PMIX_SERVER_REMOTE_CONNECTIONS}{"pmix.srvr.remote"}{bool}{
Allow connections from remote tools. Forces the PMIx server to not exclusively use loopback device.
}

%
\declareAttribute{PMIX_SERVER_SYSTEM_SUPPORT}{"pmix.srvr.sys"}{bool}{
The host \ac{RM} wants to declare itself as being the local system server for PMIx connection requests.
}

%
\declareAttribute{PMIX_SERVER_TMPDIR}{"pmix.srvr.tmpdir"}{char*}{
Top-level temporary directory for all \emph{client} processes connected to this server, and where the PMIx server will place its \emph{tool} rendezvous point and contact information.
}

%
\declareAttribute{PMIX_SYSTEM_TMPDIR}{"pmix.sys.tmpdir"}{char*}{
Temporary directory for this system, and where a PMIx server that declares itself to be a system-level server will place a \emph{tool} rendezvous point and contact information.
}

%
\declareAttribute{PMIX_REGISTER_NODATA}{"pmix.reg.nodata"}{bool}{
Registration is for the namespace only. Do not copy job data.
}

%
\declareAttribute{PMIX_SERVER_ENABLE_MONITORING}{"pmix.srv.monitor"}{bool}{
Enable PMIx internal monitoring by the PMIx server.
}

%
\declareAttribute{PMIX_SERVER_NSPACE}{"pmix.srv.nspace"}{char*}{
Name of the namespace to use for this PMIx server.
}

%
\declareAttribute{PMIX_SERVER_RANK}{"pmix.srv.rank"}{pmix_rank_t}{
Rank of this PMIx server
}


%%%%%%%%%%%
\subsection{Tool-related attributes}
\label{api:struct:attributes:tool}

These attributes are defined to assist PMIx-enabled tools to connect with the PMIx server.

%
\declareAttribute{PMIX_TOOL_NSPACE}{"pmix.tool.nspace"}{char*}{
Name of the namespace to use for this tool.
}

%
\declareAttribute{PMIX_TOOL_RANK}{"pmix.tool.rank"}{uint32_t}{
Rank of this tool.
}

%
\declareAttribute{PMIX_SERVER_PIDINFO}{"pmix.srvr.pidinfo"}{pid_t}{
\ac{PID} of the target PMIx server for a tool.
}

%
\declareAttribute{PMIX_CONNECT_TO_SYSTEM}{"pmix.cnct.sys"}{bool}{
The requestor requires that a connection be made only to a local, system-level PMIx server.
}

%
\declareAttribute{PMIX_CONNECT_SYSTEM_FIRST}{"pmix.cnct.sys.first"}{bool}{
Preferentially, look for a system-level PMIx server first.
}

%
\declareAttribute{PMIX_SERVER_URI}{"pmix.srvr.uri"}{char*}{
\ac{URI} of the PMIx server to be contacted.
}

%
\declareAttribute{PMIX_SERVER_HOSTNAME}{"pmix.srvr.host"}{char*}{
Host where target PMIx server is located.
}

%
\declareAttribute{PMIX_CONNECT_MAX_RETRIES}{"pmix.tool.mretries"}{uint32_t}{
Maximum number of times to try to connect to PMIx server.
}

%
\declareAttribute{PMIX_CONNECT_RETRY_DELAY}{"pmix.tool.retry"}{uint32_t}{
Time in seconds between connection attempts to a PMIx server.
}

%
\declareAttribute{PMIX_TOOL_DO_NOT_CONNECT}{"pmix.tool.nocon"}{bool}{
The tool wants to use internal PMIx support, but does not want to connect to a PMIx server.
}


%%%%%%%%%%%
\subsection{Identification attributes}
\label{api:struct:attributes:ident}

These attributes are defined to identify a process and it's associated PMIx-enabled library.

%
\declareAttribute{PMIX_USERID}{"pmix.euid"}{uint32_t}{
Effective user id.
}

%
\declareAttribute{PMIX_GRPID}{"pmix.egid"}{uint32_t}{
Effective group id.
}

%
\declareAttribute{PMIX_DSTPATH}{"pmix.dstpath"}{char*}{
Path to shared memory data storage (dstore) files.
}

%
\declareAttribute{PMIX_VERSION_INFO}{"pmix.version"}{char*}{
PMIx version of contractor.
}

%
\declareAttribute{PMIX_PROGRAMMING_MODEL}{"pmix.pgm.model"}{char*}{
Programming model being initialized (e.g., ``MPI'' or ``OpenMP'')
}

%
\declareAttribute{PMIX_MODEL_LIBRARY_NAME}{"pmix.mdl.name"}{char*}{
Programming model implementation ID (e.g., ``OpenMPI'' or ``MPICH'')
}

%
\declareAttribute{PMIX_MODEL_LIBRARY_VERSION}{"pmix.mld.vrs"}{char*}{
Programming model version string (e.g., ``2.1.1'')
}

%
\declareAttribute{PMIX_THREADING_MODEL}{"pmix.threads"}{char*}{
Threading model used (e.g., ``pthreads'')
}

%
\declareAttribute{PMIX_REQUESTOR_IS_TOOL}{"pmix.req.tool"}{bool}{
The requesting process is a PMIx tool.
}

%
\declareAttribute{PMIX_REQUESTOR_IS_CLIENT}{"pmix.req.client"}{bool}{
The requesting process is a PMIx client.
}


%%%%%%%%%%%
\subsection{UNIX socket rendezvous socket attributes}
\label{api:struct:attributes:usock}

These attributes are used to describe a UNIX socket for rendezvous with the local \ac{RM}.

%
\declareAttribute{PMIX_USOCK_DISABLE}{"pmix.usock.disable"}{bool}{
Disable legacy UNIX socket (usock) support
}

%
\declareAttribute{PMIX_SOCKET_MODE}{"pmix.sockmode"}{uint32_t}{
POSIX \var{mode_t} (9 bits valid)
}

%
\declareAttribute{PMIX_SINGLE_LISTENER}{"pmix.sing.listnr"}{bool}{
Use only one rendezvous socket, letting priorities and/or environment parameters select the active transport.
}


%%%%%%%%%%%
\subsection{TCP connection attributes}
\label{api:struct:attributes:tcp}

These attributes are used to describe a TCP socket for rendezvous with the local \ac{RM}.

%
\declareAttribute{PMIX_TCP_REPORT_URI}{"pmix.tcp.repuri"}{char*}{
If provided, directs that the TCP \ac{URI} be reported and indicates the desired method of reporting: \code{'-'} for stdout, \code{'+'} for stderr, or filename.
}

%
\declareAttribute{PMIX_TCP_URI}{"pmix.tcp.uri"}{char*}{
The \ac{URI} of the PMIx server to connect to, or a file name containing it in the form of \code{file:<name of file containing it>}.
}

%
\declareAttribute{PMIX_TCP_IF_INCLUDE}{"pmix.tcp.ifinclude"}{char*}{
Comma-delimited list of devices and/or \ac{CIDR} notation to include when establishing the TCP connection.
}

%
\declareAttribute{PMIX_TCP_IF_EXCLUDE}{"pmix.tcp.ifexclude"}{char*}{
Comma-delimited list of devices and/or \ac{CIDR} notation to exclude when establishing the TCP connection.
}

%
\declareAttribute{PMIX_TCP_IPV4_PORT}{"pmix.tcp.ipv4"}{int}{
The IPv4 port to be used.
}

%
\declareAttribute{PMIX_TCP_IPV6_PORT}{"pmix.tcp.ipv6"}{int}{
The IPv6 port to be used.
}

%
\declareAttribute{PMIX_TCP_DISABLE_IPV4}{"pmix.tcp.disipv4"}{bool}{
Set to \code{true} to disable IPv4 family of addresses.
}

%
\declareAttribute{PMIX_TCP_DISABLE_IPV6}{"pmix.tcp.disipv6"}{bool}{
Set to \code{true} to disable IPv6 family of addresses.
}


%%%%%%%%%%%
\subsection{Global Data Storage (GDS) attributes}
\label{api:struct:attributes:gds}

These attributes are used to define the behavior of the \ac{GDS} used to manage key/value pairs.

%
\declareAttribute{PMIX_GDS_MODULE}{"pmix.gds.mod"}{char*}{
Comma-delimited string of desired modules.
}


%%%%%%%%%%%
\subsection{General process-level attributes}
\label{api:struct:attributes:gproc}

These attributes are used to define process attributes.

%
\declareAttribute{PMIX_CPUSET}{"pmix.cpuset"}{char*}{
hwloc\footnote{\url{https://www.open-mpi.org/projects/hwloc/}} bitmap to be applied to the process upon launch.
}

%
\declareAttribute{PMIX_CREDENTIAL}{"pmix.cred"}{char*}{
Security credential assigned to the process.
}

%
\declareAttribute{PMIX_SPAWNED}{"pmix.spawned"}{bool}{
\code{true} if this process resulted from a call to \refapi{PMIx_Spawn}.
}

%
\declareAttribute{PMIX_ARCH}{"pmix.arch"}{uint32_t}{
Architecture flag.
}


%%%%%%%%%%%
\subsection{Scratch directory attributes}
\label{api:struct:attributes:scratchdir}

These attributes are used to define an application scratch directory.

%
\declareAttribute{PMIX_TMPDIR}{"pmix.tmpdir"}{char*}{
Full path to the top-level temporary directory assigned to the session.
}

%
\declareAttribute{PMIX_NSDIR}{"pmix.nsdir"}{char*}{
Full path to the temporary directory assigned to the namespace, under \refattr{PMIX_TMPDIR}.
}

%
\declareAttribute{PMIX_PROCDIR}{"pmix.pdir"}{char*}{
Full path to the subdirectory under \refattr{PMIX_NSDIR} assigned to the process.
}

%
\declareAttribute{PMIX_TDIR_RMCLEAN}{"pmix.tdir.rmclean"}{bool}{
Resource Manager will clean session directories
}


%%%%%%%%%%%
\subsection{Relative Rank Descriptive Attributes}
\label{api:struct:attributes:relrankinfo}

These attributes are used to describe information about relative ranks as assigned by the \ac{RM}.

%
\declareAttribute{PMIX_PROCID}{"pmix.procid"}{pmix_proc_t}{
Process identifier
}

%
\declareAttribute{PMIX_NSPACE}{"pmix.nspace"}{char*}{
Namespace of the job.
}

%
\declareAttribute{PMIX_JOBID}{"pmix.jobid"}{char*}{
Job identifier assigned by the scheduler.
}

%
\declareAttribute{PMIX_APPNUM}{"pmix.appnum"}{uint32_t}{
Application number within the job.
}

%
\declareAttribute{PMIX_RANK}{"pmix.rank"}{pmix_rank_t}{
Process rank within the job.
}

%
\declareAttribute{PMIX_GLOBAL_RANK}{"pmix.grank"}{pmix_rank_t}{
Process rank spanning across all jobs in this session.
}

%
\declareAttribute{PMIX_APP_RANK}{"pmix.apprank"}{pmix_rank_t}{
Process rank within this application.
}

%
\declareAttribute{PMIX_NPROC_OFFSET}{"pmix.offset"}{pmix_rank_t}{
Starting global rank of this job.
}

%
\declareAttribute{PMIX_LOCAL_RANK}{"pmix.lrank"}{uint16_t}{
Local rank on this node within this job.
}

%
\declareAttribute{PMIX_NODE_RANK}{"pmix.nrank"}{uint16_t}{
Process rank on this node spanning all jobs.
}

%
\declareAttribute{PMIX_LOCALLDR}{"pmix.lldr"}{pmix_rank_t}{
Lowest rank on this node within this job.
}

%
\declareAttribute{PMIX_APPLDR}{"pmix.aldr"}{pmix_rank_t}{
Lowest rank in this application within this job.
}

%
\declareAttribute{PMIX_PROC_PID}{"pmix.ppid"}{pid_t}{
\ac{PID} of specified process.
}

%
\declareAttribute{PMIX_SESSION_ID}{"pmix.session.id"}{uint32_t}{
Session identifier.
}

%
\declareAttribute{PMIX_NODE_LIST}{"pmix.nlist"}{char*}{
Comma-delimited list of nodes running processes for the specified namespace.
}

%
\declareAttribute{PMIX_ALLOCATED_NODELIST}{"pmix.alist"}{char*}{
Comma-delimited list of all nodes in this allocation regardless of whether or not they currently host processes.
}

%
\declareAttribute{PMIX_HOSTNAME}{"pmix.hname"}{char*}{
Name of the host where the specified process is running.
}

%
\declareAttribute{PMIX_NODEID}{"pmix.nodeid"}{uint32_t}{
Node identifier where the specified process is located, expressed as the node's index (beginning at zero) in an array of nodes comprising the users allocation
}

%
\declareAttribute{PMIX_LOCAL_PEERS}{"pmix.lpeers"}{char*}{
Comma-delimited list of ranks on this node within the specified namespace.
}

%
\declareAttribute{PMIX_LOCAL_PROCS}{"pmix.lprocs"}{pmix_proc_t array}{
Array of \refstruct{pmix_proc_t} of processes on the specified node.
}

%
\declareAttribute{PMIX_LOCAL_CPUSETS}{"pmix.lcpus"}{char*}{
Colon-delimited cpusets of local peers within the specified namespace.
}

%
\declareAttribute{PMIX_PROC_URI}{"pmix.puri"}{char*}{
\ac{URI} containing contact information for a given process.
}

%
\declareAttribute{PMIX_LOCALITY}{"pmix.loc"}{uint16_t}{
Relative locality of two processes.
}

%
\declareAttribute{PMIX_PARENT_ID}{"pmix.parent"}{pmix_proc_t}{
Process identifier of the parent process of the calling process.
}


%%%%%%%%%%%
\subsection{Size information attributes}
\label{api:struct:attributes:sizeinfo}

These attributes are used to describe the size of various dimensions of the PMIx universe.

%
\declareAttribute{PMIX_UNIV_SIZE}{"pmix.univ.size"}{uint32_t}{
Number of processes in this namespace.
}

%
\declareAttribute{PMIX_JOB_SIZE}{"pmix.job.size"}{uint32_t}{
Number of processes in this job.
}

%
\declareAttribute{PMIX_JOB_NUM_APPS}{"pmix.job.napps"}{uint32_t}{
Number of applications in this job.
}

%
\declareAttribute{PMIX_APP_SIZE}{"pmix.app.size"}{uint32_t}{
Number of processes in this application.
}

%
\declareAttribute{PMIX_LOCAL_SIZE}{"pmix.local.size"}{uint32_t}{
Number of processes in this job on this node.
}

%
\declareAttribute{PMIX_NODE_SIZE}{"pmix.node.size"}{uint32_t}{
Number of processes across all jobs on this node.
}

%
\declareAttribute{PMIX_MAX_PROCS}{"pmix.max.size"}{uint32_t}{
Maximum number of processes for this job.
}

%
\declareAttribute{PMIX_NUM_NODES}{"pmix.num.nodes"}{uint32_t}{
Number of nodes in this namespace.
}


%%%%%%%%%%%
\subsection{Memory information attributes}
\label{api:struct:attributes:meminfo}

These attributes are used to describe memory available and used in the system.

%
\declareAttribute{PMIX_AVAIL_PHYS_MEMORY}{"pmix.pmem"}{uint64_t}{
Total available physical memory on this node.
}

%
\declareAttribute{PMIX_DAEMON_MEMORY}{"pmix.dmn.mem"}{float}{
Megabytes of memory currently used by the \ac{RM} daemon.
}

%
\declareAttribute{PMIX_CLIENT_AVG_MEMORY}{"pmix.cl.mem.avg"}{float}{
Average Megabytes of memory used by client processes.
}


%%%%%%%%%%%
\subsection{Topology information attributes}
\label{api:struct:attributes:topoinfo}

These attributes are used to describe topology information in the PMIx universe.

%
\declareAttribute{PMIX_NET_TOPO}{"pmix.ntopo"}{char*}{
\ac{XML} representation of the network topology.
}

%
\declareAttribute{PMIX_LOCAL_TOPO}{"pmix.ltopo"}{char*}{
\ac{XML} representation of local node topology.
}

%
\declareAttribute{PMIX_NODE_LIST}{"pmix.nlist"}{char*}{
Comma-delimited list of nodes running processes for this job.
}

%
\declareAttribute{PMIX_TOPOLOGY}{"pmix.topo"}{hwloc_topology_t}{
Pointer to the PMIx client's internal hwloc topology object.
}

%
\declareAttribute{PMIX_TOPOLOGY_SIGNATURE}{"pmix.toposig"}{char*}{
Topology signature string.
}

%
\declareAttribute{PMIX_LOCALITY_STRING}{"pmix.locstr"}{char*}{
String describing a process's bound location.
The string is of the form:\\
\code{NM\%s:SK\%s:L3\%s:L2\%s:L1\%s:CR\%s:HT\%s}\\
Where the \code{\%s} is replaced with an integer index or inclusive range for hwloc.
\code{NM} identifies the numa node(s).
\code{SK} identifies the socket(s).
\code{L3} identifies the L3 cache(s).
\code{L2} identifies the L2 cache(s).
\code{L1} identifies the L1 cache(s).
\code{CR} identifies the cores(s).
\code{HT} identifies the hardware thread(s).
If your architecture does not have the specified hardware designation then it can be omitted from the signature.
\\
For example: \code{NM0:SK0:L30-4:L20-4:L10-4:CR0-4:HT0-39}.\\
This means numa node \code{0}, socket \code{0}, L3 caches \code{0,1,2,3,4}, L2 caches \code{0-4}, L1 caches \code{0-4}, cores \code{0,1,2,3,4}, and hardware threads \code{0-39}.
}

%
\declareAttribute{PMIX_HWLOC_SHMEM_ADDR}{"pmix.hwlocaddr"}{size_t}{
Address of the hwloc shared memory segment.
}

%
\declareAttribute{PMIX_HWLOC_SHMEM_SIZE}{"pmix.hwlocsize"}{size_t}{
Size of the hwloc shared memory segment.
}

%
\declareAttribute{PMIX_HWLOC_SHMEM_FILE}{"pmix.hwlocfile"}{char*}{
Path to the hwloc shared memory file.
}

%
\declareAttribute{PMIX_HWLOC_XML_V1}{"pmix.hwlocxml1"}{char*}{
\ac{XML} representation of local topology using hwloc's v1.x format.
}

%
\declareAttribute{PMIX_HWLOC_XML_V2}{"pmix.hwlocxml2"}{char*}{
\ac{XML} representation of local topology using hwloc's v2.x format.
}


%%%%%%%%%%%
\subsection{Request-related attributes}
\label{api:struct:attributes:request}

These attributes are used to influence the behavior of various PMIx operations.

%
\declareAttribute{PMIX_COLLECT_DATA}{"pmix.collect"}{bool}{
Collect data and return it at the end of the operation.
}

%
\declareAttribute{PMIX_TIMEOUT}{"pmix.timeout"}{int}{
Time in seconds before the specified operation should time out (\var{0} indicating infinite) in error.
The timeout parameter can help avoid ``hangs'' due to programming errors that prevent the target process from ever exposing its data.
}

%
\declareAttribute{PMIX_IMMEDIATE}{"pmix.immediate"}{bool}{
Specified operation should immediately return an error from the PMIx server if the requested data cannot be found - do not request it from the host \ac{RM}.
}

%
\declareAttribute{PMIX_WAIT}{"pmix.wait"}{int}{
Caller requests that the PMIx server wait until at least the specified number of values are found (\var{0} indicates all and is the default).
}

%
\declareAttribute{PMIX_COLLECTIVE_ALGO}{"pmix.calgo"}{char*}{
Comma-delimited list of algorithms to use for the collective operation.
}

%
\declareAttribute{PMIX_COLLECTIVE_ALGO_REQD}{"pmix.calreqd"}{bool}{
If \code{true}, indicates that the requested choice of algorithm is mandatory.
}

%
\declareAttribute{PMIX_NOTIFY_COMPLETION}{"pmix.notecomp"}{bool}{
Notify the parent process upon termination of child job.
}

%
\declareAttribute{PMIX_RANGE}{"pmix.range"}{pmix_data_range_t}{
Value for calls to publish/lookup/unpublish or for monitoring event notifications.
}

%
\declareAttribute{PMIX_PERSISTENCE}{"pmix.persist"}{pmix_persistence_t}{
Value for calls to \refapi{PMIx_Publish}.
}

%
\declareAttribute{PMIX_DATA_SCOPE}{"pmix.scope"}{pmix_scope_t}{
Scope of the data to be found in a \refapi{PMIx_Get} call.
}

%
\declareAttribute{PMIX_OPTIONAL}{"pmix.optional"}{bool}{
Look only in the client's local data store for the requested value - do not request data from the PMIx server if not found.
}

%
\declareAttribute{PMIX_EMBED_BARRIER}{"pmix.embed.barrier"}{bool}{
Execute a blocking fence operation before executing the specified operation.
By default, \refapi{PMIx_Finalize} does not include an internal barrier operation.
This attribute directs \refapi{PMIx_Finalize} to execute a barrier as part of the finalize operation.
}

%
\declareAttribute{PMIX_JOB_TERM_STATUS}{"pmix.job.term.status"}{pmix_status_t}{
Status to be returned upon job termination.
}

%
\declareAttribute{PMIX_PROC_STATE_STATUS}{"pmix.proc.state"}{pmix_proc_state_t}{
Process state
}


%%%%%%%%%%%
\subsection{Sever-to-PMIx library attributes}
\label{api:struct:attributes:server2cl}

Attributes used by the host PMIx server to pass data to the PMIx server's PMIx library.
The data will then be parsed and provided to the local PMIx clients.

%
\declareAttribute{PMIX_REGISTER_NODATA}{"pmix.reg.nodata"}{bool}{
Registration is for this namespace only, do not copy job data.
}

%
\declareAttribute{PMIX_PROC_DATA}{"pmix.pdata"}{pmix_data_array_t}{
Array of process data. Starts with rank, then contains more data.
}

%
\declareAttribute{PMIX_NODE_MAP}{"pmix.nmap"}{char*}{
Regular expression of nodes containing processes for this job.
}

%
\declareAttribute{PMIX_PROC_MAP}{"pmix.pmap"}{char*}{
Regular expression describing processes on each node within this job.
}

%
\declareAttribute{PMIX_ANL_MAP}{"pmix.anlmap"}{char*}{
Process mapping in Argonne National Laboratory's PMI-1/PMI-2 notation.
}

%
\declareAttribute{PMIX_APP_MAP_TYPE}{"pmix.apmap.type"}{char*}{
Type of mapping used to layout the application (e.g., \code{cyclic}).
}

%
\declareAttribute{PMIX_APP_MAP_REGEX}{"pmix.apmap.regex"}{char*}{
Regular expression describing the result of the process mapping.
}


%%%%%%%%%%%
\subsection{Sever-to-Client attributes}
\label{api:struct:attributes:server2client}

Attributes used internally to communicate data from the PMIx server to the PMIx client.

%
\declareAttribute{PMIX_PROC_BLOB}{"pmix.pblob"}{pmix_byte_object_t}{
Packed blob of process data.
}

%
\declareAttribute{PMIX_MAP_BLOB}{"pmix.mblob"}{pmix_byte_object_t}{
Packed blob of process location.
}


%%%%%%%%%%%
\subsection{Event handler registration and notification attributes}
\label{api:struct:attributes:event}

Attributes to support event registration and notification.

%
\declareAttribute{PMIX_EVENT_HDLR_NAME}{"pmix.evname"}{char*}{
String name identifying this handler.
}

%
\declareAttribute{PMIX_EVENT_JOB_LEVEL}{"pmix.evjob"}{bool}{
Register for job-specific events only.
}

%
\declareAttribute{PMIX_EVENT_ENVIRO_LEVEL}{"pmix.evenv"}{bool}{
Register for environment events only.
}

%
\declareAttribute{PMIX_EVENT_HDLR_FIRST}{"pmix.evfirst"}{bool}{
Invoke this event handler before any other handlers.
}

%
\declareAttribute{PMIX_EVENT_HDLR_LAST}{"pmix.evlast"}{bool}{
Invoke this event handler after all other handlers have been called.
}

%
\declareAttribute{PMIX_EVENT_HDLR_FIRST_IN_CATEGORY}{"pmix.evfirstcat"}{bool}{
Invoke this event handler before any other handlers in this category.
}

%
\declareAttribute{PMIX_EVENT_HDLR_LAST_IN_CATEGORY}{"pmix.evlastcat"}{bool}{
Invoke this event handler after all other handlers in this category have been called.
}

%
\declareAttribute{PMIX_EVENT_HDLR_BEFORE}{"pmix.evbefore"}{char*}{
Put this event handler immediately before the one specified in the \code{(char*)} value.
}

%
\declareAttribute{PMIX_EVENT_HDLR_AFTER}{"pmix.evafter"}{char*}{
Put this event handler immediately after the one specified in the \code{(char*)} value.
}

%
\declareAttribute{PMIX_EVENT_HDLR_PREPEND}{"pmix.evprepend"}{bool}{
Prepend this handler to the precedence list within its category.
}

%
\declareAttribute{PMIX_EVENT_HDLR_APPEND}{"pmix.evappend"}{bool}{
Append this handler to the precedence list within its category.
}

%
\declareAttribute{PMIX_EVENT_CUSTOM_RANGE}{"pmix.evrange"}{pmix_data_array_t*}{
Array of \refstruct{pmix_proc_t} defining range of event notification.
}

%
\declareAttribute{PMIX_EVENT_AFFECTED_PROC}{"pmix.evproc"}{pmix_proc_t}{
The single process that was affected.
}

%
\declareAttribute{PMIX_EVENT_AFFECTED_PROCS}{"pmix.evaffected"}{pmix_data_array_t*}{
Array of \refstruct{pmix_proc_t} defining affected processes.
}

%
\declareAttribute{PMIX_EVENT_NON_DEFAULT}{"pmix.evnondef"}{bool}{
Event is not to be delivered to default event handlers.
}

%
\declareAttribute{PMIX_EVENT_RETURN_OBJECT}{"pmix.evobject"}{void *}{
Object to be returned whenever the registered callback function \code{cbfunc} is invoked.
The object will \emph{only} be returned to the process that registered it.
}

%
\declareAttribute{PMIX_EVENT_DO_NOT_CACHE}{"pmix.evnocache"}{bool}{
Instruct the PMIx server not to cache the event.
}

%
\declareAttribute{PMIX_EVENT_SILENT_TERMINATION}{"pmix.evsilentterm"}{bool}{
Do not generate an event when this job normally terminates.
}


%%%%%%%%%%%
\subsection{Fault tolerance attributes}
\label{api:struct:attributes:faulttolerance}

Attributes to support fault tolerance behaviors.

%
\declareAttribute{PMIX_EVENT_TERMINATE_SESSION}{"pmix.evterm.sess"}{bool}{
The \ac{RM} intends to terminate this session.
}

%
\declareAttribute{PMIX_EVENT_TERMINATE_JOB}{"pmix.evterm.job"}{bool}{
The \ac{RM} intends to terminate this job.
}

%
\declareAttribute{PMIX_EVENT_TERMINATE_NODE}{"pmix.evterm.node"}{bool}{
The \ac{RM} intends to terminate all processes on this node.
}

%
\declareAttribute{PMIX_EVENT_TERMINATE_PROC}{"pmix.evterm.proc"}{bool}{
The \ac{RM} intends to terminate just this process.
}

%
\declareAttribute{PMIX_EVENT_ACTION_TIMEOUT}{"pmix.evtimeout"}{int}{
The time in seconds before the \ac{RM} will execute error response.
}

%
\declareAttribute{PMIX_EVENT_NO_TERMINATION}{"pmix.evnoterm"}{bool}{
Indicates that the handler has satisfactorily handled the event and believes termination of the application is not required.
}

%
\declareAttribute{PMIX_EVENT_WANT_TERMINATION}{"pmix.evterm"}{bool}{
Indicates that the handler has determined that the application should be terminated
}


%%%%%%%%%%%
\subsection{Spawn attributes}
\label{api:struct:attributes:spawn}

Attributes used to describe \refapi{PMIx_Spawn} behavior.

%
\declareAttribute{PMIX_PERSONALITY}{"pmix.pers"}{char*}{
Name of personality to use.
}

%
\declareAttribute{PMIX_HOST}{"pmix.host"}{char*}{
Comma-delimited list of hosts to use for spawned processes.
}

%
\declareAttribute{PMIX_HOSTFILE}{"pmix.hostfile"}{char*}{
Hostfile to use for spawned processes.
}

%
\declareAttribute{PMIX_ADD_HOST}{"pmix.addhost"}{char*}{
Comma-delimited list of hosts to add to the allocation.
}

%
\declareAttribute{PMIX_ADD_HOSTFILE}{"pmix.addhostfile"}{char*}{
Hostfile listing hosts to add to existing allocation.
}

%
\declareAttribute{PMIX_PREFIX}{"pmix.prefix"}{char*}{
Prefix to use for starting spawned processes.
}

%
\declareAttribute{PMIX_WDIR}{"pmix.wdir"}{char*}{
Working directory for spawned processes.
}

%
\declareAttribute{PMIX_MAPPER}{"pmix.mapper"}{char*}{
Mapping mechanism to use for placing spawned processes.
}

%
\declareAttribute{PMIX_DISPLAY_MAP}{"pmix.dispmap"}{bool}{
Display process mapping upon spawn.
}

%
\declareAttribute{PMIX_PPR}{"pmix.ppr"}{char*}{
Number of processes to spawn on each identified resource.
}

%
\declareAttribute{PMIX_MAPBY}{"pmix.mapby"}{char*}{
Process mapping policy.
}

%
\declareAttribute{PMIX_RANKBY}{"pmix.rankby"}{char*}{
Process ranking policy.
}

%
\declareAttribute{PMIX_BINDTO}{"pmix.bindto"}{char*}{
Process binding policy.
}

%
\declareAttribute{PMIX_PRELOAD_BIN}{"pmix.preloadbin"}{bool}{
Preload binaries onto nodes.
}

%
\declareAttribute{PMIX_PRELOAD_FILES}{"pmix.preloadfiles"}{char*}{
Comma-delimited list of files to pre-position on nodes.
}

%
\declareAttribute{PMIX_NON_PMI}{"pmix.nonpmi"}{bool}{
Spawned processes will not call \refapi{PMIx_Init}.
}

%
\declareAttribute{PMIX_STDIN_TGT}{"pmix.stdin"}{uint32_t}{
Spawned process rank that is to receive \code{stdin}.
}

%
\declareAttribute{PMIX_FWD_STDIN}{"pmix.fwd.stdin"}{bool}{
Forward this process's \code{stdin} to the designated process.
}

%
\declareAttribute{PMIX_FWD_STDOUT}{"pmix.fwd.stdout"}{bool}{
Forward \code{stdout} from spawned processes to this process.
}

%
\declareAttribute{PMIX_FWD_STDERR}{"pmix.fwd.stderr"}{bool}{
Forward \code{stderr} from spawned processes to this process.
}

%
\declareAttribute{PMIX_DEBUGGER_DAEMONS}{"pmix.debugger"}{bool}{
Spawned application consists of debugger daemons.
}

%
\declareAttribute{PMIX_COSPAWN_APP}{"pmix.cospawn"}{bool}{
Designated application is to be spawned as a disconnected job.
Meaning that it is not part of the ``comm_world'' of the job.
}

%
\declareAttribute{PMIX_SET_SESSION_CWD}{"pmix.ssncwd"}{bool}{
Set the application's current working directory to the session working directory assigned by the \ac{RM}.
}

%
\declareAttribute{PMIX_TAG_OUTPUT}{"pmix.tagout"}{bool}{
Tag application output with the identity of the source process.
}

%
\declareAttribute{PMIX_TIMESTAMP_OUTPUT}{"pmix.tsout"}{bool}{
Timestamp output from applications.
}

%
\declareAttribute{PMIX_MERGE_STDERR_STDOUT}{"pmix.mergeerrout"}{bool}{
Merge \code{stdout} and \code{stderr} streams from application processes.
}

%
\declareAttribute{PMIX_OUTPUT_TO_FILE}{"pmix.outfile"}{char*}{
Output application output to the specified file.
}

%
\declareAttribute{PMIX_INDEX_ARGV}{"pmix.indxargv"}{bool}{
Mark the \code{argv} with the rank of the process.
}

%
\declareAttribute{PMIX_CPUS_PER_PROC}{"pmix.cpuperproc"}{uint32_t}{
Number of cpus to assign to each rank.
}

%
\declareAttribute{PMIX_NO_PROCS_ON_HEAD}{"pmix.nolocal"}{bool}{
Do not place processes on the head node.
}

%
\declareAttribute{PMIX_NO_OVERSUBSCRIBE}{"pmix.noover"}{bool}{
Do not oversubscribe the cpus.
}

%
\declareAttribute{PMIX_REPORT_BINDINGS}{"pmix.repbind"}{bool}{
Report bindings of the individual processes.
}

%
\declareAttribute{PMIX_CPU_LIST}{"pmix.cpulist"}{char*}{
List of cpus to use for this job.
}

%
\declareAttribute{PMIX_JOB_RECOVERABLE}{"pmix.recover"}{bool}{
Application supports recoverable operations.
}

%
\declareAttribute{PMIX_JOB_CONTINUOUS}{"pmix.continuous"}{bool}{
Application is continuous, all failed processes should be immediately restarted.
}

%
\declareAttribute{PMIX_MAX_RESTARTS}{"pmix.maxrestarts"}{uint32_t}{
Maximum number of times to restart a job.
}


%%%%%%%%%%%
\subsection{Query attributes}
\label{api:struct:attributes:query}

Attributes used to describe \refapi{PMIx_Query_info_nb} behavior.

%
\declareAttribute{PMIX_QUERY_NAMESPACES}{"pmix.qry.ns"}{char*}{
Request a comma-delimited list of active namespaces.
}

%
\declareAttribute{PMIX_QUERY_JOB_STATUS}{"pmix.qry.jst"}{pmix_status_t}{
Status of a specified, currently executing job.
}

%
\declareAttribute{PMIX_QUERY_QUEUE_LIST}{"pmix.qry.qlst"}{char*}{
Request a comma-delimited list of scheduler queues.
}

%
\declareAttribute{PMIX_QUERY_QUEUE_STATUS}{"pmix.qry.qst"}{TBD}{
Status of a specified scheduler queue.
}

%
\declareAttribute{PMIX_QUERY_PROC_TABLE}{"pmix.qry.ptable"}{char*}{
Input namespace of the job whose information is being requested returns (\refstruct{pmix_data_array_t}) an array of \refstruct{pmix_proc_info_t}.
}

%
\declareAttribute{PMIX_QUERY_LOCAL_PROC_TABLE}{"pmix.qry.lptable"}{char*}{
Input namespace of the job whose information is being requested returns (\refstruct{pmix_data_array_t}) an array of \refstruct{pmix_proc_info_t} for processes in job on same node.
}

%
\declareAttribute{PMIX_QUERY_AUTHORIZATIONS}{"pmix.qry.auths"}{bool}{
Return operations the PMIx tool is authorized to perform.
}

%
\declareAttribute{PMIX_QUERY_SPAWN_SUPPORT}{"pmix.qry.spawn"}{bool}{
Return a comma-delimited list of supported spawn attributes.
}

%
\declareAttribute{PMIX_QUERY_DEBUG_SUPPORT}{"pmix.qry.debug"}{bool}{
Return a comma-delimited list of supported debug attributes.
}

%
\declareAttribute{PMIX_QUERY_MEMORY_USAGE}{"pmix.qry.mem"}{bool}{
Return information on memory usage for the processes indicated in the qualifiers.
}

%
\declareAttribute{PMIX_QUERY_LOCAL_ONLY}{"pmix.qry.local"}{bool}{
Constrain the query to local information only.
}

%
\declareAttribute{PMIX_QUERY_REPORT_AVG}{"pmix.qry.avg"}{bool}{
Report average values.
}

%
\declareAttribute{PMIX_QUERY_REPORT_MINMAX}{"pmix.qry.minmax"}{bool}{
Report minimum and maximum values.
}

%
\declareAttribute{PMIX_QUERY_ALLOC_STATUS}{"pmix.query.alloc"}{char*}{
String identifier of the allocation whose status is being requested.
}

%
\declareAttribute{PMIX_TIME_REMAINING}{"pmix.time.remaining"}{char*}{
Query number of seconds (\code{uint32_t}) remaining in allocation for the specified namespace.
}


%%%%%%%%%%%
\subsection{Log attributes}
\label{api:struct:attributes:log}

Attributes used to describe \refapi{PMIx_Log_nb} behavior.

%
\declareAttribute{PMIX_LOG_STDERR}{"pmix.log.stderr"}{char*}{
Log string to \code{stderr}.
}

%
\declareAttribute{PMIX_LOG_STDOUT}{"pmix.log.stdout"}{char*}{
Log string to \code{stdout}.
}

%
\declareAttribute{PMIX_LOG_SYSLOG}{"pmix.log.syslog"}{char*}{
Log data to syslog.
Defaults to \code{ERROR} priority.
}

%
\declareAttribute{PMIX_LOG_MSG}{"pmix.log.msg"}{pmix_byte_object_t}{
Message blob to be sent somewhere.
}

%
\declareAttribute{PMIX_LOG_EMAIL}{"pmix.log.email"}{pmix_data_array_t}{
Log via email based on \refstruct{pmix_info_t} containing directives.
}

%
\declareAttribute{PMIX_LOG_EMAIL_ADDR}{"pmix.log.emaddr"}{char*}{
Comma-delimited list of email addresses that are to receive the message.
}

%
\declareAttribute{PMIX_LOG_EMAIL_SUBJECT}{"pmix.log.emsub"}{char*}{
Subject line for email.
}

%
\declareAttribute{PMIX_LOG_EMAIL_MSG}{"pmix.log.emmsg"}{char*}{
Message to be included in email.
}


%%%%%%%%%%%
\subsection{Debugger attributes}
\label{api:struct:attributes:debugger}

Attributes used to assist debuggers.

%
\declareAttribute{PMIX_DEBUG_STOP_ON_EXEC}{"pmix.dbg.exec"}{bool}{
Job is being spawned under debugger.
The processes are instructed to pause on start.
}

%
\declareAttribute{PMIX_DEBUG_STOP_IN_INIT}{"pmix.dbg.init"}{bool}{
Instruct job to stop processes during \refapi{PMIx_Init}.
}

%
\declareAttribute{PMIX_DEBUG_WAIT_FOR_NOTIFY}{"pmix.dbg.notify"}{bool}{
Block at desired point until receiving debugger release notification.
}

%
\declareAttribute{PMIX_DEBUG_JOB}{"pmix.dbg.job"}{char*}{
Namespace of the job to be debugged.
}

%
\declareAttribute{PMIX_DEBUG_WAITING_FOR_NOTIFY}{"pmix.dbg.waiting"}{bool}{
Job to be debugged is waiting for a release.
}


%%%%%%%%%%%
\subsection{Resource manager attributes}
\label{api:struct:attributes:rm}

Attributes used to describe the \ac{RM}.

%
\declareAttribute{PMIX_RM_NAME}{"pmix.rm.name"}{char*}{
String name of the \ac{RM}.
}

%
\declareAttribute{PMIX_RM_VERSION}{"pmix.rm.version"}{char*}{
\ac{RM} version string.
}


%%%%%%%%%%%
\subsection{Environment variable attributes}
\label{api:struct:attributes:envar}

Attributes used to adjust environment variables.

%
\declareAttribute{PMIX_SET_ENVAR}{"pmix.set.envar"}{char*}{
String ``\code{key=value}'' value shall be put into the environment.
}

%
\declareAttribute{PMIX_UNSET_ENVAR}{"pmix.unset.envar"}{char*}{
Unset the environment variable specified in the string.
}


%%%%%%%%%%%
\subsection{Job Allocation attributes}
\label{api:struct:attributes:joballoc}

Attributes used to describe the job allocation.

%
\declareAttribute{PMIX_ALLOC_ID}{"pmix.alloc.id"}{char*}{
Provide a string identifier for this allocation request which can later be used to query status of the request.
}

%
\declareAttribute{PMIX_ALLOC_NUM_NODES}{"pmix.alloc.nnodes"}{uint64_t}{
The number of nodes.
}

%
\declareAttribute{PMIX_ALLOC_NODE_LIST}{"pmix.alloc.nlist"}{char*}{
Regular expression of the specific nodes.
}

%
\declareAttribute{PMIX_ALLOC_NUM_CPUS}{"pmix.alloc.ncpus"}{uint64_t}{
Number of cpus.
}

%
\declareAttribute{PMIX_ALLOC_NUM_CPU_LIST}{"pmix.alloc.ncpulist"}{char*}{
Regular expression of the number of cpus for each node.
}

%
\declareAttribute{PMIX_ALLOC_CPU_LIST}{"pmix.alloc.cpulist"}{char*}{
Regular expression of the specific cpus indicating the cpus involved.
}

%
\declareAttribute{PMIX_ALLOC_MEM_SIZE}{"pmix.alloc.msize"}{float}{
Number of Megabytes.
}

%
\declareAttribute{PMIX_ALLOC_NETWORK}{"pmix.alloc.net"}{array}{
Array of \refstruct{pmix_info_t} describing network resources.
If not given as part of an \refstruct{pmix_info_t} struct that identifies the impacted nodes, then the description will be applied across all nodes in the requestor's allocation.
}

%
\declareAttribute{PMIX_ALLOC_NETWORK_ID}{"pmix.alloc.netid"}{char*}{
Name of the network.
}

%
\declareAttribute{PMIX_ALLOC_BANDWIDTH}{"pmix.alloc.bw"}{float}{
Mbits/sec.
}

%
\declareAttribute{PMIX_ALLOC_NETWORK_QOS}{"pmix.alloc.netqos"}{char*}{
Quality of service level.
}

%
\declareAttribute{PMIX_ALLOC_TIME}{"pmix.alloc.time"}{uint32_t}{
Time in seconds.
}


%%%%%%%%%%%
\subsection{Job control attributes}
\label{api:struct:attributes:jobcontrol}

Attributes used to control the job allocation.

%
\declareAttribute{PMIX_JOB_CTRL_ID}{"pmix.jctrl.id"}{char*}{
Provide a string identifier for this request.
}

%
\declareAttribute{PMIX_JOB_CTRL_PAUSE}{"pmix.jctrl.pause"}{bool}{
Pause the specified processes.
}

%
\declareAttribute{PMIX_JOB_CTRL_RESUME}{"pmix.jctrl.resume"}{bool}{
Resume (``un-pause'') the specified processes.
}

%
\declareAttribute{PMIX_JOB_CTRL_CANCEL}{"pmix.jctrl.cancel"}{char*}{
Cancel the specified request (\code{NULL} implies cancel all requests from this requestor).
}

%
\declareAttribute{PMIX_JOB_CTRL_KILL}{"pmix.jctrl.kill"}{bool}{
Forcibly terminate the specified processes and cleanup.
}

%
\declareAttribute{PMIX_JOB_CTRL_RESTART}{"pmix.jctrl.restart"}{char*}{
Restart the specified processes using the given checkpoint ID.
}

%
\declareAttribute{PMIX_JOB_CTRL_CHECKPOINT}{"pmix.jctrl.ckpt"}{char*}{
Checkpoint the specified processes and assign the given ID to it.
}

%
\declareAttribute{PMIX_JOB_CTRL_CHECKPOINT_EVENT}{"pmix.jctrl.ckptev"}{bool}{
Use event notification to trigger a process checkpoint.
}

%
\declareAttribute{PMIX_JOB_CTRL_CHECKPOINT_SIGNAL}{"pmix.jctrl.ckptsig"}{int}{
Use the given signal to trigger a process checkpoint.
}

%
\declareAttribute{PMIX_JOB_CTRL_CHECKPOINT_TIMEOUT}{"pmix.jctrl.ckptsig"}{int}{
Time in seconds to wait for a checkpoint to complete.
}

%
\declareAttribute{PMIX_JOB_CTRL_CHECKPOINT_METHOD}{"pmix.jctrl.ckmethod"}{pmix_data_array_t}{
Array of \refstruct{pmix_info_t} declaring each method and value supported by this application.
}

%
\declareAttribute{PMIX_JOB_CTRL_SIGNAL}{"pmix.jctrl.sig"}{int}{
Send given signal to specified processes.
}

%
\declareAttribute{PMIX_JOB_CTRL_PROVISION}{"pmix.jctrl.pvn"}{char*}{
Regular expression identifying nodes that are to be provisioned.
}

%
\declareAttribute{PMIX_JOB_CTRL_PROVISION_IMAGE}{"pmix.jctrl.pvnimg"}{char*}{
Name of the image that is to be provisioned.
}

%
\declareAttribute{PMIX_JOB_CTRL_PREEMPTIBLE}{"pmix.jctrl.preempt"}{bool}{
Indicate that the job can be pre-empted.
}

%
\declareAttribute{PMIX_JOB_CTRL_TERMINATE}{"pmix.jctrl.term"}{bool}{
Politely terminate the specified processes.
}


%%%%%%%%%%%
\subsection{Monitoring attributes}
\label{api:struct:attributes:monitor}

Attributes used to monitor the job.

%
\declareAttribute{PMIX_MONITOR_ID}{"pmix.monitor.id"}{char*}{
Provide a string identifier for this request.
}

%
\declareAttribute{PMIX_MONITOR_CANCEL}{"pmix.monitor.cancel"}{char*}{
Identifier to be canceled (\code{NULL} means cancel all monitoring for this process).
}

%
\declareAttribute{PMIX_MONITOR_APP_CONTROL}{"pmix.monitor.appctrl"}{bool}{
The application desires to control the response to a monitoring event.
}

%
\declareAttribute{PMIX_MONITOR_HEARTBEAT}{"pmix.monitor.mbeat"}{void}{
Register to have the PMIx server monitor the requestor for heartbeats.
}

%
\declareAttribute{PMIX_SEND_HEARTBEAT}{"pmix.monitor.beat"}{void}{
Send heartbeat to local PMIx server.
}

%
\declareAttribute{PMIX_MONITOR_HEARTBEAT_TIME}{"pmix.monitor.btime"}{uint32_t}{
Time in seconds before declaring heartbeat missed.
}

%
\declareAttribute{PMIX_MONITOR_HEARTBEAT_DROPS}{"pmix.monitor.bdrop"}{uint32_t}{
Number of heartbeats that can be missed before generating the event.
}

%
\declareAttribute{PMIX_MONITOR_FILE}{"pmix.monitor.fmon"}{char*}{
Register to monitor file for signs of life.
}

%
\declareAttribute{PMIX_MONITOR_FILE_SIZE}{"pmix.monitor.fsize"}{bool}{
Monitor size of given file is growing to determine if the application is running.
}

%
\declareAttribute{PMIX_MONITOR_FILE_ACCESS}{"pmix.monitor.faccess"}{char*}{
Monitor time since last access of given file to determine if the application is running.
}

%
\declareAttribute{PMIX_MONITOR_FILE_MODIFY}{"pmix.monitor.fmod"}{char*}{
Monitor time since last modified of given file to determine if the application is running.
}

%
\declareAttribute{PMIX_MONITOR_FILE_CHECK_TIME}{"pmix.monitor.ftime"}{uint32_t}{
Time in seconds between checking the file.
}

%
\declareAttribute{PMIX_MONITOR_FILE_DROPS}{"pmix.monitor.fdrop"}{uint32_t}{
Number of file checks that can be missed before generating the event.
}



%%%%%%%%%%%
\section{Callback Functions}

PMIx provides blocking and nonblocking versions of most APIs.
In the nonblocking versions, a callback is activated upon completion of the the operation.
This section describes many of those callbacks.


%%%%%%%%%%%
\subsection{Release Callback Function}
\declareapi{pmix_release_cbfunc_t}

%%%%
\summary

The \refapi{pmix_release_cbfunc_t} is used by the \refapi{pmix_modex_cbfunc_t} and \refapi{pmix_info_cbfunc_t} operations to indicate that the callback data may be reclaimed/freed by the caller.

%%%%
\format

\cspecificstart
\begin{codepar}
typedef void (*pmix_release_cbfunc_t)
    (void *cbdata)
\end{codepar}
\cspecificend

\begin{arglist}
\arginout{cbdata}{Callback data passed to original API call (memory reference)}
\end{arglist}

%%%%
\descr

Since the data is ``owned'' by the host server, provide a callback function to notify the host server that we are done with the data so it can be released.


%%%%%%%%%%%
\subsection{Modex Callback Function}
\declareapi{pmix_modex_cbfunc_t}

%%%%
\summary

The \refapi{pmix_modex_cbfunc_t} is used by the \refapi{pmix_server_fencenb_fn_t} and \refapi{pmix_server_dmodex_req_fn_t} PMIx server operations to return modex \ac{BCX} data.

\cspecificstart
\begin{codepar}
typedef void (*pmix_modex_cbfunc_t)
    (pmix_status_t status,
     const char *data, size_t ndata,
     void *cbdata,
     pmix_release_cbfunc_t release_fn,
     void *release_cbdata)
\end{codepar}
\cspecificend

\begin{arglist}
\argin{status}{Status associated with the operation (handle)}
\argin{data}{Data to be passed (pointer)}
\argin{ndata}{size of the data (\code{size_t})}
\argin{cbdata}{Callback data passed to original API call (pointer)}
\argin{release_fn}{Callback for releasing \argref{release_cbdata} (function pointer)}
\argin{release_cbdata}{Pointer to be passed to \argref{release_fn} (memory reference)}
\end{arglist}

%%%%
\descr

A callback function that is solely used by PMIx servers, and not clients, to return modex \ac{BCX} data in response to ``fence'' and ``get'' operations.
The returned blob contains the data collected from each server participating in the operation.



%%%%%%%%%%%
\subsection{Spawn Callback Function}
\declareapi{pmix_spawn_cbfunc_t}

%%%%
\summary

The \refapi{pmix_spawn_cbfunc_t} is used on the PMIx client side by \refapi{PMIx_Spawn_nb} and on the PMIx server side by \refapi{pmix_server_spawn_fn_t}.

\cspecificstart
\begin{codepar}
typedef void (*pmix_spawn_cbfunc_t)
    (pmix_status_t status,
     char nspace[], void *cbdata);
\end{codepar}
\cspecificend

\begin{arglist}
\argin{status}{Status associated with the operation (handle)}
\argin{nspace}{Namespace string (string)}
\argin{cbdata}{Callback data passed to original API call (memory reference)}
\end{arglist}


%%%%
\descr

The callback will be executed upon launch of the specified applications in \refapi{PMIx_Spawn_nb}, or upon failure to launch any of them.

The \refarg{status} of the callback will indicate whether or not the spawn succeeded.
The \refarg{nspace} of the spawned processes will be returned, along with any provided callback data.
Note that the returned \refarg{nspace} value will be released by the \ac{PRI} upon return from the callback function, so the receiver must copy it if it needs to be retained.


%%%%%%%%%%%
\subsection{Op Callback Function}
\declareapi{pmix_op_cbfunc_t}

%%%%
\summary

The \refapi{pmix_op_cbfunc_t} is used by operations that simply return a status.

\cspecificstart
\begin{codepar}
typedef void (*pmix_op_cbfunc_t)
    (pmix_status_t status, void *cbdata);
\end{codepar}
\cspecificend

\begin{arglist}
\argin{status}{Status associated with the operation (handle)}
\argin{cbdata}{Callback data passed to original API call (memory reference)}
\end{arglist}

%%%%
\descr

Used by a wide range of \ac{PMIx} API's including \refapi{PMIx_Fence_nb}, \refapi{pmix_server_client_connected_fn_t}, \refapi{PMIx_server_register_nspace}.
This callback function is used to return a status to an often nonblocking operation.


%%%%%%%%%%%
\subsection{Lookup Callback Function}
\declareapi{pmix_lookup_cbfunc_t}

%%%%
\summary

The \refapi{pmix_lookup_cbfunc_t} is used by \refapi{PMIx_Lookup_nb} to return data.

\cspecificstart
\begin{codepar}
typedef void (*pmix_lookup_cbfunc_t)
    (pmix_status_t status,
     pmix_pdata_t data[], size_t ndata,
     void *cbdata);
\end{codepar}
\cspecificend

\begin{arglist}
\argin{status}{Status associated with the operation (handle)}
\argin{data}{data returned (\refstruct{pmix_pdata_t})}
\argin{ndata}{size of the data returned (\code{size_t})}
\argin{cbdata}{Callback data passed to original API call (memory reference)}
\end{arglist}


%%%%
\descr

A callback function for calls to \refapi{PMIx_Lookup_nb}
The function will be called upon completion of the command with the \refarg{status} indicating the success or failure of the request.
Any retrieved data will be returned in an array of \refstruct{pmix_pdata_t} structs.
The namespace and rank of the process that provided each data element is also returned.

Note that these structures will be released upon return from the callback function, so the receiver must copy/protect the data prior to returning if it needs to be retained.


%%%%%%%%%%%
\subsection{Value Callback Function}
\declareapi{pmix_value_cbfunc_t}

%%%%
\summary

The \refapi{pmix_value_cbfunc_t} is used by \refapi{PMIx_Get_nb} to return data.

\cspecificstart
\begin{codepar}
typedef void (*pmix_value_cbfunc_t)
    (pmix_status_t status,
     pmix_value_t *kv, void *cbdata);
\end{codepar}
\cspecificend

\begin{arglist}
\argin{status}{Status associated with the operation (handle)}
\argin{kv}{Key/value pair representing the data (\refstruct{pmix_value_t})}
\argin{cbdata}{Callback data passed to original API call (memory reference)}
\end{arglist}


%%%%
\descr

A callback function for calls to \refapi{PMIx_Get_nb}.
The \refarg{status} indicates if the requested data was found or not.
A pointer to the \refstruct{pmix_value_t} structure containing the found data is returned.
The pointer will be \code{NULL} if the requested data was not found.


%%%%%%%%%%%
\subsection{Info Function}
\declareapi{pmix_info_cbfunc_t}

%%%%
\summary

The \refapi{pmix_info_cbfunc_t} is a general information callback used by various APIs.

\cspecificstart
\begin{codepar}
typedef void (*pmix_info_cbfunc_t)
    (pmix_status_t status,
     pmix_info_t *info, size_t ninfo,
     void *cbdata,
     pmix_release_cbfunc_t release_fn,
     void *release_cbdata);
\end{codepar}
\cspecificend

\begin{arglist}
\argin{status}{Status associated with the operation (\refattr{pmix_status_t})}
\argin{data}{data returned (\refstruct{pmix_pdata_t})}
\argin{ndata}{size of the data returned (\code{size_t})}
\argin{cbdata}{Callback data passed to original API call (memory reference)}
\end{arglist}


%%%%
\descr

The \refarg{status} indicates if requested data was found or not.
An array of \refstruct{pmix_info_t} will contain the key/value pairs.


%%%%%%%%%%%
\subsection{Event Handler Registration Function}
\declareapi{pmix_evhdlr_reg_cbfunc_t}

The \refapi{pmix_evhdlr_reg_cbfunc_t} callback function.

\cspecificstart
\begin{codepar}
typedef void (*pmix_evhdlr_reg_cbfunc_t)
    (pmix_status_t status,
     size_t evhdlr_ref,
     void *cbdata)
\end{codepar}
\cspecificend

\begin{arglist}
\argin{status}{Status indicates if the request was successful or not (\refattr{pmix_status_t})}
\argin{evhdlr_ref}{Reference assigned to the event handler by \ac{PMIx} --- this reference
 * must be used to deregister the err handler (\code{size_t})}
\argin{cbdata}{Callback data passed to original API call (memory reference)}
\end{arglist}


%%%%
\descr

Define a callback function for calls to \refapi{PMIx_Register_event_handler}


%%%%%%%%%%%
\subsection{Notification Handler Completion Callback Function}
\declareapi{pmix_event_notification_cbfunc_fn_t}

%%%%
\summary

The \refapi{pmix_event_notification_cbfunc_fn_t} is called by event handlers to indicate completion of their operations.

\cspecificstart
\begin{codepar}
typedef void (*pmix_event_notification_cbfunc_fn_t)
    (pmix_status_t status,
     pmix_info_t *results, size_t nresults,
     pmix_op_cbfunc_t cbfunc, void *thiscbdata,
     void *notification_cbdata);
\end{codepar}
\cspecificend

\begin{arglist}
\argin{status}{Status returned by the event handler's operation (\refattr{pmix_status_t})}
\argin{results}{Results from this event handler's operation on the event (\refstruct{pmix_info_t})}
\argin{nresults}{Number of elements in the results array (\code{size_t})}
\argin{cbfunc}{\refapi{pmix_op_cbfunc_t} function to be executed when \ac{PMIx} completes processing the callback (function reference)}
\argin{thiscbdata}{Callback data that was passed in to the handler (memory reference)}
\argin{cbdata}{Callback data to be returned when \ac{PMIx} executes cbfunc (memory reference)}
\end{arglist}

%%%%
\descr

Define a callback by which an event handler can notify the \ac{PMIx} library that it has completed its response to the notification. The handler is \textit{required} to execute this callback so the library can determine if additional handlers need to be called. The handler shall return \refconst{PMIX_ERR_EVENT_COMPLETE} if no further action is required. The return status of each event handler and any returned \refstruct{pmix_info_t} structures will be added to the \textit{results} array of \refstruct{pmix_info_t} passed to any subsequent event handlers to help guide their operation.

If non-NULL, the provided callback function will be called to allow the event handler to release the provided info array and execute any other required cleanup operations.


%%%%%%%%%%%
\subsection{Notification Function}
\declareapi{pmix_notification_fn_t}

%%%%
\summary

The \refapi{pmix_notification_fn_t} is called by \ac{PMIx} to deliver notification of an event.

\cspecificstart
\begin{codepar}
typedef void (*pmix_notification_fn_t)
    (size_t evhdlr_registration_id,
     pmix_status_t status,
     const pmix_proc_t *source,
     pmix_info_t info[], size_t ninfo,
     pmix_info_t *results, size_t nresults,
     pmix_event_notification_cbfunc_fn_t cbfunc,
     void *cbdata);
\end{codepar}
\cspecificend

\begin{arglist}
\argin{evhdlr_registration_id}{Registration number of the handler being called (\code{size_t})}
\argin{status}{Status associated with the operation (\refattr{pmix_status_t})}
\argin{source}{Identifier of the process that generated the event (\refstruct{pmix_proc_t})}. If the source is the \ac{SMS}, then the nspace will be empty and the rank will be PMIX_RANK_UNDEF
\argin{info}{Information describing the event (\refstruct{pmix_info_t})}. This argument will be NULL if no additional information was provided by the event generator.
\argin{ninfo}{Number of elements in the info array (\code{size_t})}
\argin{results}{Aggregated results from prior event handlers servicing this event (\refstruct{pmix_info_t})}. This argument will be NULL if this is the first handler servicing the event, or if no prior handlers provided results.
\argin{nresults}{Number of elements in the results array (\code{size_t})}
\argin{cbfunc}{\refapi{pmix_event_notification_cbfunc_fn_t} callback function to be executed upon completion of the handler's operation and prior to handler return (function reference)}.
\argin{cbdata}{Callback data to be passed to cbfunc (memory reference)}
\end{arglist}

%%%%
\descr

Note that different \acp{RM} may provide differing levels of support for event notification to application processes. Thus, the \textit{info} array may be NULL or may contain detailed information of the event. It is the responsibility of the application to parse any provided info array for defined key-values if it so desires.

\adviceuserstart
Possible uses of the pmix_info_t object include:

\begin{itemize}
\item for the RM to alert the process as to planned actions, such as to abort the session, in response to the reported event

\item provide a timeout for alternative action to occur, such as for the application to request an alternate response to the event
\end{itemize}

For example, the RM might alert the application to the failure of a node that resulted in termination of several processes, and indicate that the overall session will be aborted unless the application requests an alternative behavior in the next 5 seconds. The application then has time to respond with a checkpoint request, or a request to recover from the failure by obtaining replacement nodes and restarting from some earlier checkpoint.

Support for these options is left to the discretion of the host RM. Info keys are included in the common definitions above, but also may be augmented on a per-RM basis.

On the server side, the notification function is used to inform the host server of a detected event in the PMIx subsystem and/or client
\adviceuserend


%%%%%%%%%%%
\subsection{Server Setup Application Callback Function}
\declareapi{pmix_setup_application_cbfunc_t}

The \refapi{PMIx_server_setup_application} callback function.

%%%%
\summary

Provide a function by which the resource manager can receive application-specific environmental variables and other setup data prior to launch of an application.

%%%%
\format

\versionMarker{2.0}
\cspecificstart
\begin{codepar}
typedef void (*pmix_setup_application_cbfunc_t)(
                        pmix_status_t status,
                        pmix_info_t info[], size_t ninfo,
                        void *provided_cbdata,
                        pmix_op_cbfunc_t cbfunc, void *cbdata)
\end{codepar}
\cspecificend

\begin{arglist}
\argin{status}{returned status of the request (\refstruct{pmix_status_t})}
\argin{info}{Array of info structures (array of handles)}
\argin{ninfo}{Number of elements in the \refarg{info} array (integer)}
\argin{provided_cbdata}{Data originally passed to call to \refapi{PMIx_server_setup_application} (memory reference)}
\argin{cbfunc}{\refapi{pmix_op_cbfunc_t} function to be called when processing completed (function reference)}
\argin{cbdata}{Data to be passed to the \refarg{cbfunc} callback function (memory reference)}
\end{arglist}

\descr

Define a function to be called by the \ac{PMIx} server library for return of application-specific setup data in response to a request from the host \ac{RM}. The returned \refarg{info} array is owned by the \ac{PMIx} server library and will be free'd when the provided \refarg{cbfunc} is called.


%%%%%%%%%%%
\subsection{Server Direct Modex Response Callback Function}
\declareapi{pmix_dmodex_response_fn_t}

The \refapi{PMIx_server_dmodex_request} callback function.

%%%%
\summary

Provide a function by which the local \ac{PMIx} server library can return connection and other data posted by local application processes to the host resource manager.

%%%%
\format

\versionMarker{1.0}
\cspecificstart
\begin{codepar}
typedef void (*pmix_dmodex_response_fn_t)(pmix_status_t status,
                        char *data, size_t sz,
                        void *cbdata);
\end{codepar}
\cspecificend


\begin{arglist}
\argin{status}{Returned status of the request (\refstruct{pmix_status_t})}
\argin{data}{Pointer to a data "blob" containing the requested information (handle)}
\argin{sz}{Number of bytes in the \refarg{data} blob (integer)}
\argin{cbdata}{Data passed into the initial call to \refapi{PMIx_server_dmodex_request} (memory reference)}
\end{arglist}


\descr
Define a function to be called by the PMIx server library for return of information posted by a local application process (via \refapi{PMIx_Put} with subsequent \refapi{PMIx_Commit}) in response to a request from the host RM. The returned \refarg{data} blob is owned by the PMIx server library and will be free’d upon return from the function.

%%%%%%%%%%%
\subsection{Constant String Functions}

Provide a string representation for several types of values.
Note that the provided string is statically defined and must NOT be \code{free}'d.

%%%%
\summary
\declareapi{PMIx_Error_string}

String representation of a \refstruct{pmix_status_t}.

\versionMarker{1.0}
\cspecificstart
\begin{codepar}
const char*
PMIx_Error_string(pmix_status_t status);
\end{codepar}
\cspecificend

%%%%
\summary
\declareapi{PMIx_Proc_state_string}

String representation of a \refstruct{pmix_proc_state_t}.

\versionMarker{2.0}
\cspecificstart
\begin{codepar}
const char*
PMIx_Proc_state_string(pmix_proc_state_t state);
\end{codepar}
\cspecificend

%%%%
\summary
\declareapi{PMIx_Scope_string}

String representation of a \refstruct{pmix_scope_t}.

\versionMarker{2.0}
\cspecificstart
\begin{codepar}
const char*
PMIx_Scope_string(pmix_scope_t scope);
\end{codepar}
\cspecificend

%%%%
\summary
\declareapi{PMIx_Persistence_string}

String representation of a \refstruct{pmix_persistence_t}.

\versionMarker{2.0}
\cspecificstart
\begin{codepar}
const char*
PMIx_Persistence_string(pmix_persistence_t persist);
\end{codepar}
\cspecificend

%%%%
\summary
\declareapi{PMIx_Data_range_string}

String representation of a \refstruct{pmix_data_range_t}.

\versionMarker{2.0}
\cspecificstart
\begin{codepar}
const char*
PMIx_Data_range_string(pmix_data_range_t range);
\end{codepar}
\cspecificend

%%%%
\summary
\declareapi{PMIx_Info_directives_string}

String representation of a \refstruct{pmix_info_directives_t}.

\versionMarker{2.0}
\cspecificstart
\begin{codepar}
const char*
PMIx_Info_directives_string(pmix_info_directives_t directives);
\end{codepar}
\cspecificend

%%%%
\summary
\declareapi{PMIx_Data_type_string}

String representation of a \refstruct{pmix_data_type_t}.

\versionMarker{2.0}
\cspecificstart
\begin{codepar}
const char*
PMIx_Data_type_string(pmix_data_type_t type);
\end{codepar}
\cspecificend

%%%%
\summary
\declareapi{PMIx_Alloc_directive_string}

String representation of a \refstruct{pmix_alloc_directive_t}.

\versionMarker{2.0}
\cspecificstart
\begin{codepar}
const char*
PMIx_Alloc_directive_string(pmix_alloc_directive_t directive);
\end{codepar}
\cspecificend


%%%%%%%%%%%%%%%%%%%%%%%%%%%%%%%%%%%%%%%%%%%%%%%%%
