%%%%%%%%%%%%%%%%%%%%%%%%%%%%%%%%%%%%%%%%%%%%%%%%%
% Chapter: API Server
%%%%%%%%%%%%%%%%%%%%%%%%%%%%%%%%%%%%%%%%%%%%%%%%%
\chapter{Server-Specific Interfaces}
\label{chap:api_server}

The \ac{RM} daemon that hosts the \ac{PMIx} server library interacts with that library in two distinct manners. First, \ac{PMIx} provides a set of \acp{API} by which the host can request specific services from its library. This includes generating regular expressions, registering information to be passed to client processes, and requesting information on behalf of a remote process. Note that the host always has access to all \ac{PMIx} client \acp{API} - the functions listed below are in addition to those available to a \ac{PMIx} client.

Second, the host can provide a set of callback functions by which the \ac{PMIx} server library can pass requests upward for servicing by the host. These include notifications of client connection and finalize, as well as requests by clients for information and/or services that the \ac{PMIx} server library does not itself provide.


%%%%%%%%%%%
\subsection{\code{PMIx_generate_regex}}
\declareapi{PMIx_generate_regex}

%%%%
\summary

Generate a regular expression representation of the input string.

%%%%
\format

\versionMarker{1.0}
\cspecificstart
\begin{codepar}
pmix_status_t
PMIx_generate_regex(const char *input, char **regex)
\end{codepar}
\cspecificend

\begin{arglist}
\argin{input}{String to process (string)}
\argout{regex}{Regular expression representation of \refarg{input} (string)}
\end{arglist}

Returns \refconst{PMIX_SUCCESS} or a negative value corresponding to a PMIx error constant.

%%%%
\descr

Given a comma-separated list of \refarg{input} values, generate a regular expression that can be passed down to the \ac{PMIx} client for parsing.
The caller is responsible for free'ing the resulting string.

If values have leading zero's, then that is preserved, as are prefix and suffix strings. For example, an input string of ``\code{odin009.org,odin010.org,odin011.org,odin012.org,odin[102-107].org}'' will return a regular expression of ``\code{pmix:odin[009-012,102-107].org}''

\adviceuserstart
The returned regular expression will have a ``\code{pmix:}'' at the beginning of the string. This informs the \ac{PMIx} parser that the string was produced using the \ac{PRI}'s regular expression generator, and thus that same plugin should be used for parsing the string
\adviceuserend


%%%%%%%%%%%
\subsection{\code{PMIx_generate_ppn}}
\declareapi{PMIx_generate_ppn}

%%%%
\summary

Generate a regular expression representation of the input string.

%%%%
\format

\versionMarker{1.0}
\cspecificstart
\begin{codepar}
pmix_status_t PMIx_generate_ppn(const char *input, char **ppn)
\end{codepar}
\cspecificend

\begin{arglist}
\argin{input}{String to process (string)}
\argout{regex}{Regular expression representation of \refarg{input} (string)}
\end{arglist}

Returns \refconst{PMIX_SUCCESS} or a negative value corresponding to a PMIx error constant.

%%%%
\descr

The input is expected to consist of a semicolon-separated list of ranges representing the ranks of processes on each node of the job. Thus, an input of "1-4;2-5;8,10,11,12;6,7,9" would generate a regex of "pmix:2x(3);8,10-12;6-7,9"

\adviceuserstart
The returned regular expression will have a ``pmix:'' at the beginning of the string. This informs the \ac{PMIx} parser that the string was produced using the \ac{PRI}'s regular expression generator, and thus that same plugin should be used for parsing the string
\adviceuserend

%%%%%%%%%%%
\subsection{\code{PMIx_server_register_nspace}}
\declareapi{PMIx_server_register_nspace}

%%%%
\summary

Setup the data about a particular namespace.

%%%%
\format

\versionMarker{1.0}
\cspecificstart
\begin{codepar}
pmix_status_t
PMIx_server_register_nspace(const char nspace[],
                        int nlocalprocs,
                        pmix_info_t info[], size_t ninfo,
                        pmix_op_cbfunc_t cbfunc, void *cbdata)
\end{codepar}
\cspecificend

\begin{arglist}
\argin{nspace}{namespace (string)}
\argin{nlocalprocs}{number of local processes (integer)}
\argin{info}{Array of info structures (array of handles)}
\argin{ninfo}{Number of elements in the \refarg{info} array (integer)}
\argin{cbfunc}{Callback function \refapi{pmix_op_cbfunc_t} (function reference)}
\argin{cbdata}{Data to be passed to the callback function (memory reference)}
\end{arglist}

Returns \refconst{PMIX_SUCCESS} or a negative value corresponding to a PMIx error constant.

\priattr
Internally supported attributes include:

\pasteAttributeItem{PMIX_REGISTER_NODATA}

%%%%
\descr

The PMIx_server_register_nspace procedure provides an opportunity for the host \ac{RM} daemon to pass job-related info down to a child process.
This might include the number of processes in the job, relative local ranks of the processes within the job, and other information of use to the process.
The host daemon is free to determine which, if any, of the supported elements it will provide (See \refsection{chap:struct}{Data Structures and Types} for values).

The \ac{PMIx} server must register \emph{all} namespaces that will participate in collective operations with local processes.
This means that the server must register a namespace even if it will not host any local processes from within that nspace \emph{if} any local process of another nspace might at some point perform an operation involving one or more processes from the new namespace.
This is necessary so that the collective operation can identify the participants and know when it is locally complete.

The caller must also provide the number of local processes that will be launched within this namespace.
This is required for the \ac{PMIx} server library to correctly handle collectives as a collective operation call can occur before all the local processes have been started.

\adviceuserstart
The number of local processes for any given nspace is generally fixed at the time of application launch. Calls to \refapi{PMIx_Spawn} result in processes launched in their own nspace, not that of their parent. However, it is possible for processes to \textit{migrate} to another node via a call to \refapi{PMIx_Job_control_nb}, thus resulting in a change to the number of local processes on both the initial node and the node to which the process moved. It is therefore \textit{critical} that applications not migrate processes without first ensuring that \ac{PMIx}-based collective operations are not in progress, and that no such operations be initiated until process migration has completed.
\adviceuserend


%%%%%%%%%%%
\subsection{\code{PMIx_server_deregister_nspace}}
\declareapi{PMIx_server_deregister_nspace}

%%%%
\summary

Deregister a namespace.

%%%%
\format

\versionMarker{1.0}
\cspecificstart
\begin{codepar}
void PMIx_server_deregister_nspace(const char nspace[],
                        pmix_op_cbfunc_t cbfunc, void *cbdata)
\end{codepar}
\cspecificend

\begin{arglist}
\argin{nspace}{Namespace (string)}
\argin{cbfunc}{Callback function \refapi{pmix_op_cbfunc_t} (function reference)}
\argin{cbdata}{Data to be passed to the callback function (memory reference)}
\end{arglist}

%%%%
\descr

Deregister the specified \refarg{nspace} and purge all objects relating to it, including any client information from that namespace.
This is intended to support persistent \ac{PMIx} servers by providing an opportunity for the host \ac{RM} to tell the \ac{PMIx} server library to release all memory for a completed job.



%%%%%%%%%%%
\subsection{\code{PMIx_server_register_client}}
\declareapi{PMIx_server_register_client}

%%%%
\summary

Register a client process with the PMIx server library.

%%%%
\format

\versionMarker{1.0}
\cspecificstart
\begin{codepar}
pmix_status_t
PMIx_server_register_client(const pmix_proc_t *proc,
                        uid_t uid, gid_t gid,
                        void *server_object,
                        pmix_op_cbfunc_t cbfunc, void *cbdata)
\end{codepar}
\cspecificend

\begin{arglist}
\argin{proc}{\refstruct{pmix_proc_t} structure (handle)}
\argin{uid}{user id (integer)}
\argin{gid}{group id (integer)}
\argin{server_object}{(memory reference)}
\argin{cbfunc}{Callback function \refapi{pmix_op_cbfunc_t} (function reference)}
\argin{cbdata}{Data to be passed to the callback function (memory reference)}
\end{arglist}

Returns \refconst{PMIX_SUCCESS} or a negative value corresponding to a PMIx error constant.

%%%%
\descr

Register a client process with the PMIx server library.
The expected user ID and group ID of the child process helps the server library to properly authenticate clients as they connect by requiring the two values to match.

The host server can also, if it desires, provide an object it wishes to be returned when a server function is called that relates to a specific process.
For example, the host server may have an object that tracks the specific client.
Passing the object to the library allows the library to provide that object to the host server during subsequent calls related to that client, such as a ``pmix_server_client_connected_fn'' function.  This allows the host server to access the object without performing a lookup based the client's namespace and rank.


%%%%%%%%%%%
\subsection{\code{PMIx_server_deregister_client}}
\declareapi{PMIx_server_deregister_client}

%%%%
\summary

Deregister a client and purge all data relating to it.

%%%%
\format

\versionMarker{1.0}
\cspecificstart
\begin{codepar}
void
PMIx_server_deregister_client(const pmix_proc_t *proc,
                        pmix_op_cbfunc_t cbfunc, void *cbdata)
\end{codepar}
\cspecificend

\begin{arglist}
\argin{proc}{\refstruct{pmix_proc_t} structure (handle)}
\argin{cbfunc}{Callback function \refapi{pmix_op_cbfunc_t} (function reference)}
\argin{cbdata}{Data to be passed to the callback function (memory reference)}
\end{arglist}


%%%%
\descr

The \refapi{PMIx_server_deregister_nspace} \ac{API} will delete all client information for that namespace. The \ac{PMIx} server library will automatically perform that operation upon disconnect of all local clients.
This \ac{API} is therefore intended primarily for use in exception cases, but can be called in non-exception cases if desired.


%%%%%%%%%%%
\subsection{\code{PMIx_server_setup_fork}}
\declareapi{PMIx_server_setup_fork}

%%%%
\summary

Setup the environment of a child process to be forked by the host.

%%%%
\format

\versionMarker{1.0}
\cspecificstart
\begin{codepar}
pmix_status_t
PMIx_server_setup_fork(const pmix_proc_t *proc,
                        char ***env)
\end{codepar}
\cspecificend

\begin{arglist}
\argin{proc}{\refstruct{pmix_proc_t} structure (handle)}
\argin{env}{Environment array (array of strings)}
\end{arglist}

Returns \refconst{PMIX_SUCCESS} or a negative value corresponding to a PMIx error constant.

%%%%
\descr

Setup the environment of a child process to be forked by the host so it can correctly interact with the PMIx server.
The PMIx client needs some setup information so it can properly connect back to the server.
This function will set appropriate environmental variables for this purpose, and will also provide any environmental variables that were specified in the launch command (e.g., via \refapi{PMIx_Spawn}).


%%%%%%%%%%%
\subsection{\code{PMIx_server_dmodex_request}}
\declareapi{PMIx_server_dmodex_request}
\declareapi{pmix_dmodex_response_fn_t}

%%%%
\summary

Define a function by which the host server can request modex data from the local PMIx server.

%%%%
\format

\versionMarker{1.0}
\cspecificstart
\begin{codepar}
typedef void (*pmix_dmodex_response_fn_t)(pmix_status_t status,
                        char *data, size_t sz,
                        void *cbdata);

pmix_status_t PMIx_server_dmodex_request(const pmix_proc_t *proc,
                        pmix_dmodex_response_fn_t cbfunc,
                        void *cbdata)
\end{codepar}
\cspecificend

\begin{arglist}
\argin{proc}{\refstruct{pmix_proc_t} structure (handle)}
\argin{cbfunc}{Callback function \refapi{pmix_dmodex_response_fn_t} (function reference)}
\argin{cbdata}{Data to be passed to the callback function (memory reference)}
\end{arglist}

Returns \refconst{PMIX_SUCCESS} or a negative value corresponding to a PMIx error constant.

%%%%
\descr

Define a function by which the host server can request modex data from the local \ac{PMIx} server.
This is used to support the direct modex operation (i.e., where data is cached locally on each \ac{PMIx} server for its own local clients, and is obtained on-demand for remote requests).
Upon receiving a request from a remote server, the host server will call this function to pass the request into the \ac{PMIx} server.
The \ac{PMIx} server will return a blob (once it becomes available) via the \refarg{cbfunc} - the host server shall send the blob back to the original requestor.

The \ac{PMIx} server will free the data blob upon return from the response function.


%%%%%%%%%%%
\subsection{\code{PMIx_server_setup_application}}
\declareapi{PMIx_server_setup_application}

%%%%
\summary

Provide a function by which the resource manager can request application-specific environmental variables prior to launch of an application.

%%%%
\format

\versionMarker{2.0}
\cspecificstart
\begin{codepar}
pmix_status_t
PMIx_server_setup_application(const char nspace[],
                        pmix_info_t info[], size_t ninfo,
                        pmix_setup_application_cbfunc_t cbfunc,
                        void *cbdata)
\end{codepar}
\cspecificend

\begin{arglist}
\argin{nspace}{namespace (string)}
\argin{info}{Array of info structures (array of handles)}
\argin{ninfo}{Number of elements in the \refarg{info} array (integer)}
\argin{cbfunc}{Callback function \refapi{pmix_setup_application_cbfunc_t} (function reference)}
\argin{cbdata}{Data to be passed to the \refarg{cbfunc} callback function (memory reference)}
\end{arglist}

Returns \refconst{PMIX_SUCCESS} or a negative value corresponding to a PMIx error constant.

%%%%
\descr

Provide a function by which the \ac{RM} can request application-specific setup data (e.g., environmental variables, fabric configuration and security credentials) from supporting \ac{PMIx} server library subsystems prior to initiating launch of an application. This is defined as a non-blocking operation in case contributing subsystems need to perform some potentially time consuming action (e.g., query a remote service) before responding. The returned data must be distributed by the \ac{RM} and subsequently delivered to the local \ac{PMIx} server on each node where application processes will execute prior to initiating execution of those processes.

In the callback function, the returned \refarg{info} array is owned by the \ac{PMIx} server library and will be free'd when the provided \refarg{cbfunc} is called.

\adviceimplstart
Support for harvesting of environmental variables and providing of local configuration information by the \ac{PRI} is solely dependent on the operations of the active \ac{PMIx} plugins. Support in the \ac{PMIx} v2.x release series is limited - more complete support will be available beginning with the \ac{PMIx} v3.x series
\adviceimplend

%%%%%%%%%%%
\subsection{\code{PMIx_server_setup_local_support}}
\declareapi{PMIx_server_setup_local_support}

%%%%
\summary

Provide a function by which the local \ac{PMIx} server can perform any application-specific operations prior to spawning local clients of a given application.

%%%%
\format

\versionMarker{2.0}
\cspecificstart
\begin{codepar}
pmix_status_t
PMIx_server_setup_local_support(const char nspace[],
                                pmix_info_t info[], size_t ninfo,
                                pmix_op_cbfunc_t cbfunc,
                                void *cbdata);
\end{codepar}
\cspecificend

\begin{arglist}
\argin{nspace}{Namespace (string)}
\argin{info}{Array of info structures (array of handles)}
\argin{ninfo}{Number of elements in the \refarg{info} array (integer)}
\argin{cbfunc}{Callback function \refapi{pmix_op_cbfunc_t} (function reference)}
\argin{cbdata}{Data to be passed to the callback function (memory reference)}
\end{arglist}

Returns \refconst{PMIX_SUCCESS} or a negative value corresponding to a PMIx error constant.


%%%%
\descr

Provide a function by which the local \ac{PMIx} server can perform any application-specific operations prior to spawning local clients of a given application. For example, a network library might need to setup the local driver for ``instant on'' addressing. The data provided in the \refarg{info} array is the data provided to there host \ac{RM} from the a call to \refapi{PMIx_server_setup_application}.


%%%%%%%%%%%
\section{Server Function Pointers}

\ac{PMIx} utilizes a "function-shipping" approach to support for implementing the server-side of the protocol. This method allows \acp{RM} to implement the server without being burdened with \ac{PMIx} internal details. Accordingly, each \ac{PMIx} \ac{API} is mirrored in a function call to be provided by the server. When a request is received from the client, the corresponding server function will be called with the information.

Any functions not supported by the \ac{RM} can be indicated by a \code{NULL} for the function pointer. Client calls to such functions will return a \refconst{PMIX_ERR_NOT_SUPPORTED} status.

The host \ac{RM} will provide the function pointers in a \refapi{pmix_server_module_t} structure passed to \refapi{PMIx_server_init}.
That module structure and associated function references are defined in this section.

\adviceimplstart
For performance purposes, the host server is required to return as quickly as possible from all functions. Execution of
the function is thus to be done asynchronously so as to allow the \ac{PMIx} server support library to handle multiple client requests
as quickly and scalably as possible.

\textit{All} data passed to the host server functions is ``owned'' by the
PMIX server support library and \textit{MUST NOT} be free'd. Data returned
by the host server via callback function is owned by the host
server, which is free to release it upon return from the callback
\adviceimplend

%%%%%%%%%%%
\subsection{\code{pmix_server_module_t} Module}
\declareapi{pmix_server_module_t}

%%%%
\summary

List of function pointers that a PMIx server passes to \refapi{PMIx_server_init} during startup.

%%%%
\format

\cspecificstart
\begin{codepar}
typedef struct pmix_server_module_2_0_0_t {
    /* v1x interfaces */
    pmix_server_client_connected_fn_t   client_connected;
    pmix_server_client_finalized_fn_t   client_finalized;
    pmix_server_abort_fn_t              abort;
    pmix_server_fencenb_fn_t            fence_nb;
    pmix_server_dmodex_req_fn_t         direct_modex;
    pmix_server_publish_fn_t            publish;
    pmix_server_lookup_fn_t             lookup;
    pmix_server_unpublish_fn_t          unpublish;
    pmix_server_spawn_fn_t              spawn;
    pmix_server_connect_fn_t            connect;
    pmix_server_disconnect_fn_t         disconnect;
    pmix_server_register_events_fn_t    register_events;
    pmix_server_deregister_events_fn_t  deregister_events;
    pmix_server_listener_fn_t           listener;
    /* v2x interfaces */
    pmix_server_notify_event_fn_t       notify_event;
    pmix_server_query_fn_t              query;
    pmix_server_tool_connection_fn_t    tool_connected;
    pmix_server_log_fn_t                log;
    pmix_server_alloc_fn_t              allocate;
    pmix_server_job_control_fn_t        job_control;
    pmix_server_monitor_fn_t            monitor;
} pmix_server_module_t;
\end{codepar}
\cspecificend

%%%%
\descr

\adviceimplstart
For performance purposes, the host server is required to return as quickly as possible from all functions.
Execution of the function is thus to be done asynchronously so as to allow the \ac{PMIx} server support library to handle multiple client requests as quickly and scalably as possible.

All data passed to the host server functions is owned by the \ac{PMIx} server support library and \textit{MUST NOT} be free'd.
Data returned by the host server via callback function is owned by the host server, which is free to release it upon return from the callback.
\adviceimplend


%%%%%%%%%%%
\subsection{\code{pmix_server_client_connected_fn_t}}
\declareapi{pmix_server_client_connected_fn_t}

%%%%
\summary

Notify the host server that a client connected to this server.

%%%%
\format

\versionMarker{1.0}
\cspecificstart
\begin{codepar}
typedef pmix_status_t (*pmix_server_client_connected_fn_t)(
                             const pmix_proc_t *proc,
                             void* server_object,
                             pmix_op_cbfunc_t cbfunc,
                             void *cbdata)
\end{codepar}
\cspecificend

\begin{arglist}
\argin{proc}{\refstruct{pmix_proc_t} structure (handle)}
\argin{server_object}{object reference (memory reference)}
\argin{cbfunc}{Callback function \refapi{pmix_op_cbfunc_t} (function reference)}
\argin{cbdata}{Data to be passed to the callback function (memory reference)}
\end{arglist}

Returns \refconst{PMIX_SUCCESS} or a negative value corresponding to a PMIx error constant.

%%%%
\descr

Notify the host server that a client has called \refapi{PMIx_Init}.
Note that the client will be in a blocked state until the host server executes the callback function, thus allowing the \ac{PMIx} server support library to release
the client.
The server_object parameter will be the value of the server_object parameter passed to
\refapi{PMIx_server_register_client} previously by the host server.  If provided, an implementation of \refapi{pmix_server_client_connected_fn_t}
is only required to
call the callback function designated.  A host server can choose to not be notified when clients connect by setting \refapi{client_connected} to \code{NULL}.

It is possible that only a subset of the clients in a namespace call \refapi{PMIx_init}.   The server's \refapi{pmix_server_client_connected_fn_t} implementation
should not depend on being called once per rank in a namespace or delay calling the callback function until all ranks have connected.
However, if a rank makes any \ac{PMIx} calls, it must first call \refapi{PMIx_Init} and
therefore the server's \refapi{mpix_server_client_connected_fn_t} will be called before any other server functions specific to the rank.

\adviceimplstart
 The \refapi{PMIx_server_client_connected_fn_t} implementation provided in the \refapi{pmix_server_module_2_0_0_t} is an opportunity for a host server
 to update the status of the ranks it manages.  It is also a convenient and well defined time to perform initialization necessary to
 support further calls into the server related to that rank.
 \adviceimplend

%%%%%%%%%%%
\subsection{\code{pmix_server_client_finalized_fn_t}}
\declareapi{pmix_server_client_finalized_fn_t}

%%%%
\summary

Notify the host server that a client called \refapi{PMIx_Finalize}.

%%%%
\format

\versionMarker{1.0}
\cspecificstart
\begin{codepar}
typedef pmix_status_t (*pmix_server_client_finalized_fn_t)(
                             const pmix_proc_t *proc,
                             void* server_object,
                             pmix_op_cbfunc_t cbfunc,
                             void *cbdata)
\end{codepar}
\cspecificend

\begin{arglist}
\argin{proc}{\refstruct{pmix_proc_t} structure (handle)}
\argin{server_object}{object reference (memory reference)}
\argin{cbfunc}{Callback function \refapi{pmix_op_cbfunc_t} (function reference)}
\argin{cbdata}{Data to be passed to the callback function (memory reference)}
\end{arglist}

Returns \refconst{PMIX_SUCCESS} or a negative value corresponding to a \ac{PMIx} error constant.

%%%%
\descr

Notify the host server that a client called \refapi{PMIx_Finalize}.
Note that the client will be in a blocked state until the host server executes the callback function, thus allowing the PMIx server support library to release the client.
The server_object parameter will be the value of the server_object parameter passed to
\refapi{PMIx_server_register_client} previously by the host server.  If provided, an implementation of \refapi{pmix_server_client_finalized_fn_t}
is only required to
call the callback function designated.  A host server can choose to not be notified when clients finalize by setting \refapi{client_finalized} to \code{NULL}.

Note that the host server is only being informed that the client has called \refapi{PMIx_Finalize}.  The client might not have exited.  If a client
exits without calling \refapi{PMIx_Finalize}, the server support library will not call the \refapi{PMIx_server_client_finalized_fn_t} implementation.

\adviceimplstart
 The \refapi{PMIx_server_client_finalized_fn_t} implementation provided in the \refapi{pmix_server_module_2_0_0_t} is an opportunity for a host server
 to update the status of the tasks it manages.  It is also a convenient and well defined time to release resources used to support that client.
 \adviceimplend


%%%%%%%%%%%
\subsection{\code{pmix_server_abort_fn_t}}
\declareapi{pmix_server_abort_fn_t}

%%%%
\summary

Notify PMIx Server that a local client called \refapi{PMIx_Abort}.

%%%%
\format

\versionMarker{1.0}
\cspecificstart
\begin{codepar}
typedef pmix_status_t (*pmix_server_abort_fn_t)(
                             const pmix_proc_t *proc,
                             void *server_object,
                             int status,
                             const char msg[],
                             pmix_proc_t procs[],
                             size_t nprocs,
                             pmix_op_cbfunc_t cbfunc,
                             void *cbdata)
\end{codepar}
\cspecificend


\begin{arglist}
\argin{proc}{\refstruct{pmix_proc_t} structure identifying the process requesting the abort (handle)}
\argin{server_object}{object reference (memory reference)}
\argin{status}{exit status (integer)}
\argin{msg}{exit status message (string)}
\argin{procs}{Array of \refstruct{pmix_proc_t} structures identifying the processes to be terminated (array of handles)}
\argin{nprocs}{Number of elements in the \refarg{procs} array (integer)}
\argin{cbfunc}{Callback function \refapi{pmix_op_cbfunc_t} (function reference)}
\argin{cbdata}{Data to be passed to the callback function (memory reference)}
\end{arglist}

Returns \refconst{PMIX_SUCCESS} or a negative value corresponding to a PMIx error constant.

%%%%
\descr

A local client called \refapi{PMIx_Abort}.
Note that the client will be in a blocked state until the host server executes the callback function, thus allowing the \ac{PMIx} server library to release the client.
The array of \refarg{procs} indicates which processes are to be terminated.
A \code{NULL} indicates that all processes in the client's namespace are to be terminated.


%%%%%%%%%%%
\subsection{\code{pmix_server_fencenb_fn_t}}
\declareapi{pmix_server_fencenb_fn_t}

%%%%
\summary

At least one client called either \refapi{PMIx_Fence} or \refapi{PMIx_Fence_nb}.

%%%%
\format

\versionMarker{1.0}
\cspecificstart
\begin{codepar}
typedef pmix_status_t (*pmix_server_fencenb_fn_t)(
                             const pmix_proc_t procs[],
                             size_t nprocs,
                             const pmix_info_t info[],
                             size_t ninfo,
                             char *data, size_t ndata,
                             pmix_modex_cbfunc_t cbfunc,
                             void *cbdata)
\end{codepar}
\cspecificend

\begin{arglist}
\argin{procs}{Array of \refstruct{pmix_proc_t} structures identifying operation participants(array of handles)}
\argin{nprocs}{Number of elements in the \refarg{procs} array (integer)}
\argin{info}{Array of info structures (array of handles)}
\argin{ninfo}{Number of elements in the \refarg{info} array (integer)}
\argin{data}{(string)}
\argin{ndata}{(integer)}
\argin{cbfunc}{Callback function \refapi{pmix_modex_cbfunc_t} (function reference)}
\argin{cbdata}{Data to be passed to the callback function (memory reference)}
\end{arglist}

Returns \refconst{PMIX_SUCCESS} or a negative value corresponding to a PMIx error constant.

\priattr
The \ac{PRI} internally supports the following attributes - if provided, these attributes will be included in the call to the host \ac{RM} daemon but can be ignored:

\pasteAttributeItem{PMIX_COLLECT_DATA}

\optattr
A complete implementation would include support for the following attributes:

\pasteAttributeItem{PMIX_TIMEOUT}
\pasteAttributeItem{PMIX_COLLECTIVE_ALGO}
\pasteAttributeItem{PMIX_COLLECTIVE_ALGO_REQD}

%%%%
\descr

All local clients in the provided array of \refarg{procs} called either \refapi{PMIx_Fence} or \refapi{PMIx_Fence_nb}.
In either case, the host server will be called via a non-blocking function to execute the specified operation once all participating local processes have contributed.
All processes in the specified \refarg{procs} array are required to participate in the \refapi{PMIx_Fence}/\refapi{PMIx_Fence_nb} operation.
The callback is to be executed once each daemon hosting at least one participant has called the host server's \refapi{pmix_server_fencenb_fn_t} function.

The provided data is to be collectively shared with all PMIx servers involved in the fence operation, and returned in the modex \refarg{cbfunc}.
A \code{NULL} data value indicates that the local processes had no data to contribute.

The array of \refarg{info} structs is used to pass user-requested options to the server.
This can include directives as to the algorithm to be used to execute the fence operation.
The directives are optional \emph{unless} the \emph{mandatory} flag has been set - in such cases, the host \ac{RM} is required to return an error if the directive cannot be met.


%%%%%%%%%%%
\subsection{\code{pmix_server_dmodex_req_fn_t}}
\declareapi{pmix_server_dmodex_req_fn_t}

%%%%
\summary

Used by the PMIx server to request its local host contact the PMIx server on the remote node that hosts the specified proc to obtain and return a direct modex blob for that proc.

%%%%
\format

\versionMarker{1.0}
\cspecificstart
\begin{codepar}
typedef pmix_status_t (*pmix_server_dmodex_req_fn_t)(
                             const pmix_proc_t *proc,
                             const pmix_info_t info[],
                             size_t ninfo,
                             pmix_modex_cbfunc_t cbfunc,
                             void *cbdata)
\end{codepar}
\cspecificend

\begin{arglist}
\argin{proc}{\refstruct{pmix_proc_t} structure identifying the process whose data is being requested (handle)}
\argin{info}{Array of info structures (array of handles)}
\argin{ninfo}{Number of elements in the \refarg{info} array (integer)}
\argin{cbfunc}{Callback function \refapi{pmix_modex_cbfunc_t} (function reference)}
\argin{cbdata}{Data to be passed to the callback function (memory reference)}
\end{arglist}

Returns \refconst{PMIX_SUCCESS} or a negative value corresponding to a PMIx error constant.

\priattr
The \ac{PRI} internally supports the following attributes - if provided, these attributes will be included in the call to the host \ac{RM} daemon but can be ignored:

\pasteAttributeItem{PMIX_IMMEDIATE}

\optattr
A complete implementation would include support for the following attributes:

\pasteAttributeItem{PMIX_TIMEOUT}

%%%%
\descr

Used by the \ac{PMIx} server to request its local host contact the \ac{PMIx} server on the remote node that hosts the specified proc to obtain and return a direct modex blob for that proc.

The array of \refarg{info} structs is used to pass user-requested options to the server.
This can include a timeout to preclude an indefinite wait for data that may never become available.
The directives are optional \emph{unless} the \emph{mandatory} flag has been set - in such cases, the host \ac{RM} is required to return an error if the directive cannot be met.


%%%%%%%%%%%
\subsection{\code{pmix_server_publish_fn_t}}
\declareapi{pmix_server_publish_fn_t}

%%%%
\summary

Publish data per the PMIx API specification.

%%%%
\format

\versionMarker{1.0}
\cspecificstart
\begin{codepar}
typedef pmix_status_t (*pmix_server_publish_fn_t)(
                             const pmix_proc_t *proc,
                             const pmix_info_t info[],
                             size_t ninfo,
                             pmix_op_cbfunc_t cbfunc,
                             void *cbdata)
\end{codepar}
\cspecificend

\begin{arglist}
\argin{proc}{\refstruct{pmix_proc_t} structure of the process publishing the data (handle)}
\argin{info}{Array of info structures (array of handles)}
\argin{ninfo}{Number of elements in the \refarg{info} array (integer)}
\argin{cbfunc}{Callback function \refapi{pmix_op_cbfunc_t} (function reference)}
\argin{cbdata}{Data to be passed to the callback function (memory reference)}
\end{arglist}

Returns \refconst{PMIX_SUCCESS} or a negative value corresponding to a PMIx error constant.

\reqattr
\acp{RM} that implement support this entry point are required to support the following attributes:

\pasteAttributeItem{PMIX_RANGE}
\pasteAttributeItem{PMIX_PERSISTENCE}

\optattr
A complete implementation would include support for the following attributes:

\pasteAttributeItem{PMIX_TIMEOUT}

%%%%
\descr

Publish data per the \refapi{PMIx_Publish} specification.
The callback is to be executed upon completion of the operation.
The default data range is expected to be \refconst{PMIX_SESSION}, and the default persistence \refconst{PMIX_PERSIST_SESSION}.
These values can be modified by including the respective \refstruct{pmix_info_t} struct in the \refarg{info} array.

Note that the host server is not required to guarantee support for any specific range - i.e., the server does not need to return an error if the data store doesn't support range-based isolation.
However, the server must return an error (a) if the key is duplicative within the storage range, and (b) if the server does not allow overwriting of published info by the original publisher - it is left to the discretion of the host server to allow info-key-based flags to modify this behavior.

The persistence indicates how long the server should retain the data.

The identifier of the publishing process is also provided and is expected to be returned on any subsequent lookup request.


%%%%%%%%%%%
\subsection{\code{pmix_server_lookup_fn_t}}
\declareapi{pmix_server_lookup_fn_t}

%%%%
\summary

Lookup published data.

%%%%
\format

\versionMarker{1.0}
\cspecificstart
\begin{codepar}
typedef pmix_status_t (*pmix_server_lookup_fn_t)(
                             const pmix_proc_t *proc,
                             char **keys,
                             const pmix_info_t info[],
                             size_t ninfo,
                             pmix_lookup_cbfunc_t cbfunc,
                             void *cbdata)
\end{codepar}
\cspecificend

\begin{arglist}
\argin{proc}{\refstruct{pmix_proc_t} structure of the process seeking the data (handle)}
\argin{keys}{(array of strings)}
\argin{info}{Array of info structures (array of handles)}
\argin{ninfo}{Number of elements in the \refarg{info} array (integer)}
\argin{cbfunc}{Callback function \refapi{pmix_lookup_cbfunc_t} (function reference)}
\argin{cbdata}{Data to be passed to the callback function (memory reference)}
\end{arglist}

Returns \refconst{PMIX_SUCCESS} or a negative value corresponding to a PMIx error constant.

\reqattr
\acp{RM} that implement support this entry point are required to support the following attributes:

\pasteAttributeItem{PMIX_RANGE}

\optattr
A complete implementation would include support for the following attributes:

\pasteAttributeItem{PMIX_TIMEOUT}
\pasteAttributeItem{PMIX_WAIT}

%%%%
\descr

Lookup published data.
The host server will be passed a \code{NULL}-terminated array of string keys identifying the data being requested.

The array of \refarg{info} structs is used to pass user-requested options to the server.
This can include a wait flag to indicate that the server should wait for all data to become available before executing the callback function, or should immediately callback with whatever data is available.
In addition, a timeout can be specified on the wait to preclude an indefinite wait for data that may never be published.

The identifier of the requesting process is provided to support authorization-based access to published information.


%%%%%%%%%%%
\subsection{\code{pmix_server_unpublish_fn_t}}
\declareapi{pmix_server_unpublish_fn_t}

%%%%
\summary

Delete data from the data store.

%%%%
\format

\versionMarker{1.0}
\cspecificstart
\begin{codepar}
typedef pmix_status_t (*pmix_server_unpublish_fn_t)(
                             const pmix_proc_t *proc,
                             char **keys,
                             const pmix_info_t info[],
                             size_t ninfo,
                             pmix_op_cbfunc_t cbfunc,
                             void *cbdata)
\end{codepar}
\cspecificend

\begin{arglist}
\argin{proc}{\refstruct{pmix_proc_t} structure identifying the process making the request (handle)}
\argin{keys}{(array of strings)}
\argin{info}{Array of info structures (array of handles)}
\argin{ninfo}{Number of elements in the \refarg{info} array (integer)}
\argin{cbfunc}{Callback function \refapi{pmix_op_cbfunc_t} (function reference)}
\argin{cbdata}{Data to be passed to the callback function (memory reference)}
\end{arglist}

Returns \refconst{PMIX_SUCCESS} or a negative value corresponding to a PMIx error constant.

\optattr
A complete implementation would include support for the following attributes:

\pasteAttributeItem{PMIX_TIMEOUT}
\pasteAttributeItem{PMIX_RANGE}

%%%%
\descr

Delete data from the data store.
The host server will be passed a \code{NULL}-terminated array of string keys, plus potential directives such as the data range within which the keys should be deleted.
The callback is to be executed upon completion of the delete procedure.

The identifier of the requesting process is provided to support checks for authority to delete the identified data.

%%%%%%%%%%%
\subsection{\code{pmix_server_spawn_fn_t}}
\declareapi{pmix_server_spawn_fn_t}

%%%%
\summary

Spawn a set of applications/processes as per the \refapi{PMIx_Spawn} API.

%%%%
\format

\versionMarker{1.0}
\cspecificstart
\begin{codepar}
typedef pmix_status_t (*pmix_server_spawn_fn_t)(
                             const pmix_proc_t *proc,
                             const pmix_info_t job_info[],
                             size_t ninfo,
                             const pmix_app_t apps[],
                             size_t napps,
                             pmix_spawn_cbfunc_t cbfunc,
                             void *cbdata)
\end{codepar}
\cspecificend

\begin{arglist}
\argin{proc}{\refstruct{pmix_proc_t} structure of the process making the request (handle)}
\argin{job_info}{Array of info structures (array of handles)}
\argin{ninfo}{Number of elements in the \refarg{jobinfo} array (integer)}
\argin{apps}{Array of \refstruct{pmix_app_t} structures (array of handles)}
\argin{napps}{Number of elements in the \refarg{apps} array (integer)}
\argin{cbfunc}{Callback function \refapi{pmix_spawn_cbfunc_t} (function reference)}
\argin{cbdata}{Data to be passed to the callback function (memory reference)}
\end{arglist}

Returns \refconst{PMIX_SUCCESS} or a negative value corresponding to a PMIx error constant.

\reqattr
\acp{RM} that implement support for \refapi{PMIx_Spawn} are required to support the following attributes:

\pasteAttributeItem{PMIX_WDIR}
\pasteAttributeItem{PMIX_SET_SESSION_CWD}
\pasteAttributeItem{PMIX_PREFIX}
\pasteAttributeItem{PMIX_HOST}
\pasteAttributeItem{PMIX_HOSTFILE}

\optattr
A complete implementation would include support for the following attributes:

\pasteAttributeItem{PMIX_ADD_HOSTFILE}
\pasteAttributeItem{PMIX_ADD_HOST}
\pasteAttributeItem{PMIX_PRELOAD_BIN}
\pasteAttributeItem{PMIX_PRELOAD_FILES}
\pasteAttributeItem{PMIX_PERSONALITY}
\pasteAttributeItem{PMIX_MAPPER}
\pasteAttributeItem{PMIX_DISPLAY_MAP}
\pasteAttributeItem{PMIX_PPR}
\pasteAttributeItem{PMIX_MAPBY}
\pasteAttributeItem{PMIX_RANKBY}
\pasteAttributeItem{PMIX_BINDTO}
\pasteAttributeItem{PMIX_NON_PMI}
\pasteAttributeItem{PMIX_STDIN_TGT}
\pasteAttributeItem{PMIX_FWD_STDIN}
\pasteAttributeItem{PMIX_FWD_STDOUT}
\pasteAttributeItem{PMIX_FWD_STDERR}
\pasteAttributeItem{PMIX_DEBUGGER_DAEMONS}
\pasteAttributeItem{PMIX_TAG_OUTPUT}
\pasteAttributeItem{PMIX_TIMESTAMP_OUTPUT}
\pasteAttributeItem{PMIX_MERGE_STDERR_STDOUT}
\pasteAttributeItem{PMIX_OUTPUT_TO_FILE}
\pasteAttributeItem{PMIX_INDEX_ARGV}
\pasteAttributeItem{PMIX_CPUS_PER_PROC}
\pasteAttributeItem{PMIX_NO_PROCS_ON_HEAD}
\pasteAttributeItem{PMIX_NO_OVERSUBSCRIBE}
\pasteAttributeItem{PMIX_REPORT_BINDINGS}
\pasteAttributeItem{PMIX_CPU_LIST}
\pasteAttributeItem{PMIX_JOB_RECOVERABLE}
\pasteAttributeItem{PMIX_JOB_CONTINUOUS}
\pasteAttributeItem{PMIX_MAX_RESTARTS}

%%%%
\descr

Spawn a set of applications/processes as per the \refapi{PMIx_Spawn} API.
Note that applications are not required to be \ac{MPI} or any other programming model.
Thus, the host server cannot make any assumptions as to their required support.
The callback function is to be executed once all processes have been started.
An error in starting any application or process in this request shall cause all applications and processes in the request to be terminated, and an error returned to the originating caller.

Note that a timeout can be specified in the job_info array to indicate that failure to start the requested job within the given time should result in termination to avoid hangs.


%%%%%%%%%%%
\subsection{\code{pmix_server_connect_fn_t}}
\declareapi{pmix_server_connect_fn_t}

%%%%
\summary

Record the specified processes as ``connected''.

%%%%
\format

\versionMarker{1.0}
\cspecificstart
\begin{codepar}
typedef pmix_status_t (*pmix_server_connect_fn_t)(
                             const pmix_proc_t procs[],
                             size_t nprocs,
                             const pmix_info_t info[],
                             size_t ninfo,
                             pmix_op_cbfunc_t cbfunc,
                             void *cbdata)
\end{codepar}
\cspecificend

\begin{arglist}
\argin{procs}{Array of \refstruct{pmix_proc_t} structures identifying participants (array of handles)}
\argin{nprocs}{Number of elements in the \refarg{procs} array (integer)}
\argin{info}{Array of info structures (array of handles)}
\argin{ninfo}{Number of elements in the \refarg{info} array (integer)}
\argin{cbfunc}{Callback function \refapi{pmix_op_cbfunc_t} (function reference)}
\argin{cbdata}{Data to be passed to the callback function (memory reference)}
\end{arglist}

Returns \refconst{PMIX_SUCCESS} or a negative value corresponding to a \ac{PMIx} error constant.

\optattr
A complete implementation would include support for the following attributes:

\pasteAttributeItem{PMIX_TIMEOUT}

%%%%
\descr

Record the specified processes as ``connected''.
This means that the resource manager should treat the failure of any process in the specified group as a reportable event, and take appropriate action.
The callback function is to be called once all participating processes have called \refapi{PMIx_Connect} or its non-blocking form.
Note that a process can only engage in \textbf{one} connect operation involving the identical set of processes at a time.
However, a process \emph{can} be simultaneously engaged in multiple connect operations, each involving a different set of processes.

Note also that this is a collective operation within the client's PMIx library, and thus the client will be blocked until all processes participate. The \ac{PMIx} server will
call this function once all local participants have called \refapi{PMIx_Connect} or its non-blocking form with the same array of participants.
The \refarg{info} array can be used to pass user directives, including a timeout.
The directives are optional \emph{unless} the \emph{mandatory} flag has been set - in such cases, the host RM is required to return an error if the directive cannot be met.


%%%%%%%%%%%
\subsection{\code{pmix_server_disconnect_fn_t}}
\declareapi{pmix_server_disconnect_fn_t}

%%%%
\summary

Disconnect a previously connected set of processes.

%%%%
\format

\versionMarker{1.0}
\cspecificstart
\begin{codepar}
typedef pmix_status_t (*pmix_server_disconnect_fn_t)(
                             const pmix_proc_t procs[],
                             size_t nprocs,
                             const pmix_info_t info[],
                             size_t ninfo,
                             pmix_op_cbfunc_t cbfunc,
                             void *cbdata)
\end{codepar}
\cspecificend

\begin{arglist}
\argin{procs}{Array of \refstruct{pmix_proc_t} structures identifying participants (array of handles)}
\argin{nprocs}{Number of elements in the \refarg{procs} array (integer)}
\argin{info}{Array of info structures (array of handles)}
\argin{ninfo}{Number of elements in the \refarg{info} array (integer)}
\argin{cbfunc}{Callback function \refapi{pmix_op_cbfunc_t} (function reference)}
\argin{cbdata}{Data to be passed to the callback function (memory reference)}
\end{arglist}

Returns \refconst{PMIX_SUCCESS} or a negative value corresponding to a \ac{PMIx} error constant.

\optattr
A complete implementation would include support for the following attributes:

\pasteAttributeItem{PMIX_TIMEOUT}

%%%%
\descr

Disconnect a previously connected set of processes.
An error should be returned if the specified set of processes was not previously ``connected''.
As above, a process may be involved in multiple simultaneous disconnect operations.
However, a process is not allowed to reconnect to a set of ranges that has not fully completed disconnect (i.e., you have to fully disconnect before you can reconnect to the same group of processes).

Note also that this is a collective operation within the client's \ac{PMIx} library, and thus the client will be blocked until all processes participate. The \ac{PMIx} server will
call this function once all local participants have called \refapi{PMIx_Disonnect} or its non-blocking form with the same array of participants.
The \refarg{info} array can be used to pass user directives, including a timeout.
The directives are optional \emph{unless} the \emph{mandatory} flag has been set - in such cases, the host RM is required to return an error if the directive cannot be met.


%%%%%%%%%%%
\subsection{\code{pmix_server_register_events_fn_t}}
\declareapi{pmix_server_register_events_fn_t}

%%%%
\summary

Register to receive notifications for the specified events.

%%%%
\format

\versionMarker{1.0}
\cspecificstart
\begin{codepar}
 typedef pmix_status_t (*pmix_server_register_events_fn_t)(
                              pmix_status_t *codes,
                              size_t ncodes,
                              const pmix_info_t info[],
                              size_t ninfo,
                              pmix_op_cbfunc_t cbfunc,
                              void *cbdata)
\end{codepar}
\cspecificend

\begin{arglist}
\argin{codes}{Array of \refstruct{pmix_status_t} values (array of handles)}
\argin{ncodes}{Number of elements in the \refarg{codes} array (integer)}
\argin{info}{Array of info structures (array of handles)}
\argin{ninfo}{Number of elements in the \refarg{info} array (integer)}
\argin{cbfunc}{Callback function \refapi{pmix_op_cbfunc_t} (function reference)}
\argin{cbdata}{Data to be passed to the callback function (memory reference)}
\end{arglist}

Returns \refconst{PMIX_SUCCESS} or a negative value corresponding to a PMIx error constant.

\reqattr
\acp{RM} that implement support for \ac{PMIx} event notification are required to support the following attributes:

\pasteAttributeItem{PMIX_EVENT_JOB_LEVEL}
\pasteAttributeItem{PMIX_EVENT_ENVIRO_LEVEL}
\pasteAttributeItem{PMIX_EVENT_AFFECTED_PROC}
\pasteAttributeItem{PMIX_EVENT_AFFECTED_PROCS}

\optattr
A complete implementation that supports \ac{PMIx} event notification would offer notifications for environmental events impacting the job, for \ac{SMS} events relating to the job, and include support for the following attributes:

\pasteAttributeItem{PMIX_EVENT_TERMINATE_SESSION}
\pasteAttributeItem{PMIX_EVENT_TERMINATE_JOB}
\pasteAttributeItem{PMIX_EVENT_TERMINATE_NODE}
\pasteAttributeItem{PMIX_EVENT_TERMINATE_PROC}
\pasteAttributeItem{PMIX_EVENT_ACTION_TIMEOUT}
\pasteAttributeItem{PMIX_EVENT_SILENT_TERMINATION}

%%%%
\descr

Register to receive notifications for the specified events.
The resource manager is \emph{required} to pass along to the local \ac{PMIx} server all events that directly relate to a registered namespace without the \ac{PMIx} server library explicitly requesting them.
However, the \ac{RM} may have access to events beyond those (e.g., environmental events).
The \ac{PMIx} server will register to receive environmental events that match specific \ac{PMIx} event codes.
If the host \ac{RM} supports such notifications, it will need to translate its own internal event codes to fit into a corresponding \ac{PMIx} event code - any specific info beyond that can be passed in via the \refstruct{pmix_info_t} upon notification.

The \refarg{info} array included in this API is reserved for possible future directives to further steer notification.



%%%%%%%%%%%
\subsection{\code{pmix_server_deregister_events_fn_t}}
\declareapi{pmix_server_deregister_events_fn_t}

%%%%
\summary

Deregister to receive notifications for the specified events.

%%%%
\format

\versionMarker{1.0}
\cspecificstart
\begin{codepar}
 typedef pmix_status_t (*pmix_server_deregister_events_fn_t)(
                              pmix_status_t *codes,
                              size_t ncodes,
                              pmix_op_cbfunc_t cbfunc,
                              void *cbdata)
\end{codepar}
\cspecificend

\begin{arglist}
\argin{codes}{Array of \refstruct{pmix_status_t} values (array of handles)}
\argin{ncodes}{Number of elements in the \refarg{codes} array (integer)}
\argin{cbfunc}{Callback function \refapi{pmix_op_cbfunc_t} (function reference)}
\argin{cbdata}{Data to be passed to the callback function (memory reference)}
\end{arglist}

Returns \refconst{PMIX_SUCCESS} or a negative value corresponding to a PMIx error constant.

%%%%
\descr

Deregister to receive notifications for the specified environmental events for which the \ac{PMIx} server has previously registered.
The host \ac{RM} remains required to notify of any namespace-related events.


%%%%%%%%%%%
\subsection{\code{pmix_server_notify_event_fn_t}}
\declareapi{pmix_server_notify_event_fn_t}

%%%%
\summary

Notify the specified processes of an event.

%%%%
\format

\versionMarker{2.0}
\cspecificstart
\begin{codepar}
typedef pmix_status_t (*pmix_server_notify_event_fn_t)(pmix_status_t code,
                             const pmix_proc_t *source,
                             pmix_data_range_t range,
                             pmix_info_t info[],
                             size_t ninfo,
                             pmix_op_cbfunc_t cbfunc,
                             void *cbdata);
\end{codepar}
\cspecificend

\begin{arglist}
\argin{code}{The \refstruct{pmix_status_t} event code being referenced structure (handle)}
\argin{source}{\refstruct{pmix_proc_t} of process that generated the event (handle)}
\argin{range}{\refstruct{pmix_data_range_t} range over which the event is to be distributed (handle)}
\argin{info}{Optional array of \refstruct{pmix_info_t} structures containing additional information on the event (array of handles)}
\argin{ninfo}{Number of elements in the \refarg{info} array (integer)}
\argin{cbfunc}{Callback function \refapi{pmix_op_cbfunc_t} (function reference)}
\argin{cbdata}{Data to be passed to the callback function (memory reference)}
\end{arglist}

Returns \refconst{PMIX_SUCCESS} or a negative value corresponding to a \ac{PMIx} error constant.

%%%%
\descr

Notify the specified processes of an event generated either by the \ac{PMIx} server itself, or by one of its local clients.
The process generating the event is provided in the \refarg{source} parameter, and any further descriptive information is
included in the \refarg{info} array.

\adviceimplstart
The callback function is to be executed once the host \ac{RM} no longer requires that the \ac{PMIx} server library maintain the provided data structures. It does \emph{not} necessarily indicate that the event has been delivered to any process, nor that the event has been distributed for delivery
\adviceimplend


%%%%%%%%%%%
\subsection{\code{pmix_connection_cbfunc_t}}
\declareapi{pmix_connection_cbfunc_t}

%%%%
\summary

Callback function for incoming connection requests from local clients.

%%%%
\format

\versionMarker{1.0}
\cspecificstart
\begin{codepar}
typedef void (*pmix_connection_cbfunc_t)(
                             int incoming_sd, void *cbdata)
\end{codepar}
\cspecificend

\begin{arglist}
\argin{incoming_sd}{(integer)}
\argin{cbdata}{ (memory reference)}
\end{arglist}

Returns \refconst{PMIX_SUCCESS} or a negative value corresponding to a PMIx error constant.

%%%%
\descr

Callback function for incoming connection requests from local clients - only used by host \acp{RM} that wish to directly handle socket connection requests.


%%%%%%%%%%%
\subsection{\code{pmix_server_listener_fn_t}}
\declareapi{pmix_server_listener_fn_t}

%%%%
\summary

Register a socket the host server can monitor for connection requests.

%%%%
\format

\versionMarker{1.0}
\cspecificstart
\begin{codepar}
typedef pmix_status_t (*pmix_server_listener_fn_t)(
                             int listening_sd,
                             pmix_connection_cbfunc_t cbfunc,
                             void *cbdata)
\end{codepar}
\cspecificend

\begin{arglist}
\argin{incoming_sd}{(integer)}
\argin{cbfunc}{Callback function \refapi{pmix_connection_cbfunc_t} (function reference)}
\argin{cbdata}{ (memory reference)}
\end{arglist}

Returns \refconst{PMIX_SUCCESS} or a negative value corresponding to a PMIx error constant.

%%%%
\descr

Register a socket the host server can monitor for connection requests, harvest them, and then call the \ac{PMIx} server library's internal callback function for further processing.
A listener thread is essential to efficiently harvesting connection requests from large numbers of local clients such as occur when running on large SMPs.
The host server listener is required to call accept on the incoming connection request, and then pass the resulting socket to the provided cbfunc.
A \code{NULL} for this function will cause the internal \ac{PMIx} server to spawn its own listener thread.


%%%%%%%%%%%
\subsection{\code{pmix_server_query_fn_t}}
\declareapi{pmix_server_query_fn_t}

%%%%
\summary

Query information from the resource manager.

%%%%
\format

\versionMarker{2.0}
\cspecificstart
\begin{codepar}
typedef pmix_status_t (*pmix_server_query_fn_t)(
                             pmix_proc_t *proct,
                             pmix_query_t *queries, size_t nqueries,
                             pmix_info_cbfunc_t cbfunc,
                             void *cbdata)
\end{codepar}
\cspecificend

\begin{arglist}
\argin{proct}{\refstruct{pmix_proc_t} structure of the requesting process (handle)}
\argin{queries}{Array of \refstruct{pmix_query_t} structures (array of handles)}
\argin{nqueries}{Number of elements in the \refarg{queries} array (integer)}
\argin{cbfunc}{Callback function \refapi{pmix_info_cbfunc_t} (function reference)}
\argin{cbdata}{Data to be passed to the callback function (memory reference)}
\end{arglist}

Returns \refconst{PMIX_SUCCESS} or a negative value corresponding to a PMIx error constant.

\optattr
A complete implementation would include support for the following query attributes:

\pasteAttributeItem{PMIX_QUERY_NAMESPACES}
\pasteAttributeItem{PMIX_QUERY_JOB_STATUS}
\pasteAttributeItem{PMIX_QUERY_QUEUE_LIST}
\pasteAttributeItem{PMIX_QUERY_QUEUE_STATUS}
\pasteAttributeItem{PMIX_QUERY_PROC_TABLE}
\pasteAttributeItem{PMIX_QUERY_LOCAL_PROC_TABLE}
\pasteAttributeItem{PMIX_QUERY_SPAWN_SUPPORT}
\pasteAttributeItem{PMIX_QUERY_DEBUG_SUPPORT}
\pasteAttributeItem{PMIX_QUERY_MEMORY_USAGE}
\pasteAttributeItem{PMIX_QUERY_LOCAL_ONLY}
\pasteAttributeItem{PMIX_QUERY_REPORT_AVG}
\pasteAttributeItem{PMIX_QUERY_REPORT_MINMAX}
\pasteAttributeItem{PMIX_QUERY_ALLOC_STATUS}
\pasteAttributeItem{PMIX_TIME_REMAINING}

%%%%
\descr

Query information from the resource manager.
The query will include the nspace/rank of the process that is requesting the info, an array of \refstruct{pmix_query_t} describing the request, and a callback function/data for the return.


%%%%%%%%%%%
\subsection{\code{pmix_tool_connection_cbfunc_t}}
\declareapi{pmix_tool_connection_cbfunc_t}

%%%%
\summary

Callback function for incoming tool connections.

%%%%
\format

\versionMarker{2.0}
\cspecificstart
\begin{codepar}
typedef void (*pmix_tool_connection_cbfunc_t)(
                             pmix_status_t status,
                             pmix_proc_t *proc, void *cbdata)
\end{codepar}
\cspecificend

\begin{arglist}
\argin{status}{\refstruct{pmix_status_t} value (handle)}
\argin{proc}{\refstruct{pmix_proc_t} structure containing the identifier assigned to the tool (handle)}
\argin{cbdata}{Data to be passed (memory reference)}
\end{arglist}

%%%%
\descr

Callback function for incoming tool connections.
The host \ac{RM} shall provide an nspace/rank for the connecting tool.
We assume that a \code{rank=0} will be the normal assignment, but allow for the future possibility of a parallel set of tools connecting, and thus each proc requiring a rank.


%%%%%%%%%%%
\subsection{\code{pmix_server_tool_connection_fn_t}}
\declareapi{pmix_server_tool_connection_fn_t}

%%%%
\summary

Register that a tool has connected to the server.

%%%%
\format

\versionMarker{2.0}
\cspecificstart
\begin{codepar}
typedef void (*pmix_server_tool_connection_fn_t)(
                             pmix_info_t *info, size_t ninfo,
                             pmix_tool_connection_cbfunc_t cbfunc,
                             void *cbdata)
\end{codepar}
\cspecificend

\begin{arglist}
\argin{info}{Array of info structures (array of handles)}
\argin{ninfo}{Number of elements in the \refarg{info} array (integer)}
\argin{cbfunc}{Callback function \refapi{pmix_tool_connection_cbfunc_t} (function reference)}
\argin{cbdata}{Data to be passed to the callback function (memory reference)}
\end{arglist}

\optattr
A complete implementation would include support for the following attributes:

\pasteAttributeItem{PMIX_USERID}
\pasteAttributeItem{PMIX_GRPID}
\pasteAttributeItem{PMIX_FWD_STDOUT}
\pasteAttributeItem{PMIX_FWD_STDERR}
\pasteAttributeItem{PMIX_FWD_STDIN}


%%%%
\descr

Register that a tool has connected to the server, and request that the tool be assigned an nspace/rank for further interactions.
The \refstruct{pmix_info_t} array is used to pass qualifiers for the connection request, including the effective uid and gid of the calling tool for authentication purposes.

\adviceimplstart
The host \ac{RM} is solely responsible for authenticating and authorizing the connection, and for authorizing all subsequent tool requests.
\adviceimplend


%%%%%%%%%%%
\subsection{\code{pmix_server_log_fn_t}}
\declareapi{pmix_server_log_fn_t}

%%%%
\summary

Log data on behalf of a client.

%%%%
\format

\versionMarker{2.0}
\cspecificstart
\begin{codepar}
typedef void (*pmix_server_log_fn_t)(
                             const pmix_proc_t *client,
                             const pmix_info_t data[], size_t ndata,
                             const pmix_info_t directives[], size_t ndirs,
                             pmix_op_cbfunc_t cbfunc, void *cbdata)
\end{codepar}
\cspecificend

\begin{arglist}
\argin{client}{\refstruct{pmix_proc_t} structure (handle)}
\argin{data}{Array of info structures (array of handles)}
\argin{ndata}{Number of elements in the \refarg{data} array (integer)}
\argin{directives}{Array of info structures (array of handles)}
\argin{ndirs}{Number of elements in the \refarg{directives} array (integer)}
\argin{cbfunc}{Callback function \refapi{pmix_op_cbfunc_t} (function reference)}
\argin{cbdata}{Data to be passed to the callback function (memory reference)}
\end{arglist}


\reqattr
\acp{RM} that implement support for the \refapi{PMIx_Log_nb} are required to support the following attributes:

\pasteAttributeItem{PMIX_LOG_STDERR}
\pasteAttributeItem{PMIX_LOG_STDOUT}
\pasteAttributeItem{PMIX_LOG_SYSLOG}

\optattr
A complete implementation that supports the \refapi{PMIx_Log_nb} would include support for the following attributes:

\pasteAttributeItem{PMIX_LOG_MSG}
\pasteAttributeItem{PMIX_LOG_EMAIL}
\pasteAttributeItem{PMIX_LOG_EMAIL_ADDR}
\pasteAttributeItem{PMIX_LOG_EMAIL_SUBJECT}
\pasteAttributeItem{PMIX_LOG_EMAIL_MSG}

%%%%
\descr

Log data on behalf of a client. This function is \textit{not} intended for output of computational results, but rather for reporting status and error messages.


%%%%%%%%%%%
\subsection{\code{pmix_server_alloc_fn_t}}
\declareapi{pmix_server_alloc_fn_t}

%%%%
\summary

Request allocation modifications on behalf of a client.

%%%%
\format

\versionMarker{2.0}
\cspecificstart
\begin{codepar}
typedef pmix_status_t (*pmix_server_alloc_fn_t)(
                             const pmix_proc_t *client,
                             pmix_alloc_directive_t directive,
                             const pmix_info_t data[], size_t ndata,
                             pmix_info_cbfunc_t cbfunc, void *cbdata)
\end{codepar}
\cspecificend

\begin{arglist}
\argin{client}{\refstruct{pmix_proc_t} structure of process making request (handle)}
\argin{directive}{Specific action being requested (\refstruct{pmix_alloc_directive_t})}
\argin{data}{Array of info structures (array of handles)}
\argin{ndata}{Number of elements in the \refarg{data} array (integer)}
\argin{cbfunc}{Callback function \refapi{pmix_info_cbfunc_t} (function reference)}
\argin{cbdata}{Data to be passed to the callback function (memory reference)}
\end{arglist}

Returns \refconst{PMIX_SUCCESS} or a negative value corresponding to a PMIx error constant.

\reqattr
\acp{RM} that implement support for the \refapi{PMIx_Allocation_request_nb} are required to support the following attributes:

\pasteAttributeItem{PMIX_ALLOC_ID}
\pasteAttributeItem{PMIX_ALLOC_NUM_NODES}
\pasteAttributeItem{PMIX_ALLOC_NUM_CPUS}
\pasteAttributeItem{PMIX_ALLOC_TIME}

\optattr
A complete implementation that supports the \refapi{PMIx_Allocation_request_nb} would include support for the following attributes:

\pasteAttributeItem{PMIX_ALLOC_NODE_LIST}
\pasteAttributeItem{PMIX_ALLOC_NUM_CPU_LIST}
\pasteAttributeItem{PMIX_ALLOC_CPU_LIST}
\pasteAttributeItem{PMIX_ALLOC_MEM_SIZE}
\pasteAttributeItem{PMIX_ALLOC_NETWORK}
\pasteAttributeItem{PMIX_ALLOC_NETWORK_ID}
\pasteAttributeItem{PMIX_ALLOC_BANDWIDTH}
\pasteAttributeItem{PMIX_ALLOC_NETWORK_QOS}

%%%%
\descr

Request new allocation or modifications to an existing allocation on behalf of a client. Several broad categories are envisioned, including the ability to:

\begin{compactitem}
%
\item Request allocation of additional resources, including memory, bandwidth, and compute.
This should be accomplished in a non-blocking manner so that the application can continue to progress while waiting for resources to become available.
Note that the new allocation will be disjoint from (i.e., not affiliated with) the allocation of the requestor - thus the termination of one allocation will not impact the other.
%
\item Extend the reservation on currently allocated resources, subject to scheduling availability and priorities.
This includes extending the time limit on current resources, and/or requesting additional resources be allocated to the requesting job.
Any additional allocated resources will be considered as part of the current allocation, and thus will be released at the same time.
%
\item Return no-longer-required resources to the scheduler.
This includes the ``loan'' of resources back to the scheduler with a promise to return them upon subsequent request.
\end{compactitem}

The callback function provides a \refarg{status} to indicate whether or not the request was granted, and to provide some information as to the reason for any denial in the \refapi{pmix_info_cbfunc_t} array of \refstruct{pmix_info_t} structures.


%%%%%%%%%%%
\subsection{\code{pmix_server_job_control_fn_t}}
\declareapi{pmix_server_job_control_fn_t}

%%%%
\summary

Execute a job control action on behalf of a client.

%%%%
\format

\versionMarker{2.0}
\cspecificstart
\begin{codepar}
typedef pmix_status_t (*pmix_server_job_control_fn_t)(
                             const pmix_proc_t *requestor,
                             const pmix_proc_t targets[], size_t ntargets,
                             const pmix_info_t directives[], size_t ndirs,
                             pmix_info_cbfunc_t cbfunc, void *cbdata)
\end{codepar}
\cspecificend

\begin{arglist}
\argin{requestor}{\refstruct{pmix_proc_t} structure of requesting process (handle)}
\argin{targets}{Array of proc structures (array of handles)}
\argin{ntargets}{Number of elements in the \refarg{targets} array (integer)}
\argin{directives}{Array of info structures (array of handles)}
\argin{ndirs}{Number of elements in the \refarg{info} array (integer)}
\argin{cbfunc}{Callback function \refapi{pmix_op_cbfunc_t} (function reference)}
\argin{cbdata}{Data to be passed to the callback function (memory reference)}
\end{arglist}

Returns \refconst{PMIX_SUCCESS} or a negative value corresponding to a PMIx error constant.

\reqattr
\acp{RM} that implement support for the \refapi{PMIx_Job_control_nb} are required to support the following attributes:

\pasteAttributeItem{PMIX_JOB_CTRL_ID}
\pasteAttributeItem{PMIX_JOB_CTRL_PAUSE}
\pasteAttributeItem{PMIX_JOB_CTRL_RESUME}
\pasteAttributeItem{PMIX_JOB_CTRL_KILL}
\pasteAttributeItem{PMIX_JOB_CTRL_SIGNAL}
\pasteAttributeItem{PMIX_JOB_CTRL_TERMINATE}

\optattr
A complete implementation that supports the \refapi{PMIx_Job_control_nb} would include support for the following attributes:

\pasteAttributeItem{PMIX_JOB_CTRL_CANCEL}
\pasteAttributeItem{PMIX_JOB_CTRL_RESTART}
\pasteAttributeItem{PMIX_JOB_CTRL_CHECKPOINT}
\pasteAttributeItem{PMIX_JOB_CTRL_CHECKPOINT_EVENT}
\pasteAttributeItem{PMIX_JOB_CTRL_CHECKPOINT_SIGNAL}
\pasteAttributeItem{PMIX_JOB_CTRL_CHECKPOINT_TIMEOUT}
\pasteAttributeItem{PMIX_JOB_CTRL_CHECKPOINT_METHOD}
\pasteAttributeItem{PMIX_JOB_CTRL_PROVISION}
\pasteAttributeItem{PMIX_JOB_CTRL_PROVISION_IMAGE}
\pasteAttributeItem{PMIX_JOB_CTRL_PREEMPTIBLE}

%%%%
\descr

Execute a job control action on behalf of a client. The \refarg{targets} array identifies the processes to which the requested job control action is to be applied.
A \code{NULL} value can be used to indicate all processes in the caller's namespace.
The use of \refconst{PMIX_RANK_WILDARD} can also be used to indicate that all processes in the given namespace are to be included.

The directives are provided as \refstruct{pmix_info_t} structures in the \refarg{directives} array.
The callback function provides a \refarg{status} to indicate whether or not the request was granted, and to provide some information as to the reason for any denial in the \refapi{pmix_info_cbfunc_t} array of \refstruct{pmix_info_t} structures.


%%%%%%%%%%%
\subsection{\code{pmix_server_monitor_fn_t}}
\declareapi{pmix_server_monitor_fn_t}

%%%%
\summary

Request that a client be monitored for activity.

%%%%
\format

\versionMarker{2.0}
\cspecificstart
\begin{codepar}
/* Request that a client be monitored for activity */
typedef pmix_status_t (*pmix_server_monitor_fn_t)(
                             const pmix_proc_t *requestor,
                             const pmix_info_t *monitor, pmix_status_t error,
                             const pmix_info_t directives[], size_t ndirs,
                             pmix_info_cbfunc_t cbfunc, void *cbdata);
\end{codepar}
\cspecificend

\begin{arglist}
\argin{requestor}{\refstruct{pmix_proc_t} structure of requesting process (handle)}
\argin{monitor}{\refstruct{pmix_info_t} identifying the type of monitor being requested (handle)}
\argin{error}{Status code to use in generating event if alarm triggers (integer)}
\argin{directives}{Array of info structures (array of handles)}
\argin{ndirs}{Number of elements in the \refarg{info} array (integer)}
\argin{cbfunc}{Callback function \refapi{pmix_op_cbfunc_t} (function reference)}
\argin{cbdata}{Data to be passed to the callback function (memory reference)}
\end{arglist}

Returns \refconst{PMIX_SUCCESS} or a negative value corresponding to a PMIx error constant. This entry point is only called for monitoring requests that are not directly supported by the \ac{PRI}.

\priattr
The \ac{PRI} internally recognizes the following monitoring types:

\pasteAttributeItem{PMIX_MONITOR_FILE}
\pasteAttributeItem{PMIX_MONITOR_HEARTBEAT}

The \ac{PRI} internally supports the following attributes - if provided, these attributes will be included in the call to the host \ac{RM} daemon but can be ignored:

\pasteAttributeItem{PMIX_MONITOR_FILE_SIZE}
\pasteAttributeItem{PMIX_MONITOR_FILE_ACCESS}
\pasteAttributeItem{PMIX_MONITOR_FILE_MODIFY}
\pasteAttributeItem{PMIX_MONITOR_FILE_DROPS}
\pasteAttributeItem{PMIX_MONITOR_FILE_CHECK_TIME}
\pasteAttributeItem{PMIX_MONITOR_HEARTBEAT_TIME}
\pasteAttributeItem{PMIX_MONITOR_HEARTBEAT_DROPS}
\pasteAttributeItem{PMIX_RANGE}

%%%%
\descr

Request that a client be monitored for activity.

\adviceimplstart
Given that \ac{PRI}-recognized monitoring requests are handled within the \ac{PRI} server itself, it is expected that monitoring requests passed to the host \ac{RM} daemon will be environment-specific. The host will \emph{only} be called if an unrecognized monitoring type is specified.
\adviceimplend

%%%%%%%%%%%%%%%%%%%%%%%%%%%%%%%%%%%%%%%%%%%%%%%%%
