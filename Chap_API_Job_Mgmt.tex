%%%%%%%%%%%%%%%%%%%%%%%%%%%%%%%%%%%%%%%%%%%%%%%%%
% Chapter: Job Allocation Management
%%%%%%%%%%%%%%%%%%%%%%%%%%%%%%%%%%%%%%%%%%%%%%%%%
\chapter{Job Allocation Management and Reporting}
\label{chap:api_job_mgmt}

Detecting stalled processes and receiving notification of external faults is, of course, only part of the problem. The application
must then work with the \ac{SMS} to actually do something about the situation. Typical current practice is to simply
terminate the impacted application, but future systems require less draconian solutions.

The \refapi{PMIx_Job_control_nb} \ac{API} enables the application and \ac{SMS} to coordinate the response to failures and other events.
This can include requesting termination of the entire job or a subset of processes within a job, but can
also be used in combination with other \ac{PMIx} capabilities (e.g., allocation support and event notification) for more nuanced responses. For example, an application notified of an incipient over-temperature condition on a node could use the \refapi{PMIx_Allocation_request_nb} interface to request replacement nodes while simultaneously using the \refapi{PMIx_Job_control_nb} interface to direct that a checkpoint event be delivered to all processes in the application. If replacement resources are not available, the application might use the \refapi{PMIx_Job_control_nb} interface to request that the job continue at a lower power setting, perhaps sufficient to avoid the over-temperature failure.

The job control API can also be used by an application to register itself as available for preemption when operating in an environment such as a cloud or where incentives, financial or otherwise, are provided to jobs willing to be preempted. Registration can include attributes indicating how many resources are being offered for preemption (e.g., all or only some portion), whether the application will require time to prepare for preemption, etc. Jobs that
request a warning will receive an event notifying them of an impending preemption (possibly including information as to the resources that will be taken away, how much time the application will be given prior to being preempted, whether the preemption will be a suspension or full termination, etc.) so they have an opportunity to save
their work. Once the application is ready, it calls the provided event completion callback function to indicate that
the SMS is free to suspend or terminate it, and can include directives regarding any desired restart.

Additional uses of the job control API continue to emerge. Recent additions include: (a) register temporary files and directories for cleanup upon termination of an individual process or the entire application to ensure complete cleanup even under abnormal termination conditions; (b) register a more generalized epilogue to be executed; (c) coordination of checkpoint execution; (d) passing of specified signals (kill, pause, resume, etc.) to specific processes or the entire application; and (e) request for provisioning of a specific image on one or more nodes.


%%%%%%%%%%%%%%%%%%%%%%%%%%%%%%%%%%%%%%%%%%%%%%
%%%%%%%%%%%%%%%%%%%%%%%%%%%%%%%%%%%%%%%%%%%%%%
\section{Allocation Requests}
\label{chap:api_job_mgmt:alloc}

This section defines functionality to request allocations from the \ac{RM}, and request modifications to existing allocations. 

%%%%%%%%%%%
\subsection{\code{PMIx_Allocation_request_nb}}
\declareapi{PMIx_Allocation_request_nb}

%%%%
\summary

Request an allocation operation from the host resource manager.

%%%%
\format

\cspecificstart
\begin{codepar}
pmix_status_t
PMIx_Allocation_request_nb(pmix_alloc_directive_t directive,
                           pmix_info_t *info, size_t ninfo,
                           pmix_info_cbfunc_t cbfunc, void *cbdata);
\end{codepar}
\cspecificend

\begin{arglist}
\argin{directive}{Allocation directive (handle)}
\argin{info}{Array of info structures (array of handles)}
\argin{ninfo}{Number of elements in the \refarg{info} array (integer)}
\argin{cbfunc}{Callback function \refapi{pmix_info_cbfunc_t} (function reference)}
\argin{cbdata}{Data to be passed to the callback function (memory reference)}
\end{arglist}

Returns \refconst{PMIX_SUCCESS} or a negative value corresponding to a PMIx error constant.

\priattr
The PMIx_Allocation_request_nb function in the \ac{PRI} does not parse or utilize the attributes directly. However, the attributes are passed to the host RM for processing.

\reqattr
\acp{RM} that implement support for the \refapi{PMIx_Allocation_request_nb} are required to support the following attributes:

\pasteAttributeItem{PMIX_ALLOC_ID}
\pasteAttributeItem{PMIX_ALLOC_NUM_NODES}
\pasteAttributeItem{PMIX_ALLOC_NUM_CPUS}
\pasteAttributeItem{PMIX_ALLOC_TIME}

\optattr
A complete implementation that supports the \refapi{PMIx_Allocation_request_nb} would include support for the following attributes:

\pasteAttributeItem{PMIX_ALLOC_NODE_LIST}
\pasteAttributeItem{PMIX_ALLOC_NUM_CPU_LIST}
\pasteAttributeItem{PMIX_ALLOC_CPU_LIST}
\pasteAttributeItem{PMIX_ALLOC_MEM_SIZE}
\pasteAttributeItem{PMIX_ALLOC_NETWORK}
\pasteAttributeItem{PMIX_ALLOC_NETWORK_ID}
\pasteAttributeItem{PMIX_ALLOC_BANDWIDTH}
\pasteAttributeItem{PMIX_ALLOC_NETWORK_QOS}


%%%%
\descr

Request an allocation operation from the host resource manager.
Several broad categories are envisioned, including the ability to:

\begin{compactitem}
%
\item Request allocation of additional resources, including memory, bandwidth, and compute.
This should be accomplished in a non-blocking manner so that the application can continue to progress while waiting for resources to become available.
Note that the new allocation will be disjoint from (i.e., not affiliated with) the allocation of the requestor - thus the termination of one allocation will not impact the other.
%
\item Extend the reservation on currently allocated resources, subject to scheduling availability and priorities.
This includes extending the time limit on current resources, and/or requesting additional resources be allocated to the requesting job.
Any additional allocated resources will be considered as part of the current allocation, and thus will be released at the same time.
%
\end{compactitem}


%%%%%%%%%%%
\subsection{\code{PMIx_Job_control_nb}}
\declareapi{PMIx_Job_control_nb}

%%%%
\summary

Request a job control action.

%%%%
\format

\cspecificstart
\begin{codepar}
pmix_status_t
PMIx_Job_control_nb(const pmix_proc_t targets[], size_t ntargets,
                    const pmix_info_t directives[], size_t ndirs,
                    pmix_info_cbfunc_t cbfunc, void *cbdata)
\end{codepar}
\cspecificend

\begin{arglist}
\argin{targets}{Array of proc structures (array of handles)}
\argin{ntargets}{Number of element in the \refarg{targets} array (integer)}
\argin{directives}{Array of info structures (array of handles)}
\argin{ndirs}{Number of element in the \refarg{directives} array (integer)}
\argin{cbfunc}{Callback function \refapi{pmix_info_cbfunc_t} (function reference)}
\argin{cbdata}{Data to be passed to the callback function (memory reference)}
\end{arglist}

Returns \refconst{PMIX_SUCCESS} or a negative value corresponding to a PMIx error constant.

\priattr
The PMIx_Job_control_nb function in the \ac{PRI} does not parse or utilize the attributes directly. However, the attributes are passed to the host RM for processing.

\reqattr
\acp{RM} that implement support for the \refapi{PMIx_Job_control_nb} are required to support the following attributes:

\pasteAttributeItem{PMIX_JOB_CTRL_ID}
\pasteAttributeItem{PMIX_JOB_CTRL_PAUSE}
\pasteAttributeItem{PMIX_JOB_CTRL_RESUME}
\pasteAttributeItem{PMIX_JOB_CTRL_KILL}
\pasteAttributeItem{PMIX_JOB_CTRL_SIGNAL}
\pasteAttributeItem{PMIX_JOB_CTRL_TERMINATE}

\optattr
A complete implementation that supports the \refapi{PMIx_Job_control_nb} would include support for the following attributes:

\pasteAttributeItem{PMIX_JOB_CTRL_CANCEL}
\pasteAttributeItem{PMIX_JOB_CTRL_RESTART}
\pasteAttributeItem{PMIX_JOB_CTRL_CHECKPOINT}
\pasteAttributeItem{PMIX_JOB_CTRL_CHECKPOINT_EVENT}
\pasteAttributeItem{PMIX_JOB_CTRL_CHECKPOINT_SIGNAL}
\pasteAttributeItem{PMIX_JOB_CTRL_CHECKPOINT_TIMEOUT}
\pasteAttributeItem{PMIX_JOB_CTRL_CHECKPOINT_METHOD}
\pasteAttributeItem{PMIX_JOB_CTRL_PROVISION}
\pasteAttributeItem{PMIX_JOB_CTRL_PROVISION_IMAGE}
\pasteAttributeItem{PMIX_JOB_CTRL_PREEMPTIBLE}


%%%%
\descr

Request a job control action.
The \refarg{targets} array identifies the processes to which the requested job control action is to be applied.
A \code{NULL} value can be used to indicate all processes in the caller's namespace.
The use of \refconst{PMIX_RANK_WILDARD} can also be used to indicate that all processes in the given namespace are to be included.

The directives are provided as \refstruct{pmix_info_t} structures in the \refarg{directives} array.
The callback function provides a \refarg{status} to indicate whether or not the request was granted, and to provide some information as to the reason for any denial in the \refapi{pmix_info_cbfunc_t} array of \refstruct{pmix_info_t} structures.
If non-\code{NULL}, then the specified \refarg{release_fn} must be called when the callback function completes - this will be used to release any provided \refstruct{pmix_info_t} array.


%%%%%%%%%%%%%%%%%%%%%%%%%%%%%%%%%%%%%%%%%%%%%%
%%%%%%%%%%%%%%%%%%%%%%%%%%%%%%%%%%%%%%%%%%%%%%
\section{Process and Job Monitoring}
\label{chap:api_job_mgmt:monitor}

In addition to external faults, a common problem encountered in \ac{HPC} applications is a failure to make
progress due to some internal conflict in the computation. These situations can
result in a significant waste of resources as the \ac{SMS} is unaware of the problem, and thus cannot terminate the
job. Various watchdog methods have been developed for detecting this situation, including requiring a periodic ``heartbeat''
from the application and monitoring a specified file for changes in size and/or modification time.

At the request of \ac{SMS} vendors and members, a monitoring support interface has been included in the PMIx v2 standard. The defined \ac{API} allows applications to request monitoring, directing what is to be monitored, the frequency of the associated check, whether or not the application is to be notified (via the event notification subsystem) of stall detection, and other characteristics of the operation. In addition, heartbeat and file monitoring methods have been included in the \ac{PRI} but are active only when requested.

%%%%%%%%%%%
\subsection{\code{PMIx_Process_monitor_nb}}
\declareapi{PMIx_Process_monitor_nb}

%%%%
\summary

Request that application processes be monitored.

%%%%
\format

\cspecificstart
\begin{codepar}
pmix_status_t
PMIx_Process_monitor_nb(const pmix_info_t *monitor, pmix_status_t error,
                        const pmix_info_t directives[], size_t ndirs,
                        pmix_info_cbfunc_t cbfunc, void *cbdata)
\end{codepar}
\cspecificend

\begin{arglist}
\argin{monitor}{info (handle)}
\argin{error}{status (integer)}
\argin{directives}{Array of info structures (array of handles)}
\argin{ndirs}{Number of elements in the \refarg{directives} array (integer)}
\argin{cbfunc}{Callback function \refapi{pmix_info_cbfunc_t} (function reference)}
\argin{cbdata}{Data to be passed to the callback function (memory reference)}
\end{arglist}

Returns \refconst{PMIX_SUCCESS} or a negative value corresponding to a PMIx error constant.

\priattr
\ac{PMIx} provides several monitoring methods in the \ac{PRI}. All other requests are passed to the host \ac{SMS}. Internally supported attributes include:

\pasteAttributeItem{PMIX_MONITOR_ID}
\pasteAttributeItem{PMIX_MONITOR_CANCEL}
\pasteAttributeItem{PMIX_MONITOR_APP_CONTROL}
\pasteAttributeItem{PMIX_MONITOR_HEARTBEAT}
\pasteAttributeItem{PMIX_MONITOR_HEARTBEAT_TIME}
\pasteAttributeItem{PMIX_MONITOR_HEARTBEAT_DROPS}
\pasteAttributeItem{PMIX_MONITOR_FILE}
\pasteAttributeItem{PMIX_MONITOR_FILE_SIZE}
\pasteAttributeItem{PMIX_MONITOR_FILE_ACCESS}
\pasteAttributeItem{PMIX_MONITOR_FILE_MODIFY}
\pasteAttributeItem{PMIX_MONITOR_FILE_CHECK_TIME}
\pasteAttributeItem{PMIX_MONITOR_FILE_DROPS}


%%%%
\descr

Request that application processes be monitored via several possible methods.
For example, that the server monitor this process for periodic heartbeats as an indication that the process has not become ``wedged''.
When a monitor detects the specified alarm condition, it will generate an event notification using the provided error code and passing along any available relevant information.
It is up to the caller to register a corresponding event handler.

The \refarg{monitor} argument is an attribute indicating the type of monitor being requested.
For example, \refattr{PMIX_MONITOR_FILE} to indicate that the requestor is asking that a file be monitored.

The \refarg{error} argument is the status code to be used when generating an event notification alerting that the monitor has been triggered.
The range of the notification defaults to \refconst{PMIX_RANGE_NAMESPACE}.
This can be changed by providing a \refconst{PMIX_RANGE} directive.

The \refarg{directives} argument characterizes the monitoring request (e.g., monitor file size) and frequency of checking to be done

The \refarg{cbfunc} function provides a \refarg{status} to indicate whether or not the request was granted, and to provide some information as to the reason for any denial in the \refapi{pmix_info_cbfunc_t} array of \refstruct{pmix_info_t} structures.


%%%%%%%%%%%
\subsection{\code{PMIx_Heartbeat}}
\declareapi{PMIx_Heartbeat}

%%%%
\summary

Send a heartbeat to the \ac{RM}

%%%%
\format

\cspecificstart
\begin{codepar}
void PMIx_Heartbeat(void)
\end{codepar}
\cspecificend


%%%%
\descr

A simplified version of \refapi{PMIx_Process_monitor_nb} that sends a heartbeat to the \ac{RM}.


%%%%%%%%%%%%%%%%%%%%%%%%%%%%%%%%%%%%%%%%%%%%%%
%%%%%%%%%%%%%%%%%%%%%%%%%%%%%%%%%%%%%%%%%%%%%%
\section{Logging}
\label{chap:api_job_mgmt:logging}

This section defines functionality to log messages to the system log, however it is implemented on the system.

\subsection{\code{PMIx_Log_nb}}
\declareapi{PMIx_Log_nb}

%%%%
\summary

Log data to a data service.

%%%%
\format

\cspecificstart
\begin{codepar}
pmix_status_t
PMIx_Log_nb(const pmix_info_t data[], size_t ndata,
            const pmix_info_t directives[], size_t ndirs,
            pmix_op_cbfunc_t cbfunc, void *cbdata)
\end{codepar}
\cspecificend

\begin{arglist}
\argin{data}{Array of info structures (array of handles)}
\argin{ndata}{Number of elements in the \refarg{data} array (\code{size_t})}
\argin{directives}{Array of info structures (array of handles)}
\argin{ndirs}{Number of elements in the \refarg{directives} array (\code{size_t})}
\argin{cbfunc}{Callback function \refapi{pmix_op_cbfunc_t} (function reference)}
\argin{cbdata}{Data to be passed to the callback function (memory reference)}
\end{arglist}

\begin{constantdesc}
\item \refconst{PMIX_SUCCESS} The logging request is valid and is being processed. The resulting status from the operation will be provided in the callback function.
\item \refconst{PMIX_ERR_BAD_PARAM} The logging request contains at least one incorrect entry that prevents it from being processed. The callback function will \emph{not} be called.
\item \refconst{PMIX_ERR_NOT_SUPPORTED} The \ac{PMIx} implementation does not support this function. The callback function will \emph{not} be called.
\end{constantdesc}

\priattr
The PMIx\_Log\_nb function in the \ac{PRI} does not parse or utilize the attributes directly. However, the attributes are passed to the host RM for processing.

\reqattr
\acp{RM} that implement support for the \refapi{PMIx_Log_nb} are required to support the following attributes:

\pasteAttributeItem{PMIX_LOG_STDERR}
\pasteAttributeItem{PMIX_LOG_STDOUT}
\pasteAttributeItem{PMIX_LOG_SYSLOG}

\optattr
A complete implementation that supports the \refapi{PMIx_Log_nb} would include support for the following attributes:

\pasteAttributeItem{PMIX_LOG_MSG}
\pasteAttributeItem{PMIX_LOG_EMAIL}
\pasteAttributeItem{PMIX_LOG_EMAIL_ADDR}
\pasteAttributeItem{PMIX_LOG_EMAIL_SUBJECT}
\pasteAttributeItem{PMIX_LOG_EMAIL_MSG}

%%%%
\descr

Log data subject to the services offered by the host environment. The data to be logged is provided in the \refarg{data} array. The (optional) \refarg{directives} can be used to direct the choice of logging channel.
The callback function will be executed when the log operation has been completed. The \refarg{data} and \refarg{directives} arrays must be maintained until the callback is provided.

\adviceuserstart
The \refapi{PMIx_Log_nb} API should \emph{never} be used for streaming data as it is not a ``performant'' transport and can perturb the application since it involves the local \ac{PMIx} server and host \ac{SMS} daemon.
\adviceuserend

%%%%%%%%%%%%%%%%%%%%%%%%%%%%%%%%%%%%%%%%%%%%%%%%%
