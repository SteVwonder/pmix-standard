%%%%%%%%%%%%%%%%%%%%%%%%%%%%%%%%%%%%%%%%%%%%%%%%%
% Appendix: Python bindings
%%%%%%%%%%%%%%%%%%%%%%%%%%%%%%%%%%%%%%%%%%%%%%%%%
\chapter{Python Bindings}
\label{app:python}

While the \ac{PMIx} Standard is defined in terms of C-based \acp{API}, there is no intent to limit the use of \ac{PMIx} to that specific language. Support for other languages is captured in the Standard by describing their equivalent syntax for the \ac{PMIx} \acp{API} and native forms for the \ac{PMIx} datatypes. This Appendix specifically deals with Python interfaces, beginning with a review of the \ac{PMIx} datatypes.

Note: the \ac{PMIx} \acp{API} have been loosely collected into three Python classes based on their \ac{PMIx} “class” (i.e., client, server, and tool). All processes have access to a basic set of the \acp{API}, and therefore those have been included in the “client” class. Servers can utilize any of those functions plus a set focused on operations not commonly executed by an application process. Finally, tools can also act as servers but have their own initialization function.


%%%%%%%%%%%%%%%%%%%%%%%%%%%%%%%%%%%%%%%%%%%%
\section{Datatype Definitions}

\ac{PMIx} defines a number of datatypes comprised of fixed-size character arrays, restricted range integers (e.g., uint32_t), and structures. Each datatype is represented by a named unsigned 16-bit integer (\code{uint16_t}) constant. Users are advised to use the named \ac{PMIx} constants for indicating datatypes instead of integer values to ensure compatibility with future PMIx versions.

With only a few exceptions, the C-based \ac{PMIx} datatypes defined in \chapterref{chap:struct} directly translate to Python. However, Python lacks the size-specific value definitions of C (e.g., \code{uint8_t}) and thus some care must be taken to protect against overflow/underflow situations when moving between the languages. Python bindings that accept values including \ac{PMIx} datatypes shall therefore have the datatype and associated value checked for compatibility with their \ac{PMIx}-defined equivalents, returning an error if:

\begin{itemize}
    \item datatypes not defined by \ac{PMIx} are encountered
    \item provided values fall outside the range of the C-equivalent definition - e.g., if a value identified as \refconst{PMIX_UINT8} lies outside the \code{uint8_t}range
\end{itemize}

Note that explicit labeling of \ac{PMIx} datatype, even when Python itself doesn’t care, is often required for the Python bindings to know how to properly interpret and label the provided value when passing it to the \ac{PMIx} library.

Table~\ref{app:python:ctopy} lists the correspondence between datatypes in the two languages.

\begin{landscape}
\begin{small}
    \begin{longtable}{ | p{4.5cm} | p{4cm} | p{3cm} | p{5.5cm} |}
        \caption{C-to-Python Datatype Correspondence} \label{app:python:ctopy} \\
        \hline
        C-Definition & PMIx Name & Python Definition & Notes \\ \hline
        \endhead
        \code{bool} & PMIX_BOOL & boolean & \\ \hline
        \code{byte} & PMIX_BYTE & A single element byte array (i.e., a byte array of length one) & \\ \hline
        \code{char*} & PMIX_STRING & string & \\ \hline
        \code{size_t} & PMIX_SIZE & integer & \\ \hline
        \code{pid_t} & PMIX_PID & integer & value shall be limited to the \code{uint32_t} range \\ \hline
        \code{int, int8_t, int16_t, int32_t, int64_t} & PMIX_INT, PMIX_INT8, PMIX_INT16, PMIX_INT32, PMIX_INT64 & integer & value shall be limited to its corresponding range \\ \hline
        \code{uint, uint8_t, uint16_t, uint32_t} & PMIX_UINT, PMIX_UINT8, PMIX_UINT16, PMIX_UINT32, PMIX_UINT64 & integer & value shall be limited to its corresponding range \\ \hline
        \code{float, double} & PMIX_FLOAT, PMIX_DOUBLE & float & value shall be limited to its corresponding range \\ \hline
        \code{struct timeval} & PMIX_TIMEVAL & \{'sec': sec, 'usec': microsec\} & each field is an integer value \\ \hline
        \code{time_t} & PMIX_TIME & integer & limited to positive values \\ \hline
        \refstruct{pmix_data_type_t} & PMIX_DATA_TYPE & integer & value shall be limited to the \code{uint16_t} range \\ \hline
        \refstruct{pmix_status_t} & PMIX_STATUS & integer & \\ \hline
        \refstruct{pmix_key_t} & N/A & \pylabel{key}string & The string's length shall be limited to one less than the size of the \refstruct{pmix_key_t} array (to reserve space for the terminating \code{NULL})  \\ \hline
        \refstruct{pmix_nspace_t} & N/A & \pylabel{nspace}string & The string's length shall be limited to one less than the size of the \refstruct{pmix_nspace_t} array (to reserve space for the terminating \code{NULL})  \\ \hline
        \refstruct{pmix_rank_t} & PMIX_PROC_RANK & \pylabel{rank}integer & value shall be limited to the \code{uint32_t} range excepting the reserved values near \code{UINT32_MAX} \\ \hline
        \refstruct{pmix_proc_t} & PMIX_PROC & \pylabel{proc}\{'nspace': nspace, 'rank': rank\} & \refarg{nspace} is a Python string and \refarg{rank} is an integer value. The \refarg{nspace} string's length shall be limited to one less than the size of the \refstruct{pmix_nspace_t} array (to reserve space for the terminating \code{NULL}), and the \refarg{rank} value shall conform to the constraints associated with \refstruct{pmix_rank_t} \\ \hline
        \refstruct{pmix_byte_object_t} & PMIX_BYTE_OBJECT & \pylabel{byteobject}\{'bytes': bytes, 'size': size\} & \refarg{bytes} is a Python byte array and \refarg{size} is the integer number of bytes in that array. \\ \hline
        \refstruct{pmix_persistence_t} & PMIX_PERSISTENCE & integer & value shall be limited to the \code{uint8_t} range \\ \hline
        \refstruct{pmix_scope_t} & PMIX_SCOPE & integer & value shall be limited to the \code{uint8_t} range \\ \hline
        \refstruct{pmix_data_range_t} & PMIX_RANGE & \pylabel{range}integer & value shall be limited to the \code{uint8_t} range \\ \hline
        \refstruct{pmix_proc_state_t} & PMIX_PROC_STATE & integer & value shall be limited to the \code{uint8_t} range \\ \hline
        \refstruct{pmix_proc_info_t} & PMIX_PROC_INFO & \{'proc': \{'nspace': nspace, 'rank': rank\}, 'hostname': hostname, 'executable': executable, 'pid': pid, 'exitcode': exitcode, 'state': state\} & \refarg{proc} is a Python \refpy{proc} dictionary; \refarg{hostname} is a Python string; and \refarg{pid}, \refarg{exitcode}, and \refarg{state} are Python integers \\ \hline
        \refstruct{pmix_data_array_t} & PMIX_DATA_ARRAY & \pylabel{array}\{'type': type, 'array': array\} & \refarg{type} is the \ac{PMIx} type of object in the array and \refarg{array} is a Python \emph{list} containing the individual array elements. Note that \refarg{array} can consist of \emph{any} \ac{PMIx} types, including (for example) a Python \refpy{info} object that itself contains an \refpy{array} value \\ \hline
        \refstruct{pmix_info_directives_t}  & PMIX_INFO_DIRECTIVES & integer & value shall be limited to the \code{uint32_t} range \\ \hline
        \refstruct{pmix_alloc_directive_t} & PMIX_ALLOC_DIRECTIVE & \pylabel{allocdir}integer & value shall be limited to the \code{uint8_t} range \\ \hline
        \refstruct{pmix_iof_channel_t} & PMIX_IOF_CHANNEL & \pylabel{channel}integer & value shall be limited to the \code{uint16_t} range \\ \hline
        \refstruct{pmix_envar_t} & PMIX_ENVAR & \{'envar': envar, 'value': value, 'separator': separator\} & \refarg{envar} and \refarg{value} are Python strings, and \refarg{separator} a single-character Python string \\ \hline
        \refstruct{pmix_value_t} & PMIX_VALUE & \pylabel{value}\{'value': value, 'val_type': type\} & \refarg{type} is the \ac{PMIx} datatype of \refarg{value}, and \refarg{value} is the associated value expressed in the appropriate Python form for the specified datatype  \\ \hline
        \refstruct{pmix_info_t} & PMIX_INFO & \pylabel{info}\{'key': key, 'value': value, 'val_type': type\} & \refarg{key} is a Python string \refpy{key}, \refarg{type} is the \ac{PMIx} datatype of \refarg{value}, and \refarg{value} is the associated value expressed in the appropriate Python form for the specified datatype \\ \hline
        \refstruct{pmix_pdata_t} & PMIX_PDATA & \pylabel{pdata}\{'proc': \{'nspace': nspace, 'rank': rank\}, 'key': key, 'value': value, 'val_type': type\} & \refarg{proc} is a Python \refpy{proc} dictionary; \refarg{key} is a Python string \refpy{key}; \refarg{type} is the \ac{PMIx} datatype of \refarg{value}; and \refarg{value} is the associated value expressed in the appropriate Python form for the specified datatype  \\ \hline
        \refstruct{pmix_app_t} & PMIX_APP & \pylabel{app}\{'cmd': cmd, 'argv': [argv], 'env': [env], 'maxprocs': maxprocs, 'info': [info]\} & \refarg{cmd} is a Python string; \refarg{argv} and \refarg{env} are Python \emph{lists} containing Python strings; \refarg{maxprocs} is an integer; and \refarg{info} is a Python \emph{list} of \refpy{info} values   \\ \hline
        \refstruct{pmix_query_t} & PMIX_QUERY & \pylabel{query}\{'keys': [keys], 'qualifiers': [info]\} & \refarg{keys} is a Python \emph{list} of Python strings, and \refarg{qualifiers} is a Python \emph{list} of \refpy{info} values \\ \hline
        \refstruct{pmix_regattr_t} & PMIX_REGATTR & \pylabel{regattr}\{'name': name, 'key': key, 'type': type, info': [info], 'description': [desc]\} & \refarg{name} and \refarg{string} are Python strings; \refarg{type} is the \ac{PMIx} datatype for the attribute's value; \refarg{info} is a Python \emph{list} of \refpy{info} values; and \refarg{description} is a list of Python strings describing the attribute  \\ \hline
        \hline
    \end{longtable}
\end{small}
\end{landscape}

%%%%%%%%%%%%%%%%%%%%%%%%%%%%%%%%%%%%%%%%%%%%
\section{Function Definitions}
\label{app:python:fns}

%%%%%%%%%%%
\subsection{IOF Delivery Function}
\pylabel{iofcbfunc}

%%%%
\summary

Callback function for delivering forwarded \ac{IO} to a process

%%%%
\format

\versionMarker{4.0}
\pyspecificstart
\begin{codepar}
def iofcbfunc(iofhdlr:integer, channel:integer,
              source:dict, payload:dict, info:list)
\end{codepar}
\pyspecificend

\begin{arglist}
\argin{iofhdlr}{Registration number of the handler being invoked (integer)}
\argin{channel}{Python \refpy{channel} bitmask identifying the channel the data arrived on (integer)}
\argin{source}{Python \refpy{proc} identifying the namespace/rank of the process that generated the data (dict)}
\argin{payload}{Python \refpy{byteobject} containing the data (dict)}
\argin{info}{List of Python \refpy{info} provided by the source containing metadata about the payload. This could include \refattr{PMIX_IOF_COMPLETE} (list)}
\end{arglist}

Returns: nothing

See \refapi{pmix_iof_cbfunc_t} for details


%%%%%%%%%%%
\subsection{Event Handler}
\pylabel{evhandler}

%%%%
\summary

Callback function for event handlers

%%%%
\format

\versionMarker{4.0}
\pyspecificstart
\begin{codepar}
def evhandler(evhdlr:integer, status:integer,
              source:dict, info:list, results:list)
\end{codepar}
\pyspecificend

\begin{arglist}
\argin{iofhdlr}{Registration number of the handler being invoked (integer)}
\argin{status}{Status associated with the operation (integer)}
\argin{source}{Python \refpy{proc} identifying the namespace/rank of the process that generated the event (dict)}
\argin{info}{List of Python \refpy{info} provided by the source containing metadata about the event (list)}
\argin{results}{List of Python \refpy{info} containing the aggregated results of all prior evhandlers (list)}
\end{arglist}

Returns:
\begin{itemize}
    \item \refarg{rc} - Status returned by the event handler's operation (integer)
    \item \refarg{results} - List of Python \refpy{info} containing results from this event handler's operation on the event (list)
\end{itemize}

See \refapi{pmix_notification_fn_t} for details


%%%%%%%%%%%
\subsection{Server Module Functions}
\pylabel{server module}

The following definitions represent functions that may be provided to the \ac{PMIx} server library at time of initialization for servicing of client requests. Module functions that are not provided default to returning "not supported" to the caller.

%%%%%%%%%%%
\subsubsection{Client Connected}

%%%%
\summary

Notify the host server that a client connected to this server.

%%%%
\format

\versionMarker{4.0}
\pyspecificstart
\begin{codepar}
def clientconnected(proc:dict is not None)
\end{codepar}
\pyspecificend

\begin{arglist}
\argin{proc}{Python \refpy{proc} identifying the namespace/rank of the process that connected (dict)}
\end{arglist}

Returns:
\begin{itemize}
    \item \refarg{rc} - \refconst{PMIX_SUCCESS} or a \ac{PMIx} error code indicating the connection should be rejected (integer)
\end{itemize}

See \refapi{pmix_server_client_connected_fn_t} for details

%%%%%%%%%%%
\subsubsection{Client Finalized}

%%%%
\summary

Notify the host environment that a client called \refapi{PMIx_Finalize}.

%%%%
\format

\versionMarker{4.0}
\pyspecificstart
\begin{codepar}
def clientfinalized(proc:dict is not None):
\end{codepar}
\pyspecificend

\begin{arglist}
\argin{proc}{Python \refpy{proc} identifying the namespace/rank of the process that finalized (dict)}
\end{arglist}

Returns: nothing

See \refapi{pmix_server_client_finalized_fn_t} for details


%%%%%%%%%%%
\subsubsection{Client Aborted}

%%%%
\summary

Notify the host environment that a local client called \refapi{PMIx_Abort}.

%%%%
\format

\versionMarker{4.0}
\pyspecificstart
\begin{codepar}
def clientaborted(proc:dict is not None, status:integer,
                  msg:str, targets:list)
\end{codepar}
\pyspecificend

\begin{arglist}
\argin{proc}{Python \refpy{proc} identifying the namespace/rank of the process that called abort (dict)}
\argin{status}{PMIx status to be returned on exit (integer)}
\argin{msg}{String message to be printed (string)}
\argin{targets}{List of Python \refpy{proc} dictionaries (list)}
\end{arglist}

Returns:
\begin{itemize}
    \item \refarg{rc} - \refconst{PMIX_SUCCESS} or a \ac{PMIx} error code indicating the operation failed (integer)
\end{itemize}

See \refapi{pmix_server_abort_fn_t} for details


%%%%%%%%%%%
\subsubsection{Fence}

%%%%
\summary

At least one client called either \refapi{PMIx_Fence} or \refapi{PMIx_Fence_nb}

%%%%
\format

\versionMarker{4.0}
\pyspecificstart
\begin{codepar}
def fence(procs:list, directives:list, data:bytearray)
\end{codepar}
\pyspecificend

\begin{arglist}
\argin{procs}{List of Python \refpy{proc} dictionaries (list)}
\argin{directives}{List of Python \refpy{info} dictionaries (list)}
\argin{data}{Python bytearray of data to be circulated during fence operation (bytearray)}
\end{arglist}

Returns:
\begin{itemize}
    \item \refarg{rc} - \refconst{PMIX_SUCCESS} or a \ac{PMIx} error code indicating the operation failed (integer)
    \item \refarg{data} - Python bytearray containing the aggregated data from all participants (bytearray)
\end{itemize}

See \refapi{pmix_server_fencenb_fn_t} for details

%%%%%%%%%%%
\subsubsection{Direct Modex}

%%%%
\summary

Used by the PMIx server to request its local host contact the \ac{PMIx} server on the remote node that hosts the specified proc to obtain and return a direct modex blob for that proc.

%%%%
\format

\versionMarker{4.0}
\pyspecificstart
\begin{codepar}
def dmodex(proc:dict, directives:list)
\end{codepar}
\pyspecificend

\begin{arglist}
\argin{proc}{Python \refpy{proc} dictionary of process whose data is being requested (list)}
\argin{directives}{List of Python \refpy{info} dictionaries (list)}
\end{arglist}

Returns:
\begin{itemize}
    \item \refarg{rc} - \refconst{PMIX_SUCCESS} or a \ac{PMIx} error code indicating the operation failed (integer)
    \item \refarg{data} - Python bytearray containing the data for the specified process (bytearray)
\end{itemize}

See \refapi{pmix_server_dmodex_req_fn_t} for details

%%%%%%%%%%%
\subsubsection{Publish}

%%%%
\summary

Publish data per the PMIx API specification.

%%%%
\format

\versionMarker{4.0}
\pyspecificstart
\begin{codepar}
def publish(proc:dict, directives:list)
\end{codepar}
\pyspecificend

\begin{arglist}
\argin{proc}{Python \refpy{proc} dictionary of process publishing the data (list)}
\argin{directives}{List of Python \refpy{info} dictionaries containing data and directives (list)}
\end{arglist}

Returns:
\begin{itemize}
    \item \refarg{rc} - \refconst{PMIX_SUCCESS} or a \ac{PMIx} error code indicating the operation failed (integer)
\end{itemize}

See \refapi{pmix_server_publish_fn_t} for details


%%%%%%%%%%%
\subsubsection{Lookup}

%%%%
\summary

Lookup published data.

%%%%
\format

\versionMarker{4.0}
\pyspecificstart
\begin{codepar}
def lookup(proc:dict, keys:list, directives:list)
\end{codepar}
\pyspecificend

\begin{arglist}
\argin{proc}{Python \refpy{proc} dictionary of process seeking the data (list)}
\argin{keys}{List of Python strings (list)}
\argin{directives}{List of Python \refpy{info} dictionaries containing directives (list)}
\end{arglist}

Returns:
\begin{itemize}
    \item \refarg{rc} - \refconst{PMIX_SUCCESS} or a \ac{PMIx} error code indicating the operation failed (integer)
    \item \refarg{pdata} - List of \refpy{pdata} containing the returned results (list)
\end{itemize}

See \refapi{pmix_server_lookup_fn_t} for details


%%%%%%%%%%%
\subsubsection{Unpublish}

%%%%
\summary

Delete data from the data store.

%%%%
\format

\versionMarker{4.0}
\pyspecificstart
\begin{codepar}
def unpublish(proc:dict, keys:list, directives:list)
\end{codepar}
\pyspecificend

\begin{arglist}
\argin{proc}{Python \refpy{proc} dictionary of process making the request (list)}
\argin{keys}{List of Python strings (list)}
\argin{directives}{List of Python \refpy{info} dictionaries containing directives (list)}
\end{arglist}

Returns:
\begin{itemize}
    \item \refarg{rc} - \refconst{PMIX_SUCCESS} or a \ac{PMIx} error code indicating the operation failed (integer)
\end{itemize}

See \refapi{pmix_server_unpublish_fn_t} for details


%%%%%%%%%%%
\subsubsection{Spawn}

%%%%
\summary

Spawn a set of applications/processes as per the \refapi{PMIx_Spawn} API.

%%%%
\format

\versionMarker{4.0}
\pyspecificstart
\begin{codepar}
def spawn(proc:dict, jobInfo:list, apps:list)
\end{codepar}
\pyspecificend

\begin{arglist}
\argin{proc}{Python \refpy{proc} dictionary of process making the request (list)}
\argin{jobInfo}{List of Python \refpy{info} job-level directives and information (list)}
\argin{apps}{List of Python \refpy{app} dictionaries describing applications to be spawned (list)}
\end{arglist}

Returns:
\begin{itemize}
    \item \refarg{rc} - \refconst{PMIX_SUCCESS} or a \ac{PMIx} error code indicating the operation failed (integer)
    \item \refarg{nspace} - Python string containing namespace of the spawned job (str)
\end{itemize}

See \refapi{pmix_server_spawn_fn_t} for details


%%%%%%%%%%%
\subsubsection{Connect}

%%%%
\summary

Record the specified processes as \textit{connected}.

%%%%
\format

\versionMarker{4.0}
\pyspecificstart
\begin{codepar}
def connect(procs:list, directives:list)
\end{codepar}
\pyspecificend

\begin{arglist}
\argin{procs}{List of Python \refpy{proc} dictionaries identifying participants (list)}
\argin{directives}{List of Python \refpy{info} directives (list)}
\end{arglist}

Returns:
\begin{itemize}
    \item \refarg{rc} - \refconst{PMIX_SUCCESS} or a \ac{PMIx} error code indicating the operation failed (integer)
\end{itemize}

See \refapi{pmix_server_connect_fn_t} for details


%%%%%%%%%%%
\subsubsection{Disconnect}

%%%%
\summary

Disconnect a previously connected set of processes.

%%%%
\format

\versionMarker{4.0}
\pyspecificstart
\begin{codepar}
def disconnect(procs:list, directives:list)
\end{codepar}
\pyspecificend

\begin{arglist}
\argin{procs}{List of Python \refpy{proc} dictionaries identifying participants (list)}
\argin{directives}{List of Python \refpy{info} directives (list)}
\end{arglist}

Returns:
\begin{itemize}
    \item \refarg{rc} - \refconst{PMIX_SUCCESS} or a \ac{PMIx} error code indicating the operation failed (integer)
\end{itemize}

See \refapi{pmix_server_disconnect_fn_t} for details


%%%%%%%%%%%
\subsubsection{Register Events}

%%%%
\summary

Register to receive notifications for the specified events.

%%%%
\format

\versionMarker{4.0}
\pyspecificstart
\begin{codepar}
def register_events(codes:list, directives:list)
\end{codepar}
\pyspecificend

\begin{arglist}
\argin{codes}{List of Python integers (list)}
\argin{directives}{List of Python \refpy{info} directives (list)}
\end{arglist}

Returns:
\begin{itemize}
    \item \refarg{rc} - \refconst{PMIX_SUCCESS} or a \ac{PMIx} error code indicating the operation failed (integer)
\end{itemize}

See \refapi{pmix_server_register_events_fn_t} for details


%%%%%%%%%%%
\subsubsection{Deregister Events}

%%%%
\summary

Deregister to receive notifications for the specified events.

%%%%
\format

\versionMarker{4.0}
\pyspecificstart
\begin{codepar}
def deregister_events(codes:list)
\end{codepar}
\pyspecificend

\begin{arglist}
\argin{codes}{List of Python integers (list)}
\end{arglist}

Returns:
\begin{itemize}
    \item \refarg{rc} - \refconst{PMIX_SUCCESS} or a \ac{PMIx} error code indicating the operation failed (integer)
\end{itemize}

See \refapi{pmix_server_deregister_events_fn_t} for details


%%%%%%%%%%%
\subsubsection{Notify Event}

%%%%
\summary

Notify the specified range of processes of an event.

%%%%
\format

\versionMarker{4.0}
\pyspecificstart
\begin{codepar}
def notify_event(code:integer, source:dict, range:integer, directives:list)
\end{codepar}
\pyspecificend

\begin{arglist}
\argin{code}{Python integer \refstruct{pmix_status_t}  (list)}
\argin{source}{Python \refpy{proc} of process that generated the event (dict)}
\argin{range}{Python \refpy{range} in which the event is to be reported (integer)}
\argin{directives}{List of Python \refpy{info} directives (list)}
\end{arglist}

Returns:
\begin{itemize}
    \item \refarg{rc} - \refconst{PMIX_SUCCESS} or a \ac{PMIx} error code indicating the operation failed (integer)
\end{itemize}

See \refapi{pmix_server_notify_event_fn_t} for details


%%%%%%%%%%%
\subsubsection{Query}

%%%%
\summary

Query information from the resource manager.

%%%%
\format

\versionMarker{4.0}
\pyspecificstart
\begin{codepar}
def query(proc:dict, queries:list)
\end{codepar}
\pyspecificend

\begin{arglist}
\argin{proc}{Python \refpy{proc} of requesting process (dict)}
\argin{queries}{List of Python \refpy{query} directives (list)}
\end{arglist}

Returns:
\begin{itemize}
    \item \refarg{rc} - \refconst{PMIX_SUCCESS} or a \ac{PMIx} error code indicating the operation failed (integer)
    \item \refarg{info} - List of Python \refpy{info} containing the returned results (list)
\end{itemize}

See \refapi{pmix_server_query_fn_t} for details


%%%%%%%%%%%
\subsubsection{Tool Connected}

%%%%
\summary

Register that a tool has connected to the server.

%%%%
\format

\versionMarker{4.0}
\pyspecificstart
\begin{codepar}
def tool_connected(info:list)
\end{codepar}
\pyspecificend

\begin{arglist}
\argin{info}{List of Python \refpy{info} containing info on the connecting tool (list)}
\end{arglist}

Returns:
\begin{itemize}
    \item \refarg{rc} - \refconst{PMIX_SUCCESS} or a \ac{PMIx} error code indicating the operation failed (integer)
    \item \refarg{proc} - Python \refpy{proc} containing the assigned namespace:rank for the tool (dict)
\end{itemize}

See \refapi{pmix_server_tool_connection_fn_t} for details


%%%%%%%%%%%
\subsubsection{Log}

%%%%
\summary

Log data on behalf of a client.

%%%%
\format

\versionMarker{4.0}
\pyspecificstart
\begin{codepar}
def log(proc:dict, data:list, directives:list)
\end{codepar}
\pyspecificend

\begin{arglist}
\argin{proc}{Python \refpy{proc} of requesting process (dict)}
\argin{data}{List of Python \refpy{info} containing data to be logged (list)}
\argin{directives}{List of Python \refpy{info} containing directives (list)}
\end{arglist}

Returns:
\begin{itemize}
    \item \refarg{rc} - \refconst{PMIX_SUCCESS} or a \ac{PMIx} error code indicating the operation failed (integer)
\end{itemize}

See \refapi{pmix_server_log_fn_t} for details


%%%%%%%%%%%
\subsubsection{Allocate Resources}

%%%%
\summary

Request allocation operations on behalf of a client.

%%%%
\format

\versionMarker{4.0}
\pyspecificstart
\begin{codepar}
def allocate(proc:dict, action:integer, directives:list)
\end{codepar}
\pyspecificend

\begin{arglist}
\argin{proc}{Python \refpy{proc} of requesting process (dict)}
\argin{action}{Python \refpy{allocdir} specifying requested action (integer)}
\argin{directives}{List of Python \refpy{info} containing directives (list)}
\end{arglist}

Returns:
\begin{itemize}
    \item \refarg{rc} - \refconst{PMIX_SUCCESS} or a \ac{PMIx} error code indicating the operation failed (integer)
    \item refarg{info} - List of Python \refpy{info} containing results of requested operation (list)
\end{itemize}

See \refapi{pmix_server_alloc_fn_t} for details


%%%%%%%%%%%
\subsubsection{Job Control}

%%%%
\summary

Execute a job control action on behalf of a client.

%%%%
\format

\versionMarker{4.0}
\pyspecificstart
\begin{codepar}
def job_control(proc:dict, targets:list, directives:list)
\end{codepar}
\pyspecificend

\begin{arglist}
\argin{proc}{Python \refpy{proc} of requesting process (dict)}
\argin{targets}{List of Python \refpy{proc} specifying target processes (list)}
\argin{directives}{List of Python \refpy{info} containing directives (list)}
\end{arglist}

Returns:
\begin{itemize}
    \item \refarg{rc} - \refconst{PMIX_SUCCESS} or a \ac{PMIx} error code indicating the operation failed (integer)
\end{itemize}

See \refapi{pmix_server_job_control_fn_t} for details


%%%%%%%%%%%
\subsubsection{Monitor}

%%%%
\summary

Request that a client be monitored for activity.

%%%%
\format

\versionMarker{4.0}
\pyspecificstart
\begin{codepar}
def monitor(proc:dict, request:list, directives:list)
\end{codepar}
\pyspecificend

\begin{arglist}
\argin{proc}{Python \refpy{proc} of requesting process (dict)}
\argin{request}{List of Python \refpy{info} specifying requested monitoring operations (list)}
\argin{directives}{List of Python \refpy{info} containing directives (list)}
\end{arglist}

Returns:
\begin{itemize}
    \item \refarg{rc} - \refconst{PMIX_SUCCESS} or a \ac{PMIx} error code indicating the operation failed (integer)
\end{itemize}

See \refapi{pmix_server_monitor_fn_t} for details


%%%%%%%%%%%
\subsubsection{Get Credential}

%%%%
\summary

Request a credential from the host environment

%%%%
\format

\versionMarker{4.0}
\pyspecificstart
\begin{codepar}
def get_credential(proc:dict, directives:list)
\end{codepar}
\pyspecificend

\begin{arglist}
\argin{proc}{Python \refpy{proc} of requesting process (dict)}
\argin{directives}{List of Python \refpy{info} containing directives (list)}
\end{arglist}

Returns:
\begin{itemize}
    \item \refarg{rc} - \refconst{PMIX_SUCCESS} or a \ac{PMIx} error code indicating the operation failed (integer)
    \item \refarg{cred} - Python \refpy{byteobject} containing returned credential (dict)
    \item \refarg{info} - List of Python \refpy{info} containing any additional info about the credential (list)
\end{itemize}

See \refapi{pmix_server_get_cred_fn_t} for details


%%%%%%%%%%%
\subsubsection{Validate Credential}

%%%%
\summary

Request validation of a credential

%%%%
\format

\versionMarker{4.0}
\pyspecificstart
\begin{codepar}
def validate_credential(proc:dict, cred:dict, directives:list)
\end{codepar}
\pyspecificend

\begin{arglist}
\argin{proc}{Python \refpy{proc} of requesting process (dict)}
\argin{cred}{Python \refpy{byteobject} containing credential (dict)}
\argin{directives}{List of Python \refpy{info} containing directives (list)}
\end{arglist}

Returns:
\begin{itemize}
    \item \refarg{rc} - \refconst{PMIX_SUCCESS} or a \ac{PMIx} error code indicating the operation failed (integer)
    \item \refarg{info} - List of Python \refpy{info} containing any additional info from the credential (list)
\end{itemize}

See \refapi{pmix_server_validate_cred_fn_t} for details


%%%%%%%%%%%
\subsubsection{IO Forward}

%%%%
\summary

Request the specified IO channels be forwarded from the given array of processes.

%%%%
\format

\versionMarker{4.0}
\pyspecificstart
\begin{codepar}
def iof_pull(sources:list, channels:integer, directives:list)
\end{codepar}
\pyspecificend

\begin{arglist}
\argin{sources}{List of Python \refpy{proc} whose IO is being requested (list)}
\argin{channels}{Bitmask of Python \refpy{channel} identifying IO channels to be forwarded (integer)}
\argin{directives}{List of Python \refpy{info} containing directives (list)}
\end{arglist}

Returns:
\begin{itemize}
    \item \refarg{rc} - \refconst{PMIX_SUCCESS} or a \ac{PMIx} error code indicating the operation failed (integer)
\end{itemize}

See \refapi{pmix_server_iof_fn_t} for details


%%%%%%%%%%%
\subsubsection{IO Push}

%%%%
\summary

Pass standard input data to the host environment for transmission to specified recipients.

%%%%
\format

\versionMarker{4.0}
\pyspecificstart
\begin{codepar}
def iof_push(source:dict, targets:list, directives:list)
\end{codepar}
\pyspecificend

\begin{arglist}
\argin{source}{Python \refpy{proc} whose stdin data is being provided (dict)}
\argin{targets}{List of Python \refpy{proc} identifying targets to receive the provided data (list)}
\argin{directives}{List of Python \refpy{info} containing directives (list)}
\end{arglist}

Returns:
\begin{itemize}
    \item \refarg{rc} - \refconst{PMIX_SUCCESS} or a \ac{PMIx} error code indicating the operation failed (integer)
\end{itemize}

See \refapi{pmix_server_stdin_fn_t} for details


%%%%%%%%%%%%%%%%%%%%%%%%%%%%%%%%%%%%%%%%%%%%

\section{PMIxClient}
\label{app:python:client}

The client Python class is by far the richest in terms of \acp{API} as it houses all the \acp{API} that an application might utilize. Due to the datatype translation requirements of the C-Python interface, only the blocking form of each \ac{API} is supported – providing a Python callback function directly to the C interface underlying the bindings was not a supportable option.

%%%%%%%%%%%
\subsection{Client.init}
\declareapi{PMIxClient.init}

\summary Initialize the \ac{PMIx} client library after obtaining a new PMIxClient object

\format

\versionMarker{4.0}
\pyspecificstart
\begin{codepar}
rc, proc = myclient.init(info:list)
\end{codepar}
\pyspecificend


\begin{arglist}
\argin{info}{List of Python \refpy{info} dictionaries (list)}
\end{arglist}

Returns:

\begin{itemize}
    \item \refarg{rc} - \refconst{PMIX_SUCCESS} or a negative value corresponding to a PMIx error constant (integer)
    \item \refarg{proc} - a Python \refpy{proc} dictionary (dict)
\end{itemize}


See \refapi{PMIx_Init} for description of all relevant attributes and behaviors

%%%%%%%%%%%
\subsection{Client.initialized}
\declareapi{PMIxClient.initialized}

\format

\versionMarker{4.0}
\pyspecificstart
\begin{codepar}
rc = myclient.initialized()
\end{codepar}
\pyspecificend



Returns:

\begin{itemize}
    \item \refarg{rc} - a value of \code{1} (true) will be returned if the \ac{PMIx} library has been initialized, and \code{0} (false) otherwise (integer)

\end{itemize}


See \refapi{PMIx_Initialized} for description of all relevant attributes and behaviors

%%%%%%%%%%%
\subsection{Client.get_version}
\declareapi{PMIxClient.get_version}

\format

\versionMarker{4.0}
\pyspecificstart
\begin{codepar}
vers = myclient.get_version()
\end{codepar}
\pyspecificend



Returns:

\begin{itemize}
    \item \refarg{vers} - Python string containing the version of the \ac{PMIx} library (e.g., "3.1.4") (integer)

\end{itemize}


See \refapi{PMIx_Get_version} for description of all relevant attributes and behaviors

%%%%%%%%%%%
\subsection{Client.finalize}
\declareapi{PMIxClient.finalize}

%%%%
\summary

Finalize the PMIx client library.

%%%%
\format

\versionMarker{4.0}
\pyspecificstart
\begin{codepar}
rc = myclient.finalize(info:list)
\end{codepar}
\pyspecificend

\begin{arglist}
\argin{info}{List of Python \refpy{info} dictionaries (list)}
\end{arglist}

Returns:

\begin{itemize}
    \item \refarg{rc} - \refconst{PMIX_SUCCESS} or a negative value corresponding to a PMIx error constant (integer)
\end{itemize}


See \refapi{PMIx_Finalize} for description of all relevant attributes and behaviors


%%%%%%%%%%%
\subsection{Client.abort}
\declareapi{PMIxClient.abort}

%%%%
\summary

Request that the provided list of procs be aborted

%%%%
\format

\versionMarker{4.0}
\pyspecificstart
\begin{codepar}
rc = myclient.abort(status:integer, msg:str, targets:list)
\end{codepar}
\pyspecificend

\begin{arglist}
\argin{status}{PMIx status to be returned on exit (integer)}
\argin{msg}{String message to be printed (string)}
\argin{targets}{List of Python \refpy{proc} dictionaries (list)}
\end{arglist}

Returns:

\begin{itemize}
    \item \refarg{rc} - \refconst{PMIX_SUCCESS} or a negative value corresponding to a PMIx error constant (integer)
\end{itemize}


See \refapi{PMIx_Abort} for description of all relevant attributes and behaviors


%%%%%%%%%%%
\subsection{Client.store_internal}
\declareapi{PMIxClient.store_internal}

%%%%
\summary

Store some data locally for retrieval by other areas of the process

%%%%
\format

\versionMarker{4.0}
\pyspecificstart
\begin{codepar}
rc = myclient.store_internal(proc:dict, key:str, value:dict)
\end{codepar}
\pyspecificend

\begin{arglist}
\argin{proc}{Python \refpy{proc} dictionary of the process being referenced (dict)}
\argin{key}{String key of the data (string)}
\argin{value}{Python \refpy{value} dictionary (dict)}
\end{arglist}

Returns:

\begin{itemize}
    \item \refarg{rc} - \refconst{PMIX_SUCCESS} or a negative value corresponding to a PMIx error constant (integer)
\end{itemize}


See \refapi{PMIx_Store_internal} for details


%%%%%%%%%%%
\subsection{Client.put}
\declareapi{PMIxClient.put}

%%%%
\summary

Push a key/value pair into the client's namespace.

%%%%
\format

\versionMarker{4.0}
\pyspecificstart
\begin{codepar}
rc = myclient.put(scope:integer, key:str, value:dict)
\end{codepar}
\pyspecificend

\begin{arglist}
\argin{scope}{Scope of the data being posted (integer)}
\argin{key}{String key of the data (string)}
\argin{value}{Python \refpy{value} dictionary (dict)}
\end{arglist}

Returns:

\begin{itemize}
    \item \refarg{rc} - \refconst{PMIX_SUCCESS} or a negative value corresponding to a PMIx error constant (integer)
\end{itemize}


See \refapi{PMIx_Put} for description of all relevant attributes and behaviors


%%%%%%%%%%%
\subsection{Client.commit}
\declareapi{PMIxClient.commit}

%%%%
\summary

Push all previously \refapi{PMIxClient.put} values to the local PMIx server.

%%%%
\format

\versionMarker{4.0}
\pyspecificstart
\begin{codepar}
rc = myclient.commit()
\end{codepar}
\pyspecificend

Returns:

\begin{itemize}
    \item \refarg{rc} - \refconst{PMIX_SUCCESS} or a negative value corresponding to a PMIx error constant (integer)
\end{itemize}


See \refapi{PMIx_Commit} for description of all relevant attributes and behaviors


%%%%%%%%%%%
\subsection{Client.fence}
\declareapi{PMIxClient.fence}

%%%%
\summary

Execute a blocking barrier across the processes identified in the specified list

%%%%
\format

\versionMarker{4.0}
\pyspecificstart
\begin{codepar}
rc = myclient.fence(peers:list, directives:list)
\end{codepar}
\pyspecificend

\begin{arglist}
\argin{peers}{List of Python \refpy{proc} dictionaries (list)}
\argin{directives}{List of Python \refpy{info} dictionaries (list)}
\end{arglist}

Returns:

\begin{itemize}
    \item \refarg{rc} - \refconst{PMIX_SUCCESS} or a negative value corresponding to a PMIx error constant (integer)
\end{itemize}


See \refapi{PMIx_Fence} for description of all relevant attributes and behaviors


%%%%%%%%%%%
\subsection{Client.get}
\declareapi{PMIxClient.get}

%%%%
\summary

Retrieve a key/value pair

%%%%
\format

\versionMarker{4.0}
\pyspecificstart
\begin{codepar}
rc, val = myclient.get(proc:dict, key:str, directives:list)
\end{codepar}
\pyspecificend

\begin{arglist}
\argin{proc}{Python \refpy{proc} whose data is being requested (dict)}
\argin{key}{Python string key of the data to be returned (str)}
\argin{directives}{List of Python \refpy{info} dictionaries (list)}
\end{arglist}

Returns:

\begin{itemize}
    \item \refarg{rc} - \refconst{PMIX_SUCCESS} or a negative value corresponding to a PMIx error constant (integer)
    \item \refarg{val} - Python \refpy{value} containing the returned data (dict)
\end{itemize}


See \refapi{PMIx_Get} for description of all relevant attributes and behaviors


%%%%%%%%%%%
\subsection{Client.publish}
\declareapi{PMIxClient.publish}

%%%%
\summary

Publish data for later access via \refapi{PMIx_Lookup}.

%%%%
\format

\versionMarker{4.0}
\pyspecificstart
\begin{codepar}
rc = myclient.publish(directives:list)
\end{codepar}
\pyspecificend

\begin{arglist}
\argin{directives}{List of Python \refpy{info} dictionaries containing data to be published and directives (list)}
\end{arglist}

Returns:

\begin{itemize}
    \item \refarg{rc} - \refconst{PMIX_SUCCESS} or a negative value corresponding to a PMIx error constant (integer)
\end{itemize}


See \refapi{PMIx_Publish} for description of all relevant attributes and behaviors



%%%%%%%%%%%
\subsection{Client.lookup}
\declareapi{PMIxClient.lookup}

%%%%
\summary

Lookup information published by this or another process with \refapi{PMIx_Publish}.

%%%%
\format

\versionMarker{4.0}
\pyspecificstart
\begin{codepar}
rc,info = myclient.lookup(pdata:list, directives:list)
\end{codepar}
\pyspecificend

\begin{arglist}
\argin{pdata}{List of Python \refpy{pdata} dictionaries identifying data to be retrieved (list)}
\argin{directives}{List of Python \refpy{info} dictionaries (list)}
\end{arglist}

Returns:

\begin{itemize}
    \item \refarg{rc} - \refconst{PMIX_SUCCESS} or a negative value corresponding to a PMIx error constant (integer)
    \item \refarg{info} - Python list of \refpy{info} containing the returned data (list)
\end{itemize}


See \refapi{PMIx_Lookup} for description of all relevant attributes and behaviors


%%%%%%%%%%%
\subsection{Client.unpublish}
\declareapi{PMIxClient.unpublish}

%%%%
\summary

Delete data published by this process with \refapi{PMIx_Publish}.

%%%%
\format

\versionMarker{4.0}
\pyspecificstart
\begin{codepar}
rc = myclient.unpublish(keys:list, directives:list)
\end{codepar}
\pyspecificend

\begin{arglist}
\argin{keys}{List of Python string keys identifying data to be deleted (list)}
\argin{directives}{List of Python \refpy{info} dictionaries (list)}
\end{arglist}

Returns:

\begin{itemize}
    \item \refarg{rc} - \refconst{PMIX_SUCCESS} or a negative value corresponding to a PMIx error constant (integer)
\end{itemize}


See \refapi{PMIx_Unpublish} for description of all relevant attributes and behaviors


%%%%%%%%%%%
\subsection{Client.spawn}
\declareapi{PMIxClient.spawn}

%%%%
\summary

Spawn a new job.

%%%%
\format

\versionMarker{4.0}
\pyspecificstart
\begin{codepar}
rc,nspace = myclient.spawn(jobinfo:list, apps:list)
\end{codepar}
\pyspecificend

\begin{arglist}
\argin{jobinfo}{List of Python \refpy{info} dictionaries (list)}
\argin{apps}{List of Python \refpy{app} dictionaries (list)}
\end{arglist}

Returns:

\begin{itemize}
    \item \refarg{rc} - \refconst{PMIX_SUCCESS} or a negative value corresponding to a PMIx error constant (integer)
    \item \refarg{nspace} - Python \refpy{nspace} of the new job (dict)
\end{itemize}


See \refapi{PMIx_Spawn} for description of all relevant attributes and behaviors


%%%%%%%%%%%
\subsection{Client.connect}
\declareapi{PMIxClient.connect}

%%%%
\summary

Connect namespaces.

%%%%
\format

\versionMarker{4.0}
\pyspecificstart
\begin{codepar}
rc = myclient.connect(peers:list, directives:list)
\end{codepar}
\pyspecificend

\begin{arglist}
\argin{peers}{List of Python \refpy{proc} dictionaries (list)}
\argin{directives}{List of Python \refpy{info} dictionaries (list)}
\end{arglist}

Returns:

\begin{itemize}
    \item \refarg{rc} - \refconst{PMIX_SUCCESS} or a negative value corresponding to a PMIx error constant (integer)
\end{itemize}


See \refapi{PMIx_Connect} for description of all relevant attributes and behaviors


%%%%%%%%%%%
\subsection{Client.disconnect}
\declareapi{PMIxClient.disconnect}

%%%%
\summary

Disconnect namespaces.

%%%%
\format

\versionMarker{4.0}
\pyspecificstart
\begin{codepar}
rc = myclient.disconnect(peers:list, directives:list)
\end{codepar}
\pyspecificend

\begin{arglist}
\argin{peers}{List of Python \refpy{proc} dictionaries (list)}
\argin{directives}{List of Python \refpy{info} dictionaries (list)}
\end{arglist}

Returns:

\begin{itemize}
    \item \refarg{rc} - \refconst{PMIX_SUCCESS} or a negative value corresponding to a PMIx error constant (integer)
\end{itemize}


See \refapi{PMIx_Disconnect} for description of all relevant attributes and behaviors


%%%%%%%%%%%
\subsection{Client.resolve_peers}
\declareapi{PMIxClient.resolve_peers}

%%%%
\summary

Return list of processes within the specified \refpy{nspace} on the given node.

%%%%
\format

\versionMarker{4.0}
\pyspecificstart
\begin{codepar}
rc,procs = myclient.resolve_peers(node:str, nspace:str)
\end{codepar}
\pyspecificend

\begin{arglist}
\argin{node}{Name of node whose processes are being requested (str)}
\argin{nspace}{Python \refpy{nspace} whose processes are to be returned (str)}
\end{arglist}

Returns:

\begin{itemize}
    \item \refarg{rc} - \refconst{PMIX_SUCCESS} or a negative value corresponding to a PMIx error constant (integer)
    \item \refarg{procs} - List of Python \refpy{proc} dictionaries (list)
\end{itemize}


See \refapi{PMIx_Resolve_peers} for description of all relevant attributes and behaviors


%%%%%%%%%%%
\subsection{Client.resolve_nodes}
\declareapi{PMIxClient.resolve_nodes}

%%%%
\summary

Return list of nodes hosting processes within the specified \refpy{nspace}.

%%%%
\format

\versionMarker{4.0}
\pyspecificstart
\begin{codepar}
rc,nodes = myclient.resolve_nodes(nspace:str)
\end{codepar}
\pyspecificend

\begin{arglist}
\argin{nspace}{Python \refpy{nspace} (str)}
\end{arglist}

Returns:

\begin{itemize}
    \item \refarg{rc} - \refconst{PMIX_SUCCESS} or a negative value corresponding to a PMIx error constant (integer)
    \item \refarg{nodes} - List of Python string node names (list)
\end{itemize}


See \refapi{PMIx_Resolve_nodes} for description of all relevant attributes and behaviors


%%%%%%%%%%%
\subsection{Client.query}
\declareapi{PMIxClient.query}

%%%%
\summary

Query information about the system in general

%%%%
\format

\versionMarker{4.0}
\pyspecificstart
\begin{codepar}
rc,info = myclient.query(queries:list, directives:list)
\end{codepar}
\pyspecificend

\begin{arglist}
\argin{queries}{List of Python \refpy{query} dictionaries (list)}
\argin{directives}{List of Python \refpy{info} dictionaries (list)}
\end{arglist}

Returns:

\begin{itemize}
    \item \refarg{rc} - \refconst{PMIX_SUCCESS} or a negative value corresponding to a PMIx error constant (integer)
    \item \refarg{info} - List of Python \refpy{info} containing results of the query (list)
\end{itemize}


See \refapi{PMIx_Query_info_nb} for description of all relevant attributes and behaviors


%%%%%%%%%%%
\subsection{Client.log}
\declareapi{PMIxClient.log}

%%%%
\summary

Log data to a central data service/store

%%%%
\format

\versionMarker{4.0}
\pyspecificstart
\begin{codepar}
rc = myclient.log(data:list, directives:list)
\end{codepar}
\pyspecificend

\begin{arglist}
\argin{data}{List of Python \refpy{info} dictionaries (list)}
\argin{directives}{List of Python \refpy{info} dictionaries (list)}
\end{arglist}

Returns:

\begin{itemize}
    \item \refarg{rc} - \refconst{PMIX_SUCCESS} or a negative value corresponding to a PMIx error constant (integer)
\end{itemize}


See \refapi{PMIx_Log} for description of all relevant attributes and behaviors


%%%%%%%%%%%
\subsection{Client.allocate}
\declareapi{PMIxClient.allocate}

%%%%
\summary

Request an allocation operation from the host resource manager.

%%%%
\format

\versionMarker{4.0}
\pyspecificstart
\begin{codepar}
rc,info = myclient.allocate(request:integer, directives:list)
\end{codepar}
\pyspecificend

\begin{arglist}
\argin{request}{Python \refpy{allocdir} specifying requested operation (integer)}
\argin{directives}{List of Python \refpy{info} dictionaries describing request (list)}
\end{arglist}

Returns:

\begin{itemize}
    \item \refarg{rc} - \refconst{PMIX_SUCCESS} or a negative value corresponding to a PMIx error constant (integer)
    \item \refarg{info} - List of Python \refpy{info} containing results of the request (list)
\end{itemize}


See \refapi{PMIx_Allocation_request_nb} for description of all relevant attributes and behaviors


%%%%%%%%%%%
\subsection{Client.job_ctrl}
\declareapi{PMIxClient.job_ctrl}

%%%%
\summary

Request a job control action

%%%%
\format

\versionMarker{4.0}
\pyspecificstart
\begin{codepar}
rc,info = myclient.job_ctrl(targets:list, directives:list)
\end{codepar}
\pyspecificend

\begin{arglist}
\argin{targets}{List of Python \refpy{proc} specifying targets of requested operation (integer)}
\argin{directives}{List of Python \refpy{info} dictionaries describing request (list)}
\end{arglist}

Returns:

\begin{itemize}
    \item \refarg{rc} - \refconst{PMIX_SUCCESS} or a negative value corresponding to a PMIx error constant (integer)
    \item \refarg{info} - List of Python \refpy{info} containing results of the request (list)
\end{itemize}


See \refapi{PMIx_Job_control_nb} for description of all relevant attributes and behaviors


%%%%%%%%%%%
\subsection{Client.monitor}
\declareapi{PMIxClient.monitor}

%%%%
\summary

Request that something be monitored

%%%%
\format

\versionMarker{4.0}
\pyspecificstart
\begin{codepar}
rc,info = myclient.monitor(targets:list, directives:list)
\end{codepar}
\pyspecificend

\begin{arglist}
\argin{targets}{List of Python \refpy{proc} specifying targets of requested operation (integer)}
\argin{directives}{List of Python \refpy{info} dictionaries describing request (list)}
\end{arglist}

Returns:

\begin{itemize}
    \item \refarg{rc} - \refconst{PMIX_SUCCESS} or a negative value corresponding to a PMIx error constant (integer)
    \item \refarg{info} - List of Python \refpy{info} containing results of the request (list)
\end{itemize}


See \refapi{PMIx_Process_monitor_nb} for description of all relevant attributes and behaviors


%%%%%%%%%%%
\subsection{Client.get_credential}
\declareapi{PMIxClient.get_credential}

%%%%
\summary

Request a credential from the PMIx server/SMS

%%%%
\format

\versionMarker{4.0}
\pyspecificstart
\begin{codepar}
rc,cred,info = myclient.get_credential(directives:list)
\end{codepar}
\pyspecificend

\begin{arglist}
\argin{directives}{List of Python \refpy{info} dictionaries describing request (list)}
\end{arglist}

Returns:

\begin{itemize}
    \item \refarg{rc} - \refconst{PMIX_SUCCESS} or a negative value corresponding to a PMIx error constant (integer)
    \item \refarg{cred} - Python \refpy{byteobject} containing returned credential (dict)
    \item \refarg{info} - List of Python \refpy{info} containing results of the request (list)
\end{itemize}


See \refapi{PMIx_Get_credential} for description of all relevant attributes and behaviors


%%%%%%%%%%%
\subsection{Client.validate_credential}
\declareapi{PMIxClient.validate_credential}

%%%%
\summary

Request validation of a credential by the PMIx server/SMS

%%%%
\format

\versionMarker{4.0}
\pyspecificstart
\begin{codepar}
rc,info = myclient.validate_credential(cred:dict, directives:list)
\end{codepar}
\pyspecificend

\begin{arglist}
\argin{cred}{Python \refpy{byteobject} containing credential (dict)}
\argin{directives}{List of Python \refpy{info} dictionaries describing request (list)}
\end{arglist}

Returns:

\begin{itemize}
    \item \refarg{rc} - \refconst{PMIX_SUCCESS} or a negative value corresponding to a PMIx error constant (integer)
    \item \refarg{info} - List of Python \refpy{info} containing additional results of the request (list)
\end{itemize}


See \refapi{PMIx_Validate_credential} for description of all relevant attributes and behaviors


%%%%%%%%%%%
\subsection{Client.group_construct}
\declareapi{PMIxClient.group_construct}

%%%%
\summary

Construct a new group composed of the specified processes and identified with
the provided group identifier

%%%%
\format

\versionMarker{4.0}
\pyspecificstart
\begin{codepar}
rc,info = myclient.construct_group(grp:string, members:list, directives:list)
\end{codepar}
\pyspecificend

\begin{arglist}
\argin{grp}{Python string identifier for the group (str)}
\argin{members}{List of Python \refpy{proc} dictionaries identifying group members (list)}
\argin{directives}{List of Python \refpy{info} dictionaries describing request (list)}
\end{arglist}

Returns:

\begin{itemize}
    \item \refarg{rc} - \refconst{PMIX_SUCCESS} or a negative value corresponding to a PMIx error constant (integer)
    \item \refarg{info} - List of Python \refpy{info} containing results of the request (list)
\end{itemize}


See \refapi{PMIx_Group_construct} for description of all relevant attributes and behaviors


%%%%%%%%%%%
\subsection{Client.group_invite}
\declareapi{PMIxClient.group_invite}

%%%%
\summary

Explicitly invite specified processes to join a group

%%%%
\format

\versionMarker{4.0}
\pyspecificstart
\begin{codepar}
rc,info = myclient.group_invite(grp:string, members:list, directives:list)
\end{codepar}
\pyspecificend

\begin{arglist}
\argin{grp}{Python string identifier for the group (str)}
\argin{members}{List of Python \refpy{proc} dictionaries identifying processes to be invited (list)}
\argin{directives}{List of Python \refpy{info} dictionaries describing request (list)}
\end{arglist}

Returns:

\begin{itemize}
    \item \refarg{rc} - \refconst{PMIX_SUCCESS} or a negative value corresponding to a PMIx error constant (integer)
    \item \refarg{info} - List of Python \refpy{info} containing results of the request (list)
\end{itemize}


See \refapi{PMIx_Group_invite} for description of all relevant attributes and behaviors


%%%%%%%%%%%
\subsection{Client.group_join}
\declareapi{PMIxClient.group_join}

%%%%
\summary

Respond to an invitation to join a group that is being asynchronously constructed

%%%%
\format

\versionMarker{4.0}
\pyspecificstart
\begin{codepar}
rc,info = myclient.group_join(grp:string, leader:dict, opt:integer, directives:list)
\end{codepar}
\pyspecificend

\begin{arglist}
\argin{grp}{Python string identifier for the group (str)}
\argin{leader}{Python \refpy{proc} dictionary identifying process leading the group (dict)}
\argin{opt}{One of the \refstruct{pmix_group_opt_t} values indicating decline/accept (integer)}
\argin{directives}{List of Python \refpy{info} dictionaries describing request (list)}
\end{arglist}

Returns:

\begin{itemize}
    \item \refarg{rc} - \refconst{PMIX_SUCCESS} or a negative value corresponding to a PMIx error constant (integer)
    \item \refarg{info} - List of Python \refpy{info} containing results of the request (list)
\end{itemize}


See \refapi{PMIx_Group_join} for description of all relevant attributes and behaviors


%%%%%%%%%%%
\subsection{Client.group_leave}
\declareapi{PMIxClient.group_leave}

%%%%
\summary

Leave a PMIx Group

%%%%
\format

\versionMarker{4.0}
\pyspecificstart
\begin{codepar}
rc = myclient.group_leave(grp:string, directives:list)
\end{codepar}
\pyspecificend

\begin{arglist}
\argin{grp}{Python string identifier for the group (str)}
\argin{directives}{List of Python \refpy{info} dictionaries describing request (list)}
\end{arglist}

Returns:

\begin{itemize}
    \item \refarg{rc} - \refconst{PMIX_SUCCESS} or a negative value corresponding to a PMIx error constant (integer)
\end{itemize}


See \refapi{PMIx_Group_leave} for description of all relevant attributes and behaviors


%%%%%%%%%%%
\subsection{Client.group_destruct}
\declareapi{PMIxClient.group_destruct}

%%%%
\summary

Destruct a PMIx Group

%%%%
\format

\versionMarker{4.0}
\pyspecificstart
\begin{codepar}
rc = myclient.group_destruct(grp:string, directives:list)
\end{codepar}
\pyspecificend

\begin{arglist}
\argin{grp}{Python string identifier for the group (str)}
\argin{directives}{List of Python \refpy{info} dictionaries describing request (list)}
\end{arglist}

Returns:

\begin{itemize}
    \item \refarg{rc} - \refconst{PMIX_SUCCESS} or a negative value corresponding to a PMIx error constant (integer)
\end{itemize}


See \refapi{PMIx_Group_destruct} for description of all relevant attributes and behaviors


%%%%%%%%%%%
\subsection{Client.register_event_handler}
\declareapi{PMIxClient.register_event_handler}

%%%%
\summary

Register an event handler to report events.

%%%%
\format

\versionMarker{4.0}
\pyspecificstart
\begin{codepar}
rc,id = myclient.register_event_handler(codes:list, directives:list, cbfunc)
\end{codepar}
\pyspecificend

\begin{arglist}
\argin{codes}{List of Python integer status codes that should be reported to this handler (llist)}
\argin{directives}{List of Python \refpy{info} dictionaries describing request (list)}
\argin{cbfunc}{Python \refpy{evhandler} to be called when event is received (func)}
\end{arglist}

Returns:

\begin{itemize}
    \item \refarg{rc} - \refconst{PMIX_SUCCESS} or a negative value corresponding to a PMIx error constant (integer)
    \item \refarg{id} - \ac{PMIx} reference identifier for handler (integer)
\end{itemize}


See \refapi{PMIx_Register_event_handler} for description of all relevant attributes and behaviors


%%%%%%%%%%%
\subsection{Client.deregister_event_handler}
\declareapi{PMIxClient.deregister_event_handler}

%%%%
\summary

Deregister an event handler

%%%%
\format

\versionMarker{4.0}
\pyspecificstart
\begin{codepar}
myclient.deregister_event_handler(id:integer)
\end{codepar}
\pyspecificend

\begin{arglist}
\argin{id}{\ac{PMIx} reference identifier for handler (integer)}
\end{arglist}

Returns: None

See \refapi{PMIx_Deregister_event_handler} for description of all relevant attributes and behaviors


%%%%%%%%%%%
\subsection{Client.notify_event}
\declareapi{PMIxClient.notify_event}

%%%%
\summary

Report an event for notification via any registered handler.

%%%%
\format

\versionMarker{4.0}
\pyspecificstart
\begin{codepar}
rc = myclient.notify_event(status:integer, source:dict,
                           range:integer, directives:list)
\end{codepar}
\pyspecificend

\begin{arglist}
\argin{status}{\ac{PMIx} status code indicating the event being reported (integer)}
\argin{source}{Python \refpy{proc} of the process that generated the event (dict)}
\argin{range}{Python \refpy{range} in which the event is to be reported (integer)}
\argin{directives}{List of Python \refpy{info} dictionaries describing request (list)}
\end{arglist}

Returns:
\begin{itemize}
    \item \refarg{rc} - \refconst{PMIX_SUCCESS} or a negative value corresponding to a PMIx error constant (integer)
\end{itemize}

See \refapi{PMIx_Notify_event} for description of all relevant attributes and behaviors


%%%%%%%%%%%
\subsection{Client.error_string}
\declareapi{PMIxClient.error_string}

%%%%
\summary

Pretty-print string representation of \refstruct{pmix_status_t}.

%%%%
\format

\versionMarker{4.0}
\pyspecificstart
\begin{codepar}
rep = myclient.error_string(status:integer)
\end{codepar}
\pyspecificend

\begin{arglist}
\argin{status}{\ac{PMIx} status code (integer)}
\end{arglist}

Returns:
\begin{itemize}
    \item \refarg{rep} - String representation of the provided status code (str)
\end{itemize}

See \refapi{PMIx_Error_string} for further details


%%%%%%%%%%%
\subsection{Client.proc_state_string}
\declareapi{PMIxClient.proc_state_string}

%%%%
\summary

Pretty-print string representation of \refstruct{pmix_proc_state_t}.

%%%%
\format

\versionMarker{4.0}
\pyspecificstart
\begin{codepar}
rep = myclient.proc_state_string(state:integer)
\end{codepar}
\pyspecificend

\begin{arglist}
\argin{state}{\ac{PMIx} process state code (integer)}
\end{arglist}

Returns:
\begin{itemize}
    \item \refarg{rep} - String representation of the provided process state (str)
\end{itemize}

See \refapi{PMIx_Proc_state_string} for further details


%%%%%%%%%%%
\subsection{Client.scope_string}
\declareapi{PMIxClient.scope_string}

%%%%
\summary

Pretty-print string representation of \refstruct{pmix_scope_t}.

%%%%
\format

\versionMarker{4.0}
\pyspecificstart
\begin{codepar}
rep = myclient.scope_string(scope:integer)
\end{codepar}
\pyspecificend

\begin{arglist}
\argin{scope}{\ac{PMIx} scope value (integer)}
\end{arglist}

Returns:
\begin{itemize}
    \item \refarg{rep} - String representation of the provided scope (str)
\end{itemize}

See \refapi{PMIx_Scope_string} for further details


%%%%%%%%%%%
\subsection{Client.persistence_string}
\declareapi{PMIxClient.persistence_string}

%%%%
\summary

Pretty-print string representation of \refstruct{pmix_persistence_t}.

%%%%
\format

\versionMarker{4.0}
\pyspecificstart
\begin{codepar}
rep = myclient.persistence_string(persistence:integer)
\end{codepar}
\pyspecificend

\begin{arglist}
\argin{persistence}{\ac{PMIx} persistence value (integer)}
\end{arglist}

Returns:
\begin{itemize}
    \item \refarg{rep} - String representation of the provided persistence (str)
\end{itemize}

See \refapi{PMIx_Persistence_string} for further details


%%%%%%%%%%%
\subsection{Client.data_range_string}
\declareapi{PMIxClient.data_range_string}

%%%%
\summary

Pretty-print string representation of \refstruct{pmix_data_range_t}.

%%%%
\format

\versionMarker{4.0}
\pyspecificstart
\begin{codepar}
rep = myclient.data_range_string(range:integer)
\end{codepar}
\pyspecificend

\begin{arglist}
\argin{range}{\ac{PMIx} data range value (integer)}
\end{arglist}

Returns:
\begin{itemize}
    \item \refarg{rep} - String representation of the provided data range (str)
\end{itemize}

See \refapi{PMIx_Data_range_string} for further details


%%%%%%%%%%%
\subsection{Client.info_directives_string}
\declareapi{PMIxClient.info_directives_string}

%%%%
\summary

Pretty-print string representation of \refstruct{pmix_info_directives_t}.

%%%%
\format

\versionMarker{4.0}
\pyspecificstart
\begin{codepar}
rep = myclient.info_directives_string(directives:integer)
\end{codepar}
\pyspecificend

\begin{arglist}
\argin{directives}{\ac{PMIx} info directives value (integer)}
\end{arglist}

Returns:
\begin{itemize}
    \item \refarg{rep} - String representation of the provided info directives (str)
\end{itemize}

See \refapi{PMIx_Info_directives_string} for further details


%%%%%%%%%%%
\subsection{Client.data_type_string}
\declareapi{PMIxClient.data_type_string}

%%%%
\summary

Pretty-print string representation of \refstruct{pmix_data_type_t}.

%%%%
\format

\versionMarker{4.0}
\pyspecificstart
\begin{codepar}
rep = myclient.data_type_string(dtype:integer)
\end{codepar}
\pyspecificend

\begin{arglist}
\argin{dtype}{\ac{PMIx} datatype value (integer)}
\end{arglist}

Returns:
\begin{itemize}
    \item \refarg{rep} - String representation of the provided datatype (str)
\end{itemize}

See \refapi{PMIx_Data_type_string} for further details


%%%%%%%%%%%
\subsection{Client.alloc_directive_string}
\declareapi{PMIxClient.alloc_directive_string}

%%%%
\summary

Pretty-print string representation of \refstruct{pmix_alloc_directive_t}.

%%%%
\format

\versionMarker{4.0}
\pyspecificstart
\begin{codepar}
rep = myclient.alloc_directive_string(adir:integer)
\end{codepar}
\pyspecificend

\begin{arglist}
\argin{adir}{\ac{PMIx} allocation directive value (integer)}
\end{arglist}

Returns:
\begin{itemize}
    \item \refarg{rep} - String representation of the provided allocation directive (str)
\end{itemize}

See \refapi{PMIx_Alloc_directive_string} for further details


%%%%%%%%%%%
\subsection{Client.iof_channel_string}
\declareapi{PMIxClient.iof_channel_string}

%%%%
\summary

Pretty-print string representation of \refstruct{pmix_iof_channel_t}.

%%%%
\format

\versionMarker{4.0}
\pyspecificstart
\begin{codepar}
rep = myclient.iof_channel_string(channel:integer)
\end{codepar}
\pyspecificend

\begin{arglist}
\argin{channel}{\ac{PMIx} IOF channel value (integer)}
\end{arglist}

Returns:
\begin{itemize}
    \item \refarg{rep} - String representation of the provided IOF channel (str)
\end{itemize}

See \refapi{PMIx_IOF_channel_string} for further details


%%%%%%%%%%%%%%%%%%%%%%%%%%%%%%%%%%%%%%%%%%%%

\section{PMIxServer}
\label{app:python:server}

The server Python class inherits the Python "client" class as its parent. Thus, it includes all client functions in addition to the ones defined in this section.

%%%%%%%%%%%
\subsection{Server.init}
\declareapi{PMIxServer.init}

\summary Initialize the \ac{PMIx} server library after obtaining a new PMIxServer object

\format

\versionMarker{4.0}
\pyspecificstart
\begin{codepar}
rc = myserver.init(directives:list, map:dict)
\end{codepar}
\pyspecificend


\begin{arglist}
\argin{directives}{List of Python \refpy{info} dictionaries (list)}
\argin{map}{Python dictionary key-function pairs that map \refpy{server module} callback functions to provided implementations (dict)}
\end{arglist}

Returns:

\begin{itemize}
    \item \refarg{rc} - \refconst{PMIX_SUCCESS} or a negative value corresponding to a PMIx error constant (integer)
\end{itemize}


See \refapi{PMIx_server_init} for description of all relevant attributes and behaviors


%%%%%%%%%%%
\subsection{Server.finalize}
\declareapi{PMIxServer.finalize}

\summary Finalize the \ac{PMIx} server library

\format

\versionMarker{4.0}
\pyspecificstart
\begin{codepar}
rc = myserver.finalize()
\end{codepar}
\pyspecificend


Returns:

\begin{itemize}
    \item \refarg{rc} - \refconst{PMIX_SUCCESS} or a negative value corresponding to a PMIx error constant (integer)
\end{itemize}


See \refapi{PMIx_server_finalize} for details


%%%%%%%%%%%
\subsection{Server.generate_regex}
\declareapi{PMIxServer.generate_regex}

\summary
Generate a regular expression representation of the input strings.

\format

\versionMarker{4.0}
\pyspecificstart
\begin{codepar}
rc,regex = myserver.generate_regex(input:list)
\end{codepar}
\pyspecificend


\begin{arglist}
\argin{input}{List of Python strings (e.g., node names)  (list)}
\end{arglist}

Returns:

\begin{itemize}
    \item \refarg{rc} - \refconst{PMIX_SUCCESS} or a negative value corresponding to a PMIx error constant (integer)
    \item \refarg{regex} - Python string containing regular expression representation of the input list (str)
\end{itemize}


See \refapi{PMIx_generate_regex} for details


%%%%%%%%%%%
\subsection{Server.generate_ppn}
\declareapi{PMIxServer.generate_ppn}

\summary
Generate a regular expression representation of the input strings.

\format

\versionMarker{4.0}
\pyspecificstart
\begin{codepar}
rc,regex = myserver.generate_ppn(input:list)
\end{codepar}
\pyspecificend


\begin{arglist}
\argin{input}{List of Python strings describing the ranks on each node (list)}
\end{arglist}

Returns:

\begin{itemize}
    \item \refarg{rc} - \refconst{PMIX_SUCCESS} or a negative value corresponding to a PMIx error constant (integer)
    \item \refarg{regex} - Python string containing regular expression representation of the input list (str)
\end{itemize}


See \refapi{PMIx_generate_ppn} for details


%%%%%%%%%%%
\subsection{Server.register_nspace}
\declareapi{PMIxServer.register_nspace}

\summary Setup the data about a particular namespace.

\format

\versionMarker{4.0}
\pyspecificstart
\begin{codepar}
rc = myserver.register_nspace(nspace:str,
                              nlocalprocs:integer,
                              directives:list)
\end{codepar}
\pyspecificend


\begin{arglist}
\argin{nspace}{Python string containing the namespace (str)}
\argin{nlocalprocs}{Number of local processes (integer)}
\argin{directives}{List of Python \refpy{info} dictionaries (list)}
\end{arglist}

Returns:

\begin{itemize}
    \item \refarg{rc} - \refconst{PMIX_SUCCESS} or a negative value corresponding to a PMIx error constant (integer)
\end{itemize}


See \refapi{PMIx_server_register_nspace} for description of all relevant attributes and behaviors


%%%%%%%%%%%
\subsection{Server.deregister_nspace}
\declareapi{PMIxServer.deregister_nspace}

\summary Deregister a namespace.

\format

\versionMarker{4.0}
\pyspecificstart
\begin{codepar}
myserver.deregister_nspace(nspace:str)
\end{codepar}
\pyspecificend


\begin{arglist}
\argin{nspace}{Python string containing the namespace (str)}
\end{arglist}

Returns: None


See \refapi{PMIx_server_deregister_nspace} for details


%%%%%%%%%%%
\subsection{Server.register_client}
\declareapi{PMIxServer.register_client}

\summary
Register a client process with the PMIx server library.


\format

\versionMarker{4.0}
\pyspecificstart
\begin{codepar}
rc = myserver.register_client(proc:dict, uid:integer, gid:integer)
\end{codepar}
\pyspecificend


\begin{arglist}
\argin{proc}{Python \refpy{proc} dictionary identifying the client process (dict)}
\argin{uid}{Linux uid value for user executing client process (integer)}
\argin{gid}{Linux gid value for user executing client process (integer)}
\end{arglist}

Returns:

\begin{itemize}
    \item \refarg{rc} - \refconst{PMIX_SUCCESS} or a negative value corresponding to a PMIx error constant (integer)
\end{itemize}


See \refapi{PMIx_server_register_client} for details


%%%%%%%%%%%
\subsection{Server.deregister_client}
\declareapi{PMIxServer.deregister_client}

\summary
Dergister a client process and purge all data relating to it


\format

\versionMarker{4.0}
\pyspecificstart
\begin{codepar}
myserver.deregister_client(proc:dict)
\end{codepar}
\pyspecificend


\begin{arglist}
\argin{proc}{Python \refpy{proc} dictionary identifying the client process (dict)}
\end{arglist}

Returns: None


See \refapi{PMIx_server_deregister_client} for details


%%%%%%%%%%%
\subsection{Server.setup_fork}
\declareapi{PMIxServer.setup_fork}

\summary
Setup the environment of a child process that is to be forked
by the host

\format

\versionMarker{4.0}
\pyspecificstart
\begin{codepar}
rc = myserver.setup_fork(proc:dict, envin:dict)
\end{codepar}
\pyspecificend


\begin{arglist}
\argin{proc}{Python \refpy{proc} dictionary identifying the client process (dict)}
\arginout{envin}{Python dictionary containing the environment to be passed to the client (dict)}
\end{arglist}

Returns:

\begin{itemize}
    \item \refarg{rc} - \refconst{PMIX_SUCCESS} or a negative value corresponding to a PMIx error constant (integer)
\end{itemize}


See \refapi{PMIx_server_setup_fork} for details


%%%%%%%%%%%
\subsection{Server.dmodex_request}
\declareapi{PMIxServer.dmodex_request}

\summary
Function by which the host server can request modex data from the local PMIx server.

\format

\versionMarker{4.0}
\pyspecificstart
\begin{codepar}
rc,data = myserver.dmodex_request(proc:dict)
\end{codepar}
\pyspecificend


\begin{arglist}
\argin{proc}{Python \refpy{proc} dictionary identifying the process whose data is requested (dict)}
\end{arglist}

Returns:

\begin{itemize}
    \item \refarg{rc} - \refconst{PMIX_SUCCESS} or a negative value corresponding to a PMIx error constant (integer)
    \item \refarg{data} - Python \refpy{byteobject} containing the returned data (dict)
\end{itemize}


See \refapi{PMIx_server_dmodex_request} for details


%%%%%%%%%%%
\subsection{Server.setup_application}
\declareapi{PMIxServer.setup_application}

\summary
Function by which the resource manager can request application-specific setup data prior to launch of a \refterm{job}.

\format

\versionMarker{4.0}
\pyspecificstart
\begin{codepar}
rc,info = myserver.setup_application(nspace:str, directives:list)
\end{codepar}
\pyspecificend


\begin{arglist}
\argin{nspace}{Namespace whose setup information is being requested (str)}
\argin{directives}{Python list of \refpy{info} directives}
\end{arglist}

Returns:

\begin{itemize}
    \item \refarg{rc} - \refconst{PMIX_SUCCESS} or a negative value corresponding to a PMIx error constant (integer)
    \item \refarg{info} - Python list of \refpy{info} dictionaries containing the returned data (list)
\end{itemize}


See \refapi{PMIx_server_setup_application} for details


%%%%%%%%%%%
\subsection{Server.register_attributes}
\declareapi{PMIxServer.register_attributes}

\summary
Register host environment attribute support for a function.

\format

\versionMarker{4.0}
\pyspecificstart
\begin{codepar}
rc = myserver.register_attributes(function:str, attrs:list)
\end{codepar}
\pyspecificend


\begin{arglist}
\argin{function}{Name of the function (str)}
\argin{attrs}{Python list of \refpy{regattr} dictionaries describing the supported attributes}
\end{arglist}

Returns:

\begin{itemize}
    \item \refarg{rc} - \refconst{PMIX_SUCCESS} or a negative value corresponding to a PMIx error constant (integer)
\end{itemize}


See \refapi{PMIx_Register_attributes} for details


%%%%%%%%%%%
\subsection{Server.setup_local_support}
\declareapi{PMIxServer.setup_local_support}

\summary
Function by which the local \ac{PMIx} server can perform any application-specific operations prior to spawning local clients of a given application

\format

\versionMarker{4.0}
\pyspecificstart
\begin{codepar}
rc = myserver.setup_local_support(nspace:str, info:list)
\end{codepar}
\pyspecificend


\begin{arglist}
\argin{nspace}{Namespace whose setup information is being requested (str)}
\argin{info}{Python list of \refpy{info} dictionaries containing the setup data (list)}
\end{arglist}

Returns:

\begin{itemize}
    \item \refarg{rc} - \refconst{PMIX_SUCCESS} or a negative value corresponding to a PMIx error constant (integer)
\end{itemize}


See \refapi{PMIx_server_setup_local_support} for details


%%%%%%%%%%%
\subsection{Server.iof_deliver}
\declareapi{PMIxServer.iof_deliver}

\summary
Function by which the host environment can pass forwarded \ac{IO} to the \ac{PMIx} server library for distribution to its clients.

\format

\versionMarker{4.0}
\pyspecificstart
\begin{codepar}
rc = myserver.iof_deliver(source:dict, channel:integer,
                          data:dict, directives:list)
\end{codepar}
\pyspecificend


\begin{arglist}
\argin{source}{Python \refpy{proc} dictionary identifying the process who generated the data (dict)}
\argin{channel}{Python \refpy{channel} bitmask identifying IO channel of the provided data (integer)}
\argin{data}{Python \refpy{byteobject} containing the data (dict)}
\argin{directives}{Python list of \refpy{info} dictionaries containing directives (list)}
\end{arglist}

Returns:

\begin{itemize}
    \item \refarg{rc} - \refconst{PMIX_SUCCESS} or a negative value corresponding to a PMIx error constant (integer)
\end{itemize}


See \refapi{PMIx_server_IOF_deliver} for details


%%%%%%%%%%%
\subsection{Server.collect_inventory}
\declareapi{PMIxServer.collect_inventory}

\summary
Collect inventory of resources on a node

\format

\versionMarker{4.0}
\pyspecificstart
\begin{codepar}
rc,info = myserver.collect_inventory(directives:list)
\end{codepar}
\pyspecificend


\begin{arglist}
\argin{directives}{Python list of \refpy{info} dictionaries containing directives (list)}
\end{arglist}

Returns:

\begin{itemize}
    \item \refarg{rc} - \refconst{PMIX_SUCCESS} or a negative value corresponding to a PMIx error constant (integer)
    \item \refarg{info} - Python list of \refpy{info} dictionaries containing the returned data (list)
\end{itemize}


See \refapi{PMIx_server_collect_inventory} for details


%%%%%%%%%%%
\subsection{Server.deliver_inventory}
\declareapi{PMIxServer.deliver_inventory}

\summary
Pass collected inventory to the \ac{PMIx} server library for storage

\format

\versionMarker{4.0}
\pyspecificstart
\begin{codepar}
rc = myserver.deliver_inventory(info:list, directives:list)
\end{codepar}
\pyspecificend


\begin{arglist}
\argin{info} - Python list of \refpy{info} dictionaries containing the inventory data (list)
\argin{directives}{Python list of \refpy{info} dictionaries containing directives (list)}
\end{arglist}

Returns:

\begin{itemize}
    \item \refarg{rc} - \refconst{PMIX_SUCCESS} or a negative value corresponding to a PMIx error constant (integer)
\end{itemize}


See \refapi{PMIx_server_deliver_inventory} for details


%%%%%%%%%%%%%%%%%%%%%%%%%%%%%%%%%%%%%%%%%%%%

\section{PMIxTool}
\label{app:python:tool}

The tool Python class inherits the Python "server" class as its parent. Thus, it includes all client and server functions in addition to the ones defined in this section.

%%%%%%%%%%%
\subsection{Tool.init}
\declareapi{PMIxTool.init}

\summary Initialize the \ac{PMIx} tool library after obtaining a new PMIxTool object

\format

\versionMarker{4.0}
\pyspecificstart
\begin{codepar}
rc,proc = mytool.init(info:list)
\end{codepar}
\pyspecificend


\begin{arglist}
\argin{info}{List of Python \refpy{info} dictionaries (list)}
\end{arglist}

Returns:

\begin{itemize}
    \item \refarg{rc} - \refconst{PMIX_SUCCESS} or a negative value corresponding to a PMIx error constant (integer)
    \item \refarg{proc} - a Python \refpy{proc} dictionary (dict)
\end{itemize}


See \refapi{PMIx_tool_init} for description of all relevant attributes and behaviors


%%%%%%%%%%%
\subsection{Tool.finalize}
\declareapi{PMIxTool.finalize}

\summary Finalize the PMIx tool library, closing the connection to the server.

\format

\versionMarker{4.0}
\pyspecificstart
\begin{codepar}
rc = mytool.finalize()
\end{codepar}
\pyspecificend


Returns:

\begin{itemize}
    \item \refarg{rc} - \refconst{PMIX_SUCCESS} or a negative value corresponding to a PMIx error constant (integer)
\end{itemize}


See \refapi{PMIx_tool_finalize} for description of all relevant attributes and behaviors


%%%%%%%%%%%
\subsection{Tool.connect_to_server}
\declareapi{PMIxTool.connect_to_server}

\summary
Switch connection from the current \ac{PMIx} server to another one, or initialize a connection to a specified server.


\format

\versionMarker{4.0}
\pyspecificstart
\begin{codepar}
rc,proc = mytool.connect_to_server(info:list)
\end{codepar}
\pyspecificend


\begin{arglist}
\argin{info}{List of Python \refpy{info} dictionaries (list)}
\end{arglist}

Returns:

\begin{itemize}
    \item \refarg{rc} - \refconst{PMIX_SUCCESS} or a negative value corresponding to a PMIx error constant (integer)
    \item \refarg{proc} - a Python \refpy{proc} dictionary (dict)
\end{itemize}


See \refapi{PMIx_tool_connect_to_server} for description of all relevant attributes and behaviors


%%%%%%%%%%%
\subsection{Tool.iof_pull}
\declareapi{PMIxTool.iof_pull}

%%%%
\summary

Register to receive output forwarded from a remote process.

%%%%
\format

\versionMarker{4.0}
\pyspecificstart
\begin{codepar}
rc,id = mytool.iof_pull(sources:list, channel:integer, directives:list, cbfunc)
\end{codepar}
\pyspecificend

\begin{arglist}
\argin{sources}{List of Python \refpy{proc} dictionaries of processes whose IO is being requested (list)}
\argin{channel}{Python \refpy{channel} bitmask identifying IO channels to be forwarded (integer)}
\argin{directives}{List of Python \refpy{info} dictionaries describing request (list)}
\argin{cbfunc}{Python \refpy{iofcbfunc} to receive IO payloads (func)}
\end{arglist}

Returns:

\begin{itemize}
    \item \refarg{rc} - \refconst{PMIX_SUCCESS} or a negative value corresponding to a PMIx error constant (integer)
    \item \refarg{id} - \ac{PMIx} reference identifier for request (integer)
\end{itemize}


See \refapi{PMIx_IOF_pull} for description of all relevant attributes and behaviors


%%%%%%%%%%%
\subsection{Tool.iof_deregister}
\declareapi{PMIxTool.iof_deregister}

%%%%
\summary

Deregister from output forwarded from a remote process.

%%%%
\format

\versionMarker{4.0}
\pyspecificstart
\begin{codepar}
rc = mytool.iof_deregister(id:integer, directives:list)
\end{codepar}
\pyspecificend

\begin{arglist}
\argin{id}{\ac{PMIx} reference identifier returned by pull request (list)}
\argin{directives}{List of Python \refpy{info} dictionaries describing request (list)}
\end{arglist}

Returns:

\begin{itemize}
    \item \refarg{rc} - \refconst{PMIX_SUCCESS} or a negative value corresponding to a PMIx error constant (integer)
\end{itemize}


See \refapi{PMIx_IOF_deregister} for description of all relevant attributes and behaviors


%%%%%%%%%%%
\subsection{Tool.iof_push}
\declareapi{PMIxTool.iof_push}

%%%%
\summary

Push data collected locally (typically from stdin) to
stdin of target recipients

%%%%
\format

\versionMarker{4.0}
\pyspecificstart
\begin{codepar}
rc = mytool.iof_push(targets:list, data:dict, directives:list)
\end{codepar}
\pyspecificend

\begin{arglist}
\argin{sources}{List of Python \refpy{proc} dictionaries of target processes (list)}
\argin{data}{Python \refpy{byteobject} dictionary containing data to be delivered (dict)}
\argin{directives}{List of Python \refpy{info} dictionaries describing request (list)}
\end{arglist}

Returns:

\begin{itemize}
    \item \refarg{rc} - \refconst{PMIX_SUCCESS} or a negative value corresponding to a PMIx error constant (integer)
\end{itemize}


See \refapi{PMIx_IOF_push} for description of all relevant attributes and behaviors


