%%%%%%%%%%%%%%%%%%%%%%%%%%%%%%%%%%%%%%%%%%%%%%%%%
% Chapter: Initialization & Finalization
%%%%%%%%%%%%%%%%%%%%%%%%%%%%%%%%%%%%%%%%%%%%%%%%%
\chapter{Initialization and Finalization}
\label{chap:api_init}

The \ac{PMIx} library is required to be initialized and finalized around the usage of most of the \acp{API}.
The \acp{API} that may be used outside of the initialized and finalized region are noted.
All other \acp{API} must be used inside this region.

There are three sets of initialization and finalization functions depending upon the role of the process in the \ac{PMIx} universe.
Each of these functional sets are described in this chapter.
Note that a process can only call \textit{one} of the init/finalize functional pairs - e.g., a process that calls the client initialization function cannot also call the tool or server initialization functions, and must call the corresponding client finalize.

\adviceuserstart
Processes that initialize as a server or tool automatically are given access to all client \acp{API}.
Server initialization includes setting up the infrastructure to support local clients - thus, it necessarily includes overhead and an increased memory footprint.
Tool initialization automatically searches for a server to which it can connect --- if declared as a \textit{launcher}, the \ac{PMIx} library sets up the required ``hooks'' for other tools (e.g., debuggers) to attach to it.
\adviceuserend


%%%%%%%%%%%%%%%%%%%%%%%%%%%%%%%%%%%%%%%%%%%%%%
%%%%%%%%%%%%%%%%%%%%%%%%%%%%%%%%%%%%%%%%%%%%%%
\section{Query}
\label{chap:api_init:general}

The \ac{API} defined in this section can be used by any \ac{PMIx} process, regardless of their role in the \ac{PMIx} universe.

%%%%%%%%%%%
\subsection{\code{PMIx_Initialized}}
\declareapi{PMIx_Initialized}

%%%%
\format

\versionMarker{1.0}
\cspecificstart
\begin{codepar}
int PMIx_Initialized(void)
\end{codepar}
\cspecificend

A value of \code{1} (true) will be returned if the \ac{PMIx} library has been initialized, and \code{0} (false) otherwise.

\rationalestart
The return value is an integer for historical reasons as that was the signature of prior PMI libraries.
\rationaleend

%%%%
\descr

Check to see if the \ac{PMIx} library has been initialized using any of the init functions:
\refapi{PMIx_Init}, \refapi{PMIx_server_init}, or \refapi{PMIx_tool_init}.

%%%%%%%%%%%
\subsection{\code{PMIx_Get_version}}
\declareapi{PMIx_Get_version}

%%%%
\summary

Get the \ac{PMIx} version information.

%%%%
\format

\versionMarker{1.0}
\cspecificstart
\begin{codepar}
const char* PMIx_Get_version(void)
\end{codepar}
\cspecificend

%%%%
\descr

Get the \ac{PMIx} version string.
Note that the provided string is statically defined and must \textit{not} be free'd.

%%%%%%%%%%%%%%%%%%%%%%%%%%%%%%%%%%%%%%%%%%%%%%
%%%%%%%%%%%%%%%%%%%%%%%%%%%%%%%%%%%%%%%%%%%%%%
\section{Client Initialization and Finalization}
\label{chap:api_init:client}

Initialization and finalization routines for \ac{PMIx} clients.

\adviceuserstart
The \ac{PMIx} \textit{ad hoc} v1.0 Standard defined the \refapi{PMIx_Init} function, but modified the function signature in the v1.2 version. The \textit{ad hoc} v1.0 version is not included in this document to avoid confusion.
\adviceuserend


%%%%%%%%%%%
\subsection{\code{PMIx_Init}}
\declareapi{PMIx_Init}

%%%%
\summary

Initialize the \ac{PMIx} client library

%%%%
\format

\versionMarker{1.2}
\cspecificstart
\begin{codepar}
pmix_status_t
PMIx_Init(pmix_proc_t *proc,
          pmix_info_t info[], size_t ninfo)
\end{codepar}
\cspecificend

\begin{arglist}
\arginout{proc}{proc structure (handle)}
\argin{info}{Array of \refstruct{pmix_info_t} structures (array of handles)}
\argin{ninfo}{Number of element in the \refarg{info} array (\code{size_t})}
\end{arglist}

Returns \refconst{PMIX_SUCCESS} or a negative value corresponding to a \ac{PMIx} error constant.

\optattrstart
The following attributes are optional for implementers of \ac{PMIx} libraries:

\pasteAttributeItemBegin{PMIX_USOCK_DISABLE} If the library supports Unix socket connections, this attribute may be supported for disabling it.
\pasteAttributeItemEnd{}
\pasteAttributeItemBegin{PMIX_SOCKET_MODE} If the library supports socket connections, this attribute may be supported for setting the socket mode.
\pasteAttributeItemEnd{}
\pasteAttributeItemBegin{PMIX_SINGLE_LISTENER} If the library supports multiple methods for clients to connect to servers, this attribute may be supported for disabling all but one of them.
\pasteAttributeItemEnd{}
\pasteAttributeItemBegin{PMIX_TCP_REPORT_URI} If the library supports TCP socket connections, this attribute may be supported for reporting the URI.
\pasteAttributeItemEnd{}
\pasteAttributeItemBegin{PMIX_TCP_IF_INCLUDE} If the library supports TCP socket connections, this attribute may be supported for specifying the interfaces to be used.
\pasteAttributeItemEnd{}
\pasteAttributeItemBegin{PMIX_TCP_IF_EXCLUDE} If the library supports TCP socket connections, this attribute may be supported for specifying the interfaces that are \textit{not} to be used.
\pasteAttributeItemEnd{}
\pasteAttributeItemBegin{PMIX_TCP_IPV4_PORT} If the library supports IPV4 connections, this attribute may be supported for specifying the port to be used.
\pasteAttributeItemEnd{}
\pasteAttributeItemBegin{PMIX_TCP_IPV6_PORT} If the library supports IPV6 connections, this attribute may be supported for specifying the port to be used.
\pasteAttributeItemEnd{}
\pasteAttributeItemBegin{PMIX_TCP_DISABLE_IPV4} If the library supports IPV4 connections, this attribute may be supported for disabling it.
\pasteAttributeItemEnd{}
\pasteAttributeItemBegin{PMIX_TCP_DISABLE_IPV6} If the library supports IPV6 connections, this attribute may be supported for disabling it.
\pasteAttributeItemEnd{}
\pasteAttributeItem{PMIX_EVENT_BASE}
\pasteAttributeItemBegin{PMIX_GDS_MODULE} This attribute is specific to the \ac{PRI} and controls only the selection of \ac{GDS} module for internal use by the process. Module selection for interacting with the server is performed dynamically during the connection process.
\pasteAttributeItemEnd{}
\optattrend

%%%%
\descr

Initialize the \ac{PMIx} client, returning the process identifier assigned to this client's application in the provided \refstruct{pmix_proc_t} struct.
Passing a value of \code{NULL} for this parameter is allowed if the user wishes solely to initialize the \ac{PMIx} system and does not require return of the identifier at that time.

When called, the \ac{PMIx} client shall check for the required connection information of the local \ac{PMIx} server and establish the connection.
If the information is not found, or the server connection fails, then an appropriate error constant shall be returned.

If successful, the function shall return \refconst{PMIX_SUCCESS} and fill the \refarg{proc} structure (if provided) with the server-assigned namespace and rank of the process within the application.
In addition, all startup information provided by the resource manager shall be made available to the client process via subsequent calls to \refapi{PMIx_Get}.

The \ac{PMIx} client library shall be reference counted, and so multiple calls to \refapi{PMIx_Init} are allowed by the standard.
Thus, one way for an application process to obtain its namespace and rank is to simply call \refapi{PMIx_Init} with a non-NULL \refarg{proc} parameter.
Note that each call to \refapi{PMIx_Init} must be balanced with a call to \refapi{PMIx_Finalize} to maintain the reference count.

Each call to \refapi{PMIx_Init} may contain an array of \refstruct{pmix_info_t} structures passing directives to the \ac{PMIx} client library as per the above attributes.

Multiple calls to \refapi{PMIx_Init} shall not include conflicting directives.
The \refapi{PMIx_Init} function will return an error when directives that conflict with prior directives are encountered.


%%%%%%%%%%%
\subsection{\code{PMIx_Finalize}}
\declareapi{PMIx_Finalize}

%%%%
\summary

Finalize the PMIx client library.

%%%%
\format

\versionMarker{1.0}
\cspecificstart
\begin{codepar}
pmix_status_t
PMIx_Finalize(const pmix_info_t info[], size_t ninfo)
\end{codepar}
\cspecificend

\begin{arglist}
\argin{info}{Array of \refstruct{pmix_info_t} structures (array of handles)}
\argin{ninfo}{Number of element in the \refarg{info} array (\code{size_t})}
\end{arglist}

Returns \refconst{PMIX_SUCCESS} or a negative value corresponding to a PMIx error constant.

\optattrstart
The following attributes are optional for implementers of \ac{PMIx} libraries:

\pasteAttributeItem{PMIX_EMBED_BARRIER}
\optattrend

%%%%
\descr

Decrement the \ac{PMIx} client library reference count.
When the reference count reaches zero, the library will finalize the \ac{PMIx} client, closing the connection with the local \ac{PMIx} server and releasing all internally allocated memory.


%%%%%%%%%%%%%%%%%%%%%%%%%%%%%%%%%%%%%%%%%%%%%%
%%%%%%%%%%%%%%%%%%%%%%%%%%%%%%%%%%%%%%%%%%%%%%
