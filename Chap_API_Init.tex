%%%%%%%%%%%%%%%%%%%%%%%%%%%%%%%%%%%%%%%%%%%%%%%%%
% Chapter: Initialization & Finalization
%%%%%%%%%%%%%%%%%%%%%%%%%%%%%%%%%%%%%%%%%%%%%%%%%
\chapter{Initialization and Finalization}
\label{chap:api_init}

% RALPH

The \ac{PMIx} library is required to be initialized and finalized around the usage of most of the APIs.
The APIs that may be used outside of the initialized and finalized region are noted.
All other APIs must be used inside this region.

There are three sets of initialization and finalization functions depending upon the role of the process in the PMIx universe.
Each of these functional sets are described in this chapter.


%%%%%%%%%%%%%%%%%%%%%%%%%%%%%%%%%%%%%%%%%%%%%%
%%%%%%%%%%%%%%%%%%%%%%%%%%%%%%%%%%%%%%%%%%%%%%
\section{Query}
\label{chap:api_init:general}

The APIs defined in this section can be used by any PMIx process, regardless of their role in the PMIx universe.

%%%%%%%%%%%
\subsection{\code{PMIx_Initialized}}
\declareapi{PMIx_Initialized}

%%%%
\format

\cspecificstart
\begin{codepar}
int PMIx_Initialized(void)
\end{codepar}
\cspecificend

A value of \code{1} (true) will be returned if the PMIx library has been initialized, and \code{0} (false) otherwise.

\rationalestart
The return value is an integer for historical reasons as that was the signature of prior PMI libraries.
\rationaleend

%%%%
\descr

Check to see if the PMIx library has been initialized using any of the init functions:
\refapi{PMIx_Init}, \refapi{PMIx_server_init}, or \refapi{PMIx_tool_init}.

%%%%%%%%%%%
\subsection{\code{PMIx_Get_version}}
\declareapi{PMIx_Get_version}

%%%%
\summary

Get the PMIx version information.

%%%%
\format

\cspecificstart
\begin{codepar}
const char* PMIx_Get_version(void)
\end{codepar}
\cspecificend

%%%%
\descr

Get the \ac{PMIx} version string.
Note that the provided string is statically defined and must \textit{not} be free'd.

%%%%%%%%%%%%%%%%%%%%%%%%%%%%%%%%%%%%%%%%%%%%%%
%%%%%%%%%%%%%%%%%%%%%%%%%%%%%%%%%%%%%%%%%%%%%%
\section{Client}
\label{chap:api_init:client}

Initialization and finalization routines for \ac{PMIx} clients.

%%%%%%%%%%%
\subsection{\code{PMIx_Init}}
\declareapi{PMIx_Init}

%%%%
\summary

Initialize the \ac{PMIx} client.

%%%%
\format

\cspecificstart
\begin{codepar}
pmix_status_t
PMIx_Init(pmix_proc_t *proc,
          pmix_info_t info[], size_t ninfo)
\end{codepar}
\cspecificend

\begin{arglist}
\arginout{proc}{proc structure (handle)}
\argin{info}{Array of \refattr{pmix_info_t} structures (array of handles)}
\argin{ninfo}{Number of element in the \refarg{info} array (\code{size_t})}
\end{arglist}

Returns \refconst{PMIX_SUCCESS} or a negative value corresponding to a \ac{PMIx} error constant.

\reqattr
The following attributes must be supported by a \ac{PMIx} implementation.

\pasteAttributeItem{PMIX_EVENT_BASE}
\pasteAttributeItem{PMIX_SYSTEM_TMPDIR}
\pasteAttributeItem{PMIX_SYSTEM_TMPDIR}


\optattr
The following attributes are optional, but should be supported by a complete PMIx implementation:

\pasteAttributeItem{PMIX_MODEL_LIBRARY_NAME}
\pasteAttributeItem{PMIX_MODEL_LIBRARY_VERSION}
\pasteAttributeItem{PMIX_THREADING_MODEL}
\pasteAttributeItem{PMIX_USOCK_DISABLE}

%%%%
\descr

Initialize the \ac{PMIx} client, returning the process identifier assigned to this client's application in the provided \refstruct{pmix_proc_t} struct.
Passing a value of \code{NULL} for this parameter is allowed if the user wishes solely to initialize the \ac{PMIx} system and does not require return of the identifier at that time.

When called, the \ac{PMIx} client shall check for the required connection information of the local \ac{PMIx} server and establish the connection.
If the information is not found, or the server connection fails, then an appropriate error constant shall be returned.

If successful, the function shall return \refconst{PMIX_SUCCESS} and fill the \refarg{proc} structure (if provided) with the server-assigned namespace and rank of the process within the application.
In addition, all startup information provided by the resource manager shall be made available to the client process via subsequent calls to \refapi{PMIx_Get}.

The \ac{PMIx} client library shall be reference counted, and so multiple calls to \refapi{PMIx_Init} are allowed by the standard.
Thus, one way for an application process to obtain its namespace and rank is to simply call \refapi{PMIx_Init} with a non-NULL \refarg{proc} parameter.
Note that each call to \refapi{PMIx_Init} must be balanced with a call to \refapi{PMIx_Finalize} to maintain the reference count.

Each call to \refapi{PMIx_Init} may contain an array of \refstruct{pmix_info_t} structures passing directives to the \ac{PMIx} client library.
This might include information about the location of temporary directories set up for the application, or constraints on communication protocols for connecting to the local \ac{PMIx} server.

Multiple calls to \refapi{PMIx_Init} shall not include conflicting directives (e.g., a directive indicating that one particular communication method be used to connect to the server, followed by a subsequent call that includes a directive that a different method be used).
The \refapi{PMIx_Init} function will return an error when directives that conflict with prior directives are encountered.


%%%%%%%%%%%
\subsection{\code{PMIx_Finalize}}
\declareapi{PMIx_Finalize}

%%%%
\summary

Finalize the PMIx client library.

%%%%
\format

\cspecificstart
\begin{codepar}
pmix_status_t
PMIx_Finalize(const pmix_info_t info[], size_t ninfo)
\end{codepar}
\cspecificend

\begin{arglist}
\argin{info}{Array of \refattr{pmix_info_t} structures (array of handles)}
\argin{ninfo}{Number of element in the \refarg{info} array (\code{size_t})}
\end{arglist}

Returns \refconst{PMIX_SUCCESS} or a negative value corresponding to a PMIx error constant.

%%%%
\descr

Decrement the \ac{PMIx} client library reference count.
When the reference count reaches zero, the library will finalize the \ac{PMIx} client, closing the connection with the local \ac{PMIx} server and releasing all internally allocated memory.

By default, \refapi{PMIx_Finalize} will not include an internal barrier operation.
Users desiring a barrier as part of the finalize operation can request it by including the \refattr{PMIX_EMBED_BARRIER} attribute in the provided \refstruct{pmix_info_t} array.



%%%%%%%%%%%%%%%%%%%%%%%%%%%%%%%%%%%%%%%%%%%%%%
%%%%%%%%%%%%%%%%%%%%%%%%%%%%%%%%%%%%%%%%%%%%%%
\section{Tool}
\label{chap:api_init:tool}

Initialization and finalization routines for \ac{PMIx} tools.

%%%%%%%%%%%
\subsection{\code{PMIx_tool_init}}
\declareapi{PMIx_tool_init}

%%%%
\summary

Initialize the \ac{PMIx} library for a tool connection.

%%%%
\format

\cspecificstart
\begin{codepar}
pmix_status_t
PMIx_tool_init(pmix_proc_t *proc,
               pmix_info_t info[], size_t ninfo)
\end{codepar}
\cspecificend

\begin{arglist}
\arginout{proc}{\refstruct{pmix_proc_t} structure (handle)}
\argin{info}{Array of \refattr{pmix_info_t} structures (array of handles)}
\argin{ninfo}{Number of element in the \refarg{info} array (\code{size_t})}
\end{arglist}

Returns \refconst{PMIX_SUCCESS} or a negative value corresponding to a PMIx error constant.

%%%%
\descr

Initialize the \ac{PMIx} tool, returning the process identifier assigned to this tool in the provided \refstruct{pmix_proc_t} struct.

When called the \ac{PMIx} tool library will check for the required connection information of the local \ac{PMIx} server and will establish the connection.
If the information is not found, or the server connection fails, then an appropriate error constant will be returned.

If successful, the function will return \refconst{PMIX_SUCCESS} and will fill the provided structure (if provided) with the server-assigned namespace and rank of the tool.

Note that the \ac{PMIx} tool library is referenced counted, and so multiple calls to \refapi{PMIx_tool_init} are allowed.
Thus, one way to obtain the namespace and rank of the process is to simply call \refapi{PMIx_tool_init} with a non-NULL parameter.

The \refarg{info} array is used to pass user requests pertaining to the init and subsequent operations.
Passing a \code{NULL} value for the array pointer is supported if no directives are desired.


%%%%%%%%%%%
\subsection{\code{PMIx_tool_finalize}}
\declareapi{PMIx_tool_finalize}

%%%%
\summary

Finalize the \ac{PMIx} library for a tool connection.

%%%%
\format

\cspecificstart
\begin{codepar}
pmix_status_t
PMIx_tool_finalize(void)
\end{codepar}
\cspecificend

Returns \refconst{PMIX_SUCCESS} or a negative value corresponding to a PMIx error constant.

%%%%
\descr

Finalize the PMIx tool library, closing the connection to the local server.
An error code will be returned if, for some reason, the connection cannot be closed.


%%%%%%%%%%%%%%%%%%%%%%%%%%%%%%%%%%%%%%%%%%%%%%
%%%%%%%%%%%%%%%%%%%%%%%%%%%%%%%%%%%%%%%%%%%%%%
\section{Server}
\label{chap:api_init:server}

Initialization and finalization routines for \ac{PMIx} servers.

%%%%%%%%%%%
\subsection{\code{PMIx_server_init}}
\declareapi{PMIx_server_init}

%%%%
\summary

Initialize the \ac{PMIx} server.

%%%%
\format

\cspecificstart
\begin{codepar}
pmix_status_t
PMIx_server_init(pmix_server_module_t *module,
                 pmix_info_t info[], size_t ninfo)
\end{codepar}
\cspecificend

\begin{arglist}
\arginout{module}{\refstruct{pmix_server_module_t} structure (handle)}
\argin{info}{Array of \refattr{pmix_info_t} structures (array of handles)}
\argin{ninfo}{Number of elements in the \refarg{info} array (\code{size_t})}
\end{arglist}

Returns \refconst{PMIX_SUCCESS} or a negative value corresponding to a PMIx error constant.

%%%%
\descr

Initialize the server support library, and provide a pointer to a \refapi{pmix_server_module_t} structure containing the caller's callback functions.
The array of \refstruct{pmix_info_t} structs is used to pass additional info that may be required by the server when initializing.
For example, a user/group ID to set on the rendezvous file for the Unix Domain Socket.
It also may include the \refconst{PMIX_SERVER_TOOL_SUPPORT} key, thereby indicating that the daemon is willing to accept connection requests from tools.


%%%%%%%%%%%
\subsection{\code{PMIx_server_finalize}}
\declareapi{PMIx_server_finalize}

%%%%
\summary

Finalize the PMIx server library.

%%%%
\format

\cspecificstart
\begin{codepar}
pmix_status_t
PMIx_server_finalize(void)
\end{codepar}
\cspecificend

Returns \refconst{PMIX_SUCCESS} or a negative value corresponding to a PMIx error constant.

%%%%
\descr

Finalize the server support library.
If internal communication mechanism is in-use, the server will shut it down at this time.
All memory usage is released.

%%%%%%%%%%%%%%%%%%%%%%%%%%%%%%%%%%%%%%%%%%%%%%%%%
