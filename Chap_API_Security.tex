%%%%%%%%%%%%%%%%%%%%%%%%%%%%%%%%%%%%%%%%%%%%%%%%%
% Chapter: Security
%%%%%%%%%%%%%%%%%%%%%%%%%%%%%%%%%%%%%%%%%%%%%%%%%
\chapter{Security}
\label{chap:api_security}

Applications and tools often interact with each other in ways that require verification of the identity of the user making the request, and access to that user's relevant authorizations. This is particularly important when tools connect directly to a system-level \ac{PMIx} server that may be operating at a privileged level. A variety of system management software packages provide this service, but the lack of standardized interfaces makes portability problematic.

This section defines two \ac{PMIx} client-side \acp{API} for this purpose. These are most likely to be used by user-space applications/tools, but are not restricted to that realm.


%%%%%%%%%%%%%%%%%%%%%%%%%%%%%%%%%%%%%%%%%%%%%%
%%%%%%%%%%%%%%%%%%%%%%%%%%%%%%%%%%%%%%%%%%%%%%
\section{Obtaining Credentials}
\label{chap:api_security:obtain}

The \ac{API} for obtaining a credential is defined as a non-blocking operation since the host environment may have to contact a remote credential service. The definition takes into account the potential that the returned credential could be sent via some mechanism to another application that resides in an environment using a different security mechanism. Thus, provision is made for the system to return additional information (e.g., the identity of the issuing agent) outside of the credential itself and visible to the application.

%%%%%%%%%%%
\subsection{\code{PMIx_Get_credential}}
\declareapi{PMIx_Get_credential}

%%%%
\summary

Request a credential from the \ac{PMIx} server library or the host environment

%%%%
\format

\versionMarker{3.0}
\cspecificstart
\begin{codepar}
pmix_status_t PMIx_Get_credential(const pmix_info_t info[], size_t ninfo,
                             pmix_credential_cbfunc_t cbfunc, void *cbdata)
\end{codepar}
\cspecificend

\begin{arglist}
\argin{info}{Array of \refattr{pmix_info_t} structures (array of handles)}
\argin{ninfo}{Number of elements in the \refarg{info} array (\code{size_t})}
\argin{cbfunc}{Callback function to return credential (function reference)}
\argin{cbdata}{Data to be passed to the callback function (memory reference)}
\end{arglist}

Returns one of the following:

\begin{itemize}
    \item \refconst{PMIX_SUCCESS}, indicating that the request has been communicated to the local \ac{PMIx} server - result will be returned in the provided \refarg{cbfunc}
    \item a \ac{PMIx} error constant indicating either an error in the input or that the request is unsupported - the \refarg{cbfunc} will \textit{not} be called
\end{itemize}

\reqattrstart
\ac{PMIx} libraries are not required to directly support any attributes for this function. However, if the request cannot be processed by the library itself, then any attributes provided by the client must be passed to the host environment for processing. In addition, the following attributes are required to be included in the passed \refarg{info} array:

\pastePRIAttributeItem{PMIX_USERID}
\pastePRIAttributeItem{PMIX_GRPID}

\reqattrend

\optattrstart
The following attributes are optional for host environments that support this operation:

\pasteAttributeItem{PMIX_TIMEOUT}

\optattrend

%%%%
\descr

Request a credential from the \ac{PMIx} server library or the host environment

%%%%%%%%%%%%%%%%%%%%%%%%%%%%%%%%%%%%%%%%%%%%%%
%%%%%%%%%%%%%%%%%%%%%%%%%%%%%%%%%%%%%%%%%%%%%%
\section{Validating Credentials}
\label{chap:api_security:validate}

The \ac{API} for validating a credential is again defined as a non-blocking operation since the host environment may have to contact a remote credential service. Provision is made for the system to return additional information regarding possible authorization limitations beyond simple authentication.

%%%%%%%%%%%
\subsection{\code{PMIx_Validate_credential}}
\declareapi{PMIx_Validate_credential}

%%%%
\summary

Request validation of a credential by the \ac{PMIx} server library or the host environment

%%%%
\format

\versionMarker{3.0}
\cspecificstart
\begin{codepar}
pmix_status_t PMIx_Validate_credential(const pmix_byte_object_t *cred,
                             const pmix_info_t info[], size_t ninfo,
                             pmix_validation_cbfunc_t cbfunc, void *cbdata)
\end{codepar}
\cspecificend

\begin{arglist}
\argin{cred}{Pointer to \refstruct{pmix_byte_object_t} containing the credential (handle)}
\argin{info}{Array of \refattr{pmix_info_t} structures (array of handles)}
\argin{ninfo}{Number of elements in the \refarg{info} array (\code{size_t})}
\argin{cbfunc}{Callback function to return result (function reference)}
\argin{cbdata}{Data to be passed to the callback function (memory reference)}
\end{arglist}

Returns one of the following:

\begin{itemize}
    \item \refconst{PMIX_SUCCESS}, indicating that the request has been communicated to the local \ac{PMIx} server - result will be returned in the provided \refarg{cbfunc}
    \item a \ac{PMIx} error constant indicating either an error in the input or that the request is unsupported - the \refarg{cbfunc} will \textit{not} be called
\end{itemize}

\reqattrstart
\ac{PMIx} libraries are not required to directly support any attributes for this function. However, if the request cannot be processed by the library itself, then any attributes provided by the client must be passed to the host environment for processing. In addition, the following attributes are required to be included in the passed \refarg{info} array:

\pastePRIAttributeItem{PMIX_USERID}
\pastePRIAttributeItem{PMIX_GRPID}

\reqattrend

\optattrstart
The following attributes are optional for host environments that support this operation:

\pasteAttributeItem{PMIX_TIMEOUT}

\optattrend

\adviceimplstart
We recommend that implementation of the \refattr{PMIX_TIMEOUT} attribute be left to the host environment due to race condition considerations between completion of the operation versus internal timeout in the \ac{PMIx} server library. Implementers that choose to support \refattr{PMIX_TIMEOUT} directly in the \ac{PMIx} server library must take care to resolve the race condition and should avoid passing \refattr{PMIX_TIMEOUT} to the host environment so that multiple competing timeouts are not created.
\adviceimplend


%%%%
\descr

Request validation of a credential by the \ac{PMIx} server library or the host environment



%%%%%%%%%%%%%%%%%%%%%%%%%%%%%%%%%%%%%%%%%%%%%%%%%
