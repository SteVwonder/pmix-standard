%%%%%%%%%%%%%%%%%%%%%%%%%%%%%%%%%%%%%%%%%%%%%%%%%
% Chapter: Process Management
%%%%%%%%%%%%%%%%%%%%%%%%%%%%%%%%%%%%%%%%%%%%%%%%%
\chapter{Process Management}
\label{chap:api_proc_mgmt}

This chapter defines functionality used by clients to create and destroy/abort processes in the \ac{PMIx} universe.

%%%%%%%%%%%%%%%%%%%%%%%%%%%%%%%%%%%%%%%%%%%%%%
%%%%%%%%%%%%%%%%%%%%%%%%%%%%%%%%%%%%%%%%%%%%%%
\section{Abort}
\label{chap:api_proc_mgmt:abort}

\ac{PMIx} provides a dedicated API by which an application can request that specified processes be aborted by the system.

%%%%%%%%%%%
\subsection{\code{PMIx_Abort}}
\declareapi{PMIx_Abort}

%%%%
\summary

Abort the specified processes

%%%%
\format

\versionMarker{1.0}
\cspecificstart
\begin{codepar}
pmix_status_t
PMIx_Abort(int status, const char msg[],
           pmix_proc_t procs[], size_t nprocs)
\end{codepar}
\cspecificend

\begin{arglist}
\argin{status}{Error code to return to invoking environment (integer)}
\argin{msg}{String message to be returned to user (string)}
\argin{procs}{Array of \refstruct{pmix_proc_t} structures (array of handles)}
\argin{nprocs}{Number of elements in the \refarg{procs} array (integer)}
\end{arglist}

Returns \refconst{PMIX_SUCCESS} or a negative value corresponding to a PMIx error constant.

%%%%
\descr

Request that the host resource manager print the provided message and abort the provided array of \refarg{procs}.
A Unix or POSIX environment should handle the provided status as a return error code from the main program that launched the application.
A \code{NULL} for the \refarg{procs} array indicates that all processes in the caller's namespace are to be aborted, including itself.
Passing a \code{NULL} \refarg{msg} parameter is allowed.

\adviceuserstart
The response to this request is somewhat dependent on the specific \acl{RM} and its configuration (e.g., some resource managers will not abort the application if the provided status is zero unless specifically configured to do so, and some cannot abort subsets of processes in an application), and thus lies outside the control of PMIx itself.
However, the PMIx client library shall inform the \ac{RM} of the request that the specified \refarg{procs} be aborted, regardless of the value of the provided status.

Note that race conditions caused by multiple processes calling \refapi{PMIx_Abort} are left to the server implementation to resolve with regard to which status is returned and what messages (if any) are printed.
\adviceuserend


%%%%%%%%%%%%%%%%%%%%%%%%%%%%%%%%%%%%%%%%%%%%%%
%%%%%%%%%%%%%%%%%%%%%%%%%%%%%%%%%%%%%%%%%%%%%%
\section{Process Creation}
\label{chap:api_proc_mgmt:spawn}

The \refapi{PMIx_Spawn} commands spawn new processes and/or applications in the \ac{PMIx} universe. This may include requests to extend the existing resource allocation or obtain a new one, depending upon provided and supported attributes.

%%%%%%%%%%%
\subsection{\code{PMIx_Spawn}}
\declareapi{PMIx_Spawn}

%%%%
\summary

Spawn a new job.

%%%%
\format

\versionMarker{1.0}
\cspecificstart
\begin{codepar}
pmix_status_t
PMIx_Spawn(const pmix_info_t job_info[], size_t ninfo,
           const pmix_app_t apps[], size_t napps,
           char nspace[])
\end{codepar}
\cspecificend

\begin{arglist}
\argin{job_info}{Array of info structures (array of handles)}
\argin{ninfo}{Number of elements in the \refarg{job_info} array (integer)}
\argin{apps}{Array of \refstruct{pmix_app_t} structures (array of handles)}
\argin{napps}{Number of elements in the \refarg{apps} array (integer)}
\argout{nspace}{Namespace of the new job (string)}
\end{arglist}

Returns \refconst{PMIX_SUCCESS} or a negative value corresponding to a PMIx error constant.

\reqattrstart
\ac{PMIx} libraries are not required to directly support any attributes for this function. However, any provided attributes must be passed to the host \ac{SMS} daemon for processing, and the \ac{PMIx} library is required to add the following attributes to those provided before passing the request to the host:

\pastePRIAttributeItem{PMIX_SPAWNED}
\pastePRIAttributeItem{PMIX_PARENT_ID}
\pastePRIAttributeItem{PMIX_REQUESTOR_IS_CLIENT}
\pastePRIAttributeItem{PMIX_REQUESTOR_IS_TOOL}

Host environments that implement support for \refapi{PMIx_Spawn} are required to pass the \refattr{PMIX_SPAWNED} and \refattr{PMIX_PARENT_ID} attributes to all \ac{PMIx} servers launching new child processes so those values can be returned to clients upon connection to the \ac{PMIx} server. In addition, they are required to support the following attributes when present in either the \refarg{job_info} or the \textit{info} array of an element of the \refarg{apps} array:

\pastePRRTEAttributeItem{PMIX_WDIR}
\pastePRRTEAttributeItem{PMIX_SET_SESSION_CWD}
\pastePRRTEAttributeItem{PMIX_PREFIX}
\pastePRRTEAttributeItem{PMIX_HOST}
\pastePRRTEAttributeItem{PMIX_HOSTFILE}

\reqattrend

\optattrstart
The following attributes are optional for host environments that support this operation:

\pastePRRTEAttributeItem{PMIX_ADD_HOSTFILE}
\pastePRRTEAttributeItem{PMIX_ADD_HOST}
\pastePRRTEAttributeItem{PMIX_PRELOAD_BIN}
\pastePRRTEAttributeItem{PMIX_PRELOAD_FILES}
\pastePRRTEAttributeItem{PMIX_PERSONALITY}
\pastePRRTEAttributeItem{PMIX_MAPPER}
\pastePRRTEAttributeItem{PMIX_DISPLAY_MAP}
\pastePRRTEAttributeItem{PMIX_PPR}
\pastePRRTEAttributeItem{PMIX_MAPBY}
\pastePRRTEAttributeItem{PMIX_RANKBY}
\pastePRRTEAttributeItem{PMIX_BINDTO}
\pastePRRTEAttributeItem{PMIX_NON_PMI}
\pastePRRTEAttributeItem{PMIX_STDIN_TGT}
\pastePRRTEAttributeItem{PMIX_FWD_STDIN}
\pastePRRTEAttributeItem{PMIX_FWD_STDOUT}
\pastePRRTEAttributeItem{PMIX_FWD_STDERR}
\pastePRRTEAttributeItem{PMIX_DEBUGGER_DAEMONS}
\pastePRRTEAttributeItem{PMIX_TAG_OUTPUT}
\pastePRRTEAttributeItem{PMIX_TIMESTAMP_OUTPUT}
\pastePRRTEAttributeItem{PMIX_MERGE_STDERR_STDOUT}
\pastePRRTEAttributeItem{PMIX_OUTPUT_TO_FILE}
\pastePRRTEAttributeItem{PMIX_INDEX_ARGV}
\pastePRRTEAttributeItem{PMIX_CPUS_PER_PROC}
\pastePRRTEAttributeItem{PMIX_NO_PROCS_ON_HEAD}
\pastePRRTEAttributeItem{PMIX_NO_OVERSUBSCRIBE}
\pastePRRTEAttributeItem{PMIX_REPORT_BINDINGS}
\pastePRRTEAttributeItem{PMIX_CPU_LIST}
\pastePRRTEAttributeItem{PMIX_JOB_RECOVERABLE}
\pastePRRTEAttributeItem{PMIX_JOB_CONTINUOUS}
\pastePRRTEAttributeItem{PMIX_MAX_RESTARTS}

\optattrend

%%%%
\descr

Spawn a new job.
The assigned namespace of the spawned applications is returned in the \refarg{nspace} parameter.
A \code{NULL} value in that location indicates that the caller doesn't wish to have the namespace returned.
The \refarg{nspace} array must be at least of size one more than \refconst{PMIX_MAX_NSLEN}.

By default, the spawned processes will be PMIx ``connected'' to the parent process upon successful launch (see \refapi{PMIx_Connect} description for details).
Note that this only means that (a) the parent process will be given a copy of the new job's
information so it can query job-level info without incurring any communication penalties, (b) newly spawned child processes will receive a copy of the parent processes job-level info, and (c) both the parent process and members of the child job will receive notification of errors from processes in their combined assemblage.

\adviceuserstart
Behavior of individual resource managers may differ, but it is expected that failure of any application process to start will result in termination/cleanup of \emph{all} processes in the newly spawned job and return of an error code to the caller.
\adviceuserend

%%%%%%%%%%%
\subsection{\code{PMIx_Spawn_nb}}
\declareapi{PMIx_Spawn_nb}

%%%%
\summary

Nonblocking version of the \refapi{PMIx_Spawn} routine.

%%%%
\format

\versionMarker{1.0}
\cspecificstart
\begin{codepar}
pmix_status_t
PMIx_Spawn_nb(const pmix_info_t job_info[], size_t ninfo,
              const pmix_app_t apps[], size_t napps,
              pmix_spawn_cbfunc_t cbfunc, void *cbdata)
\end{codepar}
\cspecificend

\begin{arglist}
\argin{job_info}{Array of info structures (array of handles)}
\argin{ninfo}{Number of elements in the \refarg{job_info} array (integer)}
\argin{apps}{Array of \refstruct{pmix_app_t} structures (array of handles)}
\argin{cbfunc}{Callback function \refapi{pmix_spawn_cbfunc_t} (function reference)}
\argin{cbdata}{Data to be passed to the callback function (memory reference)}
\end{arglist}

Returns \refconst{PMIX_SUCCESS} or a negative value corresponding to a PMIx error constant.

\reqattrstart
\ac{PMIx} libraries are not required to directly support any attributes for this function. However, any provided attributes must be passed to the host \ac{SMS} daemon for processing, and the \ac{PMIx} library is required to add the following attributes to those provided before passing the request to the host:

\pastePRIAttributeItem{PMIX_SPAWNED}
\pastePRIAttributeItem{PMIX_PARENT_ID}
\pastePRIAttributeItem{PMIX_REQUESTOR_IS_CLIENT}
\pastePRIAttributeItem{PMIX_REQUESTOR_IS_TOOL}

Host environments that implement support for \refapi{PMIx_Spawn} are required to pass the \refattr{PMIX_SPAWNED} and \refattr{PMIX_PARENT_ID} attributes to all \ac{PMIx} servers launching new child processes so those values can be returned to clients upon connection to the \ac{PMIx} server. In addition, they are required to support the following attributes when present in either the \refarg{job_info} or the \textit{info} array of an element of the \refarg{apps} array:

\pastePRRTEAttributeItem{PMIX_WDIR}
\pastePRRTEAttributeItem{PMIX_SET_SESSION_CWD}
\pastePRRTEAttributeItem{PMIX_PREFIX}
\pastePRRTEAttributeItem{PMIX_HOST}
\pastePRRTEAttributeItem{PMIX_HOSTFILE}

\reqattrend

\optattrstart
The following attributes are optional for host environments that support this operation:

\pastePRRTEAttributeItem{PMIX_ADD_HOSTFILE}
\pastePRRTEAttributeItem{PMIX_ADD_HOST}
\pastePRRTEAttributeItem{PMIX_PRELOAD_BIN}
\pastePRRTEAttributeItem{PMIX_PRELOAD_FILES}
\pastePRRTEAttributeItem{PMIX_PERSONALITY}
\pastePRRTEAttributeItem{PMIX_MAPPER}
\pastePRRTEAttributeItem{PMIX_DISPLAY_MAP}
\pastePRRTEAttributeItem{PMIX_PPR}
\pastePRRTEAttributeItem{PMIX_MAPBY}
\pastePRRTEAttributeItem{PMIX_RANKBY}
\pastePRRTEAttributeItem{PMIX_BINDTO}
\pastePRRTEAttributeItem{PMIX_NON_PMI}
\pastePRRTEAttributeItem{PMIX_STDIN_TGT}
\pastePRRTEAttributeItem{PMIX_FWD_STDIN}
\pastePRRTEAttributeItem{PMIX_FWD_STDOUT}
\pastePRRTEAttributeItem{PMIX_FWD_STDERR}
\pastePRRTEAttributeItem{PMIX_DEBUGGER_DAEMONS}
\pastePRRTEAttributeItem{PMIX_TAG_OUTPUT}
\pastePRRTEAttributeItem{PMIX_TIMESTAMP_OUTPUT}
\pastePRRTEAttributeItem{PMIX_MERGE_STDERR_STDOUT}
\pastePRRTEAttributeItem{PMIX_OUTPUT_TO_FILE}
\pastePRRTEAttributeItem{PMIX_INDEX_ARGV}
\pastePRRTEAttributeItem{PMIX_CPUS_PER_PROC}
\pastePRRTEAttributeItem{PMIX_NO_PROCS_ON_HEAD}
\pastePRRTEAttributeItem{PMIX_NO_OVERSUBSCRIBE}
\pastePRRTEAttributeItem{PMIX_REPORT_BINDINGS}
\pastePRRTEAttributeItem{PMIX_CPU_LIST}
\pastePRRTEAttributeItem{PMIX_JOB_RECOVERABLE}
\pastePRRTEAttributeItem{PMIX_JOB_CONTINUOUS}
\pastePRRTEAttributeItem{PMIX_MAX_RESTARTS}

\optattrend

%%%%
\descr

Nonblocking version of the \refapi{PMIx_Spawn} routine. The provided callback function will be executed upon successful start of \textit{all} specified application processes.

\adviceuserstart
Behavior of individual resource managers may differ, but it is expected that failure of any application process to start will result in termination/cleanup of \emph{all} processes in the newly spawned job and return of an error code to the caller.
\adviceuserend



%%%%%%%%%%%%%%%%%%%%%%%%%%%%%%%%%%%%%%%%%%%%%%
%%%%%%%%%%%%%%%%%%%%%%%%%%%%%%%%%%%%%%%%%%%%%%
\section{Connecting and Disconnecting Processes}
\label{chap:api_proc_mgmt:connect}

This section defines functions to connect and disconnect processes in two or more separate \ac{PMIx} namespaces. The \ac{PMIx} definition of \textit{connected} solely implies the following:

\begin{itemize}
    \item job-level information for each namespace involved in the operation is to be made available to all processes in the connected assemblage
    \item any data posted by a process in the connected assemblage (via calls to \refapi{PMIx_Put} committed via \refapi{PMIx_Commit}) prior to execution of the \refapi{PMIx_Connect} operation is to be made accessible to all processes in the assemblage - any data posted after execution of the \textit{connect} operation must be exchanged via a separate \refapi{PMIx_Fence} operation spanning the connected processes
    \item the host environment should treat the failure of any process in the assemblage as a reportable event, taking action on the assemblage as if it were a single application. For example, if the environment defaults (in the absence of any application directives) to terminating an application upon failure of any process in that application, then the environment should terminate all processes in the connected assemblage upon failure of any member.
\end{itemize}

\advicermstart
The host environment may choose to assign a new namespace to the connected assemblage and/or assign new ranks for its members for its own internal tracking purposes. However, it is not required to communicate such assignments to the participants (e.g., in response to an appropriate call to \refapi{PMIx_Query_info_nb}). The host environment is required to generate a \refconst{PMIX_ERR_INVALID_TERMINATION} event should any process in the assemblage terminate or call \refapi{PMIx_Finalize} without first \textit{disconnecting} from the assemblage.
\advicermend

\adviceuserstart
Attempting to \textit{connect} processes solely within the same namespace is essentially a \textit{no-op} operation. While not explicitly prohibited, users are advised that a \ac{PMIx} implementation or host environment may return an error in such cases.

Neither the \ac{PMIx} implementation nor host environment are required to provide any tracking support for the assemblage. Thus, the application is responsible for maintaining the membership list of the assemblage.

\adviceuserend

%%%%%%%%%%%
\subsection{\code{PMIx_Connect}}
\declareapi{PMIx_Connect}

%%%%
\summary

Connect namespaces.

%%%%
\format

\versionMarker{1.0}
\cspecificstart
\begin{codepar}
pmix_status_t
PMIx_Connect(const pmix_proc_t procs[], size_t nprocs,
             const pmix_info_t info[], size_t ninfo)
\end{codepar}
\cspecificend

\begin{arglist}
\argin{procs}{Array of proc structures (array of handles)}
\argin{nprocs}{Number of elements in the \refarg{procs} array (integer)}
\argin{info}{Array of info structures (array of handles)}
\argin{ninfo}{Number of elements in the \refarg{info} array (integer)}
\end{arglist}

Returns \refconst{PMIX_SUCCESS} or a negative value corresponding to a PMIx error constant.

\reqattrstart
\ac{PMIx} libraries are not required to directly support any attributes for this function. However, any provided attributes must be passed to the host \ac{SMS} daemon for processing.

\reqattrend

\optattrstart
The following attributes are optional for host environments that support this operation:

\pastePRRTEAttributeItem{PMIX_TIMEOUT}
\pasteAttributeItem{PMIX_COLLECTIVE_ALGO}
\pasteAttributeItem{PMIX_COLLECTIVE_ALGO_REQD}

\optattrend

\adviceimplstart
We recommend that implementation of the \refattr{PMIX_TIMEOUT} attribute be left to the host environment due to race condition considerations between completion of the operation versus internal timeout in the \ac{PMIx} server library. Implementers that choose to support \refattr{PMIX_TIMEOUT} directly in the \ac{PMIx} server library must take care to resolve the race condition and should avoid passing \refattr{PMIX_TIMEOUT} to the host environment so that multiple competing timeouts are not created.
\adviceimplend

%%%%
\descr

Record the processes specified by the \refarg{procs} array as \textit{connected} as per the \ac{PMIx} definition. The function will return once all processes identified in \refarg{procs} have called either \refapi{PMIx_Connect} or its non-blocking version, \textit{and} the host environment has completed any supporting operations required to meet the terms of the \ac{PMIx} definition of \textit{connected} processes.

\adviceuserstart
All processes engaged in a given \refapi{PMIx_Connect} operation must provide the identical \refarg{procs} array as ordering of entries in the array and the method by which those processes are identified (e.g., use of \refconst{PMIX_RANK_WILDCARD} versus listing the individual processes) \textit{may} impact the host environment's algorithm for uniquely identifying an operation.
\adviceuserend

Processes that combine via \refapi{PMIx_Connect} must call \refapi{PMIx_Disconnect} prior to finalizing and/or terminating - any process in the assemblage failing to meet this requirement will cause a \refconst{PMIX_ERR_INVALID_TERMINATION} event to be generated.

A process can only engage in \emph{one} connect operation involving the identical \refarg{procs} array at a time.
However, a process \emph{can} be simultaneously engaged in multiple connect operations, each involving a different \refarg{procs} array.

As in the case of the \refapi{PMIx_Fence} operation, the \refarg{info} array can be used to pass user-level directives regarding the algorithm to be used for any collective operation involved in the operation, timeout constraints, and other options available from the host \ac{RM}.


%%%%%%%%%%%
\subsection{\code{PMIx_Connect_nb}}
\declareapi{PMIx_Connect_nb}

%%%%
\summary

Nonblocking \refapi{PMIx_Connect_nb} routine.

%%%%
\format

\versionMarker{1.0}
\cspecificstart
\begin{codepar}
pmix_status_t
PMIx_Connect_nb(const pmix_proc_t procs[], size_t nprocs,
                const pmix_info_t info[], size_t ninfo,
                pmix_op_cbfunc_t cbfunc, void *cbdata)
\end{codepar}
\cspecificend

\begin{arglist}
\argin{procs}{Array of proc structures (array of handles)}
\argin{nprocs}{Number of elements in the \refarg{procs} array (integer)}
\argin{info}{Array of info structures (array of handles)}
\argin{ninfo}{Number of element in the \refarg{info} array (integer)}
\argin{cbfunc}{Callback function \refapi{pmix_op_cbfunc_t} (function reference)}
\argin{cbdata}{Data to be passed to the callback function (memory reference)}
\end{arglist}

Returns \refconst{PMIX_SUCCESS} or a negative value corresponding to a PMIx error constant.

\reqattrstart
\ac{PMIx} libraries are not required to directly support any attributes for this function. However, any provided attributes must be passed to the host \ac{SMS} daemon for processing.

\reqattrend

\optattrstart
The following attributes are optional for host environments that support this operation:

\pastePRRTEAttributeItem{PMIX_TIMEOUT}
\pasteAttributeItem{PMIX_COLLECTIVE_ALGO}
\pasteAttributeItem{PMIX_COLLECTIVE_ALGO_REQD}

\optattrend

\adviceimplstart
We recommend that implementation of the \refattr{PMIX_TIMEOUT} attribute be left to the host environment due to race condition considerations between completion of the operation versus internal timeout in the \ac{PMIx} server library. Implementers that choose to support \refattr{PMIX_TIMEOUT} directly in the \ac{PMIx} server library must take care to resolve the race condition and should avoid passing \refattr{PMIX_TIMEOUT} to the host environment so that multiple competing timeouts are not created.
\adviceimplend

%%%%
\descr

Nonblocking version of \refapi{PMIx_Connect}. The callback function is called once all processes identified in \refarg{procs} have called either \refapi{PMIx_Connect} or its non-blocking version, \textit{and} the host environment has completed any supporting operations required to meet the terms of the \ac{PMIx} definition of \textit{connected} processes.


%%%%%%%%%%%
\subsection{\code{PMIx_Disconnect}}
\declareapi{PMIx_Disconnect}

%%%%
\summary

Disconnect a previously connected set of processes.

%%%%
\format

\versionMarker{1.0}
\cspecificstart
\begin{codepar}
pmix_status_t
PMIx_Disconnect(const pmix_proc_t procs[], size_t nprocs,
                const pmix_info_t info[], size_t ninfo);
\end{codepar}
\cspecificend

\begin{arglist}
\argin{procs}{Array of proc structures (array of handles)}
\argin{nprocs}{Number of elements in the \refarg{procs} array (integer)}
\argin{info}{Array of info structures (array of handles)}
\argin{ninfo}{Number of element in the \refarg{info} array (integer)}
\end{arglist}

Returns \refconst{PMIX_SUCCESS} or a negative value corresponding to a PMIx error constant.

\reqattrstart
\ac{PMIx} libraries are not required to directly support any attributes for this function. However, any provided attributes must be passed to the host \ac{SMS} daemon for processing.

\reqattrend

\optattrstart
The following attributes are optional for host environments that support this operation:

\pastePRRTEAttributeItem{PMIX_TIMEOUT}

\optattrend

\adviceimplstart
We recommend that implementation of the \refattr{PMIX_TIMEOUT} attribute be left to the host environment due to race condition considerations between completion of the operation versus internal timeout in the \ac{PMIx} server library. Implementers that choose to support \refattr{PMIX_TIMEOUT} directly in the \ac{PMIx} server library must take care to resolve the race condition and should avoid passing \refattr{PMIX_TIMEOUT} to the host environment so that multiple competing timeouts are not created.
\adviceimplend


%%%%
\descr

Disconnect a previously connected set of processes.
A \refconst{PMIX_ERR_INVALID_OPERATION} error will be returned if the specified set of \refarg{procs} was not previously \textit{connected} via a call to \refapi{PMIx_Connect} or its non-blocking form. The function will return once all processes identified in \refarg{procs} have called either \refapi{PMIx_Disconnect} or its non-blocking version, \textit{and} the host environment has completed any required supporting operations.

\adviceuserstart
All processes engaged in a given \refapi{PMIx_Disconnect} operation must provide the identical \refarg{procs} array as ordering of entries in the array and the method by which those processes are identified (e.g., use of \refconst{PMIX_RANK_WILDCARD} versus listing the individual processes) \textit{may} impact the host environment's algorithm for uniquely identifying an operation.
\adviceuserend

A process can only engage in \emph{one} disconnect operation involving the identical \refarg{procs} array at a time.
However, a process \emph{can} be simultaneously engaged in multiple disconnect operations, each involving a different \refarg{procs} array.

As in the case of the \refapi{PMIx_Fence} operation, the \refarg{info} array can be used to pass user-level directives regarding the algorithm to be used for any collective operation involved in the operation, timeout constraints, and other options available from the host \ac{RM}.


%%%%%%%%%%%
\subsection{\code{PMIx_Disconnect_nb}}
\declareapi{PMIx_Disconnect_nb}

%%%%
\summary

Nonblocking \refapi{PMIx_Disconnect} routine.

%%%%
\format

\versionMarker{1.0}
\cspecificstart
\begin{codepar}
pmix_status_t
PMIx_Disconnect_nb(const pmix_proc_t procs[], size_t nprocs,
                   const pmix_info_t info[], size_t ninfo,
                   pmix_op_cbfunc_t cbfunc, void *cbdata);
\end{codepar}
\cspecificend

\begin{arglist}
\argin{procs}{Array of proc structures (array of handles)}
\argin{nprocs}{Number of elements in the \refarg{procs} array (integer)}
\argin{info}{Array of info structures (array of handles)}
\argin{ninfo}{Number of element in the \refarg{info} array (integer)}
\argin{cbfunc}{Callback function \refapi{pmix_op_cbfunc_t} (function reference)}
\argin{cbdata}{Data to be passed to the callback function (memory reference)}
\end{arglist}

Returns \refconst{PMIX_SUCCESS} or a negative value corresponding to a PMIx error constant.

\reqattrstart
\ac{PMIx} libraries are not required to directly support any attributes for this function. However, any provided attributes must be passed to the host \ac{SMS} daemon for processing.

\reqattrend

\optattrstart
The following attributes are optional for host environments that support this operation:

\pastePRRTEAttributeItem{PMIX_TIMEOUT}

\optattrend

\adviceimplstart
We recommend that implementation of the \refattr{PMIX_TIMEOUT} attribute be left to the host environment due to race condition considerations between completion of the operation versus internal timeout in the \ac{PMIx} server library. Implementers that choose to support \refattr{PMIX_TIMEOUT} directly in the \ac{PMIx} server library must take care to resolve the race condition and should avoid passing \refattr{PMIX_TIMEOUT} to the host environment so that multiple competing timeouts are not created.
\adviceimplend

%%%%
\descr

Nonblocking \refapi{PMIx_Disconnect} routine. The callback function is called once all processes identified in \refarg{procs} have called either \refapi{PMIx_Disconnect_nb} or its blocking version, \textit{and} the host environment has completed any required supporting operations.


%%%%%%%%%%%%%%%%%%%%%%%%%%%%%%%%%%%%%%%%%%%%%%%%%
