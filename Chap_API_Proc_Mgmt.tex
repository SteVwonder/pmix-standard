%%%%%%%%%%%%%%%%%%%%%%%%%%%%%%%%%%%%%%%%%%%%%%%%%
% Chapter: Process Management
%%%%%%%%%%%%%%%%%%%%%%%%%%%%%%%%%%%%%%%%%%%%%%%%%
\chapter{Process Management}
\label{chap:api_proc_mgmt}

This chapter defines functionality used by clients to create and destroy/abort processes in the \ac{PMIx} universe.

%%%%%%%%%%%%%%%%%%%%%%%%%%%%%%%%%%%%%%%%%%%%%%
%%%%%%%%%%%%%%%%%%%%%%%%%%%%%%%%%%%%%%%%%%%%%%
\section{Abort}
\label{chap:api_proc_mgmt:abort}

\ac{PMIx} provides a dedicated API by which an application can request that specified processes be aborted by the system.

%%%%%%%%%%%
\subsection{\code{PMIx_Abort}}
\declareapi{PMIx_Abort}

%%%%
\summary

Abort the specified processes

%%%%
\format

\versionMarker{1.0}
\cspecificstart
\begin{codepar}
pmix_status_t
PMIx_Abort(int status, const char msg[],
           pmix_proc_t procs[], size_t nprocs)
\end{codepar}
\cspecificend

\begin{arglist}
\argin{status}{Error code to return to invoking environment (integer)}
\argin{msg}{String message to be returned to user (string)}
\argin{procs}{Array of \refstruct{pmix_proc_t} structures (array of handles)}
\argin{nprocs}{Number of elements in the \refarg{procs} array (integer)}
\end{arglist}

Returns \refconst{PMIX_SUCCESS} or a negative value corresponding to a PMIx error constant.

%%%%
\descr

Request that the host resource manager print the provided message and abort the provided array of \refarg{procs}.
A Unix or POSIX environment should handle the provided status as a return error code from the main program that launched the application.
A \code{NULL} for the \refarg{procs} array indicates that all processes in the caller's namespace are to be aborted, including itself.
Passing a \code{NULL} \refarg{msg} parameter is allowed.

\adviceuserstart
The response to this request is somewhat dependent on the specific \acl{RM} and its configuration (e.g., some resource managers will not abort the application if the provided status is zero unless specifically configured to do so, and some cannot abort subsets of processes in an application), and thus lies outside the control of PMIx itself.
However, the PMIx client library shall inform the \ac{RM} of the request that the specified \refarg{procs} be aborted, regardless of the value of the provided status.

Note that race conditions caused by multiple processes calling \refapi{PMIx_Abort} are left to the server implementation to resolve with regard to which status is returned and what messages (if any) are printed.
\adviceuserend


%%%%%%%%%%%%%%%%%%%%%%%%%%%%%%%%%%%%%%%%%%%%%%
%%%%%%%%%%%%%%%%%%%%%%%%%%%%%%%%%%%%%%%%%%%%%%
\section{Process Creation}
\label{chap:api_proc_mgmt:spawn}

The \refapi{PMIx_Spawn} commands spawn new jobs in the \ac{PMIx} universe.
The \ac{PMIx} calls made by the client into the \ac{PRI} no not, itself, spawn processes.
Instead the request is passed by the \ac{PRI} client-side library to the \ac{PMIx}-enabled \ac{RM} which will then create the processes on the allocated resources.

%%%%%%%%%%%
\subsection{\code{PMIx_Spawn}}
\declareapi{PMIx_Spawn}

%%%%
\summary

Spawn a new job.

%%%%
\format

\versionMarker{1.0}
\cspecificstart
\begin{codepar}
pmix_status_t
PMIx_Spawn(const pmix_info_t job_info[], size_t ninfo,
           const pmix_app_t apps[], size_t napps,
           char nspace[])
\end{codepar}
\cspecificend

\begin{arglist}
\argin{job_info}{Array of info structures (array of handles)}
\argin{ninfo}{Number of elements in the \refarg{job_info} array (integer)}
\argin{apps}{Array of \refstruct{pmix_app_t} structures (array of handles)}
\argin{napps}{Number of elements in the \refarg{apps} array (integer)}
\argout{nspace}{Namespace of the new job (string)}
\end{arglist}

Returns \refconst{PMIX_SUCCESS} or a negative value corresponding to a PMIx error constant.

\priattr
The PMIx_Spawn function in the \ac{PRI} does not parse or utilize the attributes directly. However, any provided attributes are passed to the host \ac{SMS} daemon for processing. Note that the \ac{PRI} automatically adds the following attributes to those provided before passing the request to the host daemon:

\pasteAttributeItem{PMIX_SPAWNED}
\pasteAttributeItem{PMIX_PARENT_ID}
\pasteAttributeItem{PMIX_REQUESTOR_IS_CLIENT}
\pasteAttributeItem{PMIX_REQUESTOR_IS_TOOL}

The \refattr{PMIX_SPAWNED} and \refattr{PMIX_PARENT_ID} attributes are passed to the child processes upon connection to a PMIx server.

\reqattr
\acp{RM} that implement support for \refapi{PMIx_Spawn} are required to support the following attributes:

\pasteAttributeItem{PMIX_WDIR}
\pasteAttributeItem{PMIX_SET_SESSION_CWD}
\pasteAttributeItem{PMIX_PREFIX}
\pasteAttributeItem{PMIX_HOST}
\pasteAttributeItem{PMIX_HOSTFILE}

\optattr
A complete implementation would include support for the following attributes:

\pasteAttributeItem{PMIX_ADD_HOSTFILE}
\pasteAttributeItem{PMIX_ADD_HOST}
\pasteAttributeItem{PMIX_PRELOAD_BIN}
\pasteAttributeItem{PMIX_PRELOAD_FILES}
\pasteAttributeItem{PMIX_PERSONALITY}
\pasteAttributeItem{PMIX_MAPPER}
\pasteAttributeItem{PMIX_DISPLAY_MAP}
\pasteAttributeItem{PMIX_PPR}
\pasteAttributeItem{PMIX_MAPBY}
\pasteAttributeItem{PMIX_RANKBY}
\pasteAttributeItem{PMIX_BINDTO}
\pasteAttributeItem{PMIX_NON_PMI}
\pasteAttributeItem{PMIX_STDIN_TGT}
\pasteAttributeItem{PMIX_FWD_STDIN}
\pasteAttributeItem{PMIX_FWD_STDOUT}
\pasteAttributeItem{PMIX_FWD_STDERR}
\pasteAttributeItem{PMIX_DEBUGGER_DAEMONS}
\pasteAttributeItem{PMIX_TAG_OUTPUT}
\pasteAttributeItem{PMIX_TIMESTAMP_OUTPUT}
\pasteAttributeItem{PMIX_MERGE_STDERR_STDOUT}
\pasteAttributeItem{PMIX_OUTPUT_TO_FILE}
\pasteAttributeItem{PMIX_INDEX_ARGV}
\pasteAttributeItem{PMIX_CPUS_PER_PROC}
\pasteAttributeItem{PMIX_NO_PROCS_ON_HEAD}
\pasteAttributeItem{PMIX_NO_OVERSUBSCRIBE}
\pasteAttributeItem{PMIX_REPORT_BINDINGS}
\pasteAttributeItem{PMIX_CPU_LIST}
\pasteAttributeItem{PMIX_JOB_RECOVERABLE}
\pasteAttributeItem{PMIX_JOB_CONTINUOUS}
\pasteAttributeItem{PMIX_MAX_RESTARTS}

%%%%
\descr

Spawn a new job.
The assigned namespace of the spawned applications is returned in the \refarg{nspace} parameter.
A \code{NULL} value in that location indicates that the caller doesn't wish to have the namespace returned.
The \refarg{nspace} array must be at least of size one more than \refconst{PMIX_MAX_NSLEN}.

By default, the spawned processes will be PMIx ``connected'' to the parent process upon successful launch (see \refapi{PMIx_Connect} description for details).
Note that this only means that (a) the parent process will be given a copy of the new job's
information so it can query job-level info without incurring any communication penalties, (b) newly spawned child processes will receive a copy of the parent processes job-level info, and (c) both the parent process and members of the child job will receive notification of errors from processes in their combined assemblage.

\adviceuserstart
Behavior of individual resource managers may differ, but it is expected that failure of any application process to start will result in termination/cleanup of \emph{all} processes in the newly spawned job and return of an error code to the caller.
\adviceuserend

%%%%%%%%%%%
\subsection{\code{PMIx_Spawn_nb}}
\declareapi{PMIx_Spawn_nb}

%%%%
\summary

Nonblocking version of the \refapi{PMIx_Spawn} routine.

%%%%
\format

\versionMarker{1.0}
\cspecificstart
\begin{codepar}
pmix_status_t
PMIx_Spawn_nb(const pmix_info_t job_info[], size_t ninfo,
              const pmix_app_t apps[], size_t napps,
              pmix_spawn_cbfunc_t cbfunc, void *cbdata)
\end{codepar}
\cspecificend

\begin{arglist}
\argin{job_info}{Array of info structures (array of handles)}
\argin{ninfo}{Number of elements in the \refarg{job_info} array (integer)}
\argin{apps}{Array of \refstruct{pmix_app_t} structures (array of handles)}
\argin{cbfunc}{Callback function \refapi{pmix_spawn_cbfunc_t} (function reference)}
\argin{cbdata}{Data to be passed to the callback function (memory reference)}
\end{arglist}

Returns \refconst{PMIX_SUCCESS} or a negative value corresponding to a PMIx error constant.

\priattr
The PMIx_Spawn_nb function in the \ac{PRI} does not parse or utilize the attributes directly. However, any provided attributes are passed to the host \ac{SMS} daemon for processing. Note that the \ac{PRI} automatically adds the following attributes to those provided before passing the request to the host daemon:

\pasteAttributeItem{PMIX_SPAWNED}
\pasteAttributeItem{PMIX_PARENT_ID}
\pasteAttributeItem{PMIX_REQUESTOR_IS_CLIENT}
\pasteAttributeItem{PMIX_REQUESTOR_IS_TOOL}

The \refattr{PMIX_SPAWNED} and \refattr{PMIX_PARENT_ID} attributes are passed to the child processes upon connection to a PMIx server.

\reqattr
\acp{RM} that implement support for \refapi{PMIx_Spawn} are required to support the following attributes:

\pasteAttributeItem{PMIX_WDIR}
\pasteAttributeItem{PMIX_SET_SESSION_CWD}
\pasteAttributeItem{PMIX_PREFIX}
\pasteAttributeItem{PMIX_HOST}
\pasteAttributeItem{PMIX_HOSTFILE}

\optattr
A complete implementation would include support for the following attributes:

\pasteAttributeItem{PMIX_ADD_HOSTFILE}
\pasteAttributeItem{PMIX_ADD_HOST}
\pasteAttributeItem{PMIX_PRELOAD_BIN}
\pasteAttributeItem{PMIX_PRELOAD_FILES}
\pasteAttributeItem{PMIX_PERSONALITY}
\pasteAttributeItem{PMIX_MAPPER}
\pasteAttributeItem{PMIX_DISPLAY_MAP}
\pasteAttributeItem{PMIX_PPR}
\pasteAttributeItem{PMIX_MAPBY}
\pasteAttributeItem{PMIX_RANKBY}
\pasteAttributeItem{PMIX_BINDTO}
\pasteAttributeItem{PMIX_NON_PMI}
\pasteAttributeItem{PMIX_STDIN_TGT}
\pasteAttributeItem{PMIX_FWD_STDIN}
\pasteAttributeItem{PMIX_FWD_STDOUT}
\pasteAttributeItem{PMIX_FWD_STDERR}
\pasteAttributeItem{PMIX_DEBUGGER_DAEMONS}
\pasteAttributeItem{PMIX_TAG_OUTPUT}
\pasteAttributeItem{PMIX_TIMESTAMP_OUTPUT}
\pasteAttributeItem{PMIX_MERGE_STDERR_STDOUT}
\pasteAttributeItem{PMIX_OUTPUT_TO_FILE}
\pasteAttributeItem{PMIX_INDEX_ARGV}
\pasteAttributeItem{PMIX_CPUS_PER_PROC}
\pasteAttributeItem{PMIX_NO_PROCS_ON_HEAD}
\pasteAttributeItem{PMIX_NO_OVERSUBSCRIBE}
\pasteAttributeItem{PMIX_REPORT_BINDINGS}
\pasteAttributeItem{PMIX_CPU_LIST}
\pasteAttributeItem{PMIX_JOB_RECOVERABLE}
\pasteAttributeItem{PMIX_JOB_CONTINUOUS}
\pasteAttributeItem{PMIX_MAX_RESTARTS}

%%%%
\descr

Nonblocking version of the \refapi{PMIx_Spawn} routine.


%%%%%%%%%%%%%%%%%%%%%%%%%%%%%%%%%%%%%%%%%%%%%%
%%%%%%%%%%%%%%%%%%%%%%%%%%%%%%%%%%%%%%%%%%%%%%
\section{Connecting and Disconnecting Processes}
\label{chap:api_proc_mgmt:connect}

This section defines functions to connect and disconnect separate \ac{PMIx} namespaces so that they may exchange information and otherwise communicate with each other.

%%%%%%%%%%%
\subsection{\code{PMIx_Connect}}
\declareapi{PMIx_Connect}

%%%%
\summary

Connect namespaces.

%%%%
\format

\versionMarker{1.0}
\cspecificstart
\begin{codepar}
pmix_status_t
PMIx_Connect(const pmix_proc_t procs[], size_t nprocs,
             const pmix_info_t info[], size_t ninfo)
\end{codepar}
\cspecificend

\begin{arglist}
\argin{procs}{Array of proc structures (array of handles)}
\argin{nprocs}{Number of elements in the \refarg{procs} array (integer)}
\argin{info}{Array of info structures (array of handles)}
\argin{ninfo}{Number of elements in the \refarg{info} array (integer)}
\end{arglist}

Returns \refconst{PMIX_SUCCESS} or a negative value corresponding to a PMIx error constant.

\priattr
The PMIx_Connect function in the \ac{PRI} does not parse or utilize the attributes directly. However, any provided attributes are passed to the host \ac{SMS} daemon for processing.

\optattr
A complete implementation would include support for the following attributes:

\pasteAttributeItem{PMIX_TIMEOUT}
\pasteAttributeItem{PMIX_COLLECTIVE_ALGO}
\pasteAttributeItem{PMIX_COLLECTIVE_ALGO_REQD}

%%%%
\descr

Record the specified processes as ``connected''.
This means that the resource manager should treat the failure of any process in the specified collection as a reportable event, and take appropriate action.
Note that different resource managers may respond to failures in different manners. The RM does not define a new identifier for the connected assemblage, nor does it define a new rank for each process within that group. In addition, the \ac{PMIx} server does not provide any tracking support for the assemblage. Thus, the caller is responsible for maintaining the membership list of the assemblage.

The function will return once all participating processes have called either \refapi{PMIx_Connect} or its non-blocking version.
The server is required to return any job-level info for the connecting processes that might not already have it (i.e., if the connect request involves \refarg{procs} from different namespaces, then each \refarg{proc} shall receive the job-level info from those namespaces other than their own). In addition, all members of the collection will receive notification of errors from processes in their combined assemblage. Processes that combine via \refapi{PMIx_Connect} must call \refapi{PMIx_Disconnect} prior to finalizing and/or terminating.

A process can only engage in \emph{one} connect operation involving the identical set of processes at a time.
However, a process \emph{can} be simultaneously engaged in multiple connect operations, each involving a different set of processes.

As in the case of the fence operation, the info array can be used to pass user-level directives regarding the algorithm to be used for the collective operation involved in the ``connect'', timeout constraints, and other options available from the host \ac{RM}.


%%%%%%%%%%%
\subsection{\code{PMIx_Connect_nb}}
\declareapi{PMIx_Connect_nb}

%%%%
\summary

Nonblocking \refapi{PMIx_Connect_nb} routine.

%%%%
\format

\versionMarker{1.0}
\cspecificstart
\begin{codepar}
pmix_status_t
PMIx_Connect_nb(const pmix_proc_t procs[], size_t nprocs,
                const pmix_info_t info[], size_t ninfo,
                pmix_op_cbfunc_t cbfunc, void *cbdata)
\end{codepar}
\cspecificend

\begin{arglist}
\argin{procs}{Array of proc structures (array of handles)}
\argin{nprocs}{Number of elements in the \refarg{procs} array (integer)}
\argin{info}{Array of info structures (array of handles)}
\argin{ninfo}{Number of element in the \refarg{info} array (integer)}
\argin{cbfunc}{Callback function \refapi{pmix_op_cbfunc_t} (function reference)}
\argin{cbdata}{Data to be passed to the callback function (memory reference)}
\end{arglist}

Returns \refconst{PMIX_SUCCESS} or a negative value corresponding to a PMIx error constant.

\priattr
The PMIx_Connect_nb function in the \ac{PRI} does not parse or utilize the attributes directly. However, any provided attributes are passed to the host \ac{SMS} daemon for processing.

\optattr
A complete implementation would include support for the following attributes:

\pasteAttributeItem{PMIX_TIMEOUT}
\pasteAttributeItem{PMIX_COLLECTIVE_ALGO}
\pasteAttributeItem{PMIX_COLLECTIVE_ALGO_REQD}

%%%%
\descr

Nonblocking version of \refapi{PMIx_Connect}. The callback function is called once all participating processes have called either \refapi{PMIx_Connect_nb} or its blocking version.


%%%%%%%%%%%
\subsection{\code{PMIx_Disconnect}}
\declareapi{PMIx_Disconnect}

%%%%
\summary

Disconnect a previously connected set of processes.

%%%%
\format

\versionMarker{1.0}
\cspecificstart
\begin{codepar}
pmix_status_t
PMIx_Disconnect(const pmix_proc_t procs[], size_t nprocs,
                const pmix_info_t info[], size_t ninfo);
\end{codepar}
\cspecificend

\begin{arglist}
\argin{procs}{Array of proc structures (array of handles)}
\argin{nprocs}{Number of elements in the \refarg{procs} array (integer)}
\argin{info}{Array of info structures (array of handles)}
\argin{ninfo}{Number of element in the \refarg{info} array (integer)}
\end{arglist}

Returns \refconst{PMIX_SUCCESS} or a negative value corresponding to a PMIx error constant.

\priattr
The PMIx_Disonnect function in the \ac{PRI} does not parse or utilize the attributes directly. However, any provided attributes are passed to the host \ac{SMS} daemon for processing.

\optattr
A complete implementation would include support for the following attributes:

\pasteAttributeItem{PMIX_TIMEOUT}

%%%%
\descr

Disconnect a previously connected set of processes.
An error will be returned if the specified set of \refarg{procs} was not previously ``connected''.
As with \refapi{PMIx_Connect}, a process may be involved in multiple simultaneous disconnect operations.
However, a process is not allowed to reconnect to a set of \refarg{procs} that has not fully completed disconnect (i.e., you have to fully disconnect before you can reconnect to the \emph{same} group of processes).
The \refarg{info} array is used as in \refapi{PMIx_Connect}.


%%%%%%%%%%%
\subsection{\code{PMIx_Disconnect_nb}}
\declareapi{PMIx_Disconnect_nb}

%%%%
\summary

Nonblocking \refapi{PMIx_Disconnect} routine.

%%%%
\format

\versionMarker{1.0}
\cspecificstart
\begin{codepar}
pmix_status_t
PMIx_Disconnect_nb(const pmix_proc_t ranges[], size_t nprocs,
                   const pmix_info_t info[], size_t ninfo,
                   pmix_op_cbfunc_t cbfunc, void *cbdata);
\end{codepar}
\cspecificend

\begin{arglist}
\argin{procs}{Array of proc structures (array of handles)}
\argin{nprocs}{Number of elements in the \refarg{procs} array (integer)}
\argin{info}{Array of info structures (array of handles)}
\argin{ninfo}{Number of element in the \refarg{info} array (integer)}
\argin{cbfunc}{Callback function \refapi{pmix_op_cbfunc_t} (function reference)}
\argin{cbdata}{Data to be passed to the callback function (memory reference)}
\end{arglist}

Returns \refconst{PMIX_SUCCESS} or a negative value corresponding to a PMIx error constant.

\priattr
The PMIx_Disonnect_nb function in the \ac{PRI} does not parse or utilize the attributes directly. However, any provided attributes are passed to the host \ac{SMS} daemon for processing.

\optattr
A complete implementation would include support for the following attributes:

\pasteAttributeItem{PMIX_TIMEOUT}

%%%%
\descr

Nonblocking \refapi{PMIx_Disconnect} routine.


%%%%%%%%%%%%%%%%%%%%%%%%%%%%%%%%%%%%%%%%%%%%%%%%%
