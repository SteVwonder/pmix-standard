%%%%%%%%%%%%%%%%%%%%%%%%%%%%%%%%%%%%%%%%%%%%%%%%%
% Chapter: Process Management
%%%%%%%%%%%%%%%%%%%%%%%%%%%%%%%%%%%%%%%%%%%%%%%%%
\chapter{Process Management}
\label{chap:api_proc_mgmt}

This chapter defines functionality used by clients to create and destroy/abort processes in the \ac{PMIx} universe.

%%%%%%%%%%%%%%%%%%%%%%%%%%%%%%%%%%%%%%%%%%%%%%
%%%%%%%%%%%%%%%%%%%%%%%%%%%%%%%%%%%%%%%%%%%%%%
\section{Abort}
\label{chap:api_proc_mgmt:abort}

\ac{PMIx} provides a dedicated API by which an application can request that specified processes be aborted by the system.

%%%%%%%%%%%
\subsection{\code{PMIx_Abort}}
\declareapi{PMIx_Abort}

%%%%
\summary

Abort the specified processes

%%%%
\format

\versionMarker{1.0}
\cspecificstart
\begin{codepar}
pmix_status_t
PMIx_Abort(int status, const char msg[],
           pmix_proc_t procs[], size_t nprocs)
\end{codepar}
\cspecificend

\begin{arglist}
\argin{status}{Error code to return to invoking environment (integer)}
\argin{msg}{String message to be returned to user (string)}
\argin{procs}{Array of \refstruct{pmix_proc_t} structures (array of handles)}
\argin{nprocs}{Number of elements in the \refarg{procs} array (integer)}
\end{arglist}

Returns \refconst{PMIX_SUCCESS} or a negative value corresponding to a PMIx error constant.

%%%%
\descr

Request that the host resource manager print the provided message and abort the provided array of \refarg{procs}.
A Unix or POSIX environment should handle the provided status as a return error code from the main program that launched the application.
A \code{NULL} for the \refarg{procs} array indicates that all processes in the caller's namespace are to be aborted, including itself.
Passing a \code{NULL} \refarg{msg} parameter is allowed.

\adviceuserstart
The response to this request is somewhat dependent on the specific \acl{RM} and its configuration (e.g., some resource managers will not abort the application if the provided status is zero unless specifically configured to do so, and some cannot abort subsets of processes in an application), and thus lies outside the control of PMIx itself.
However, the PMIx client library shall inform the \ac{RM} of the request that the specified \refarg{procs} be aborted, regardless of the value of the provided status.

Note that race conditions caused by multiple processes calling \refapi{PMIx_Abort} are left to the server implementation to resolve with regard to which status is returned and what messages (if any) are printed.
\adviceuserend


%%%%%%%%%%%%%%%%%%%%%%%%%%%%%%%%%%%%%%%%%%%%%%
%%%%%%%%%%%%%%%%%%%%%%%%%%%%%%%%%%%%%%%%%%%%%%
\section{Process Creation}
\label{chap:api_proc_mgmt:spawn}

The \refapi{PMIx_Spawn} commands spawn new jobs in the \ac{PMIx} universe.
The \ac{PMIx} calls made by the client into the \ac{PRI} no not, itself, spawn processes.
Instead the request is passed by the \ac{PRI} client-side library to the \ac{PMIx}-enabled \ac{RM} which will then create the processes on the allocated resources.

%%%%%%%%%%%
\subsection{\code{PMIx_Spawn}}
\declareapi{PMIx_Spawn}

%%%%
\summary

Spawn a new job.

%%%%
\format

\versionMarker{1.0}
\cspecificstart
\begin{codepar}
pmix_status_t
PMIx_Spawn(const pmix_info_t job_info[], size_t ninfo,
           const pmix_app_t apps[], size_t napps,
           char nspace[])
\end{codepar}
\cspecificend

\begin{arglist}
\argin{job_info}{Array of info structures (array of handles)}
\argin{ninfo}{Number of elements in the \refarg{job_info} array (integer)}
\argin{apps}{Array of \refstruct{pmix_app_t} structures (array of handles)}
\argin{napps}{Number of elements in the \refarg{apps} array (integer)}
\argout{nspace}{Namespace of the new job (string)}
\end{arglist}

Returns \refconst{PMIX_SUCCESS} or a negative value corresponding to a PMIx error constant.

\priattr
The PMIx_Spawn function in the \ac{PRI} does not parse or utilize the attributes directly. However, any provided attributes are passed to the host \ac{SMS} daemon for processing. Note that the \ac{PRI} automatically adds the following attributes to those provided before passing the request to the host daemon:

\pasteAttributeItem{PMIX_SPAWNED}
\pasteAttributeItem{PMIX_PARENT_ID}
\pasteAttributeItem{PMIX_REQUESTOR_IS_CLIENT}
\pasteAttributeItem{PMIX_REQUESTOR_IS_TOOL}

The \refattr{PMIX_SPAWNED} and \refattr{PMIX_PARENT_ID} attributes are passed to the child processes upon connection to a PMIx server.

\reqattr
\acp{RM} that implement support for \refapi{PMIx_Spawn} are required to support the following attributes:

\pasteAttributeItem{PMIX_WDIR}
\pasteAttributeItem{PMIX_SET_SESSION_CWD}
\pasteAttributeItem{PMIX_PREFIX}
\pasteAttributeItem{PMIX_HOST}
\pasteAttributeItem{PMIX_HOSTFILE}

\optattr
A complete implementation would include support for the following attributes:

\pasteAttributeItem{PMIX_ADD_HOSTFILE}
\pasteAttributeItem{PMIX_ADD_HOST}
\pasteAttributeItem{PMIX_PRELOAD_BIN}
\pasteAttributeItem{PMIX_PRELOAD_FILES}
\pasteAttributeItem{PMIX_PERSONALITY}
\pasteAttributeItem{PMIX_MAPPER}
\pasteAttributeItem{PMIX_DISPLAY_MAP}
\pasteAttributeItem{PMIX_PPR}
\pasteAttributeItem{PMIX_MAPBY}
\pasteAttributeItem{PMIX_RANKBY}
\pasteAttributeItem{PMIX_BINDTO}
\pasteAttributeItem{PMIX_NON_PMI}
\pasteAttributeItem{PMIX_STDIN_TGT}
\pasteAttributeItem{PMIX_FWD_STDIN}
\pasteAttributeItem{PMIX_FWD_STDOUT}
\pasteAttributeItem{PMIX_FWD_STDERR}
\pasteAttributeItem{PMIX_DEBUGGER_DAEMONS}
\pasteAttributeItem{PMIX_TAG_OUTPUT}
\pasteAttributeItem{PMIX_TIMESTAMP_OUTPUT}
\pasteAttributeItem{PMIX_MERGE_STDERR_STDOUT}
\pasteAttributeItem{PMIX_OUTPUT_TO_FILE}
\pasteAttributeItem{PMIX_INDEX_ARGV}
\pasteAttributeItem{PMIX_CPUS_PER_PROC}
\pasteAttributeItem{PMIX_NO_PROCS_ON_HEAD}
\pasteAttributeItem{PMIX_NO_OVERSUBSCRIBE}
\pasteAttributeItem{PMIX_REPORT_BINDINGS}
\pasteAttributeItem{PMIX_CPU_LIST}
\pasteAttributeItem{PMIX_JOB_RECOVERABLE}
\pasteAttributeItem{PMIX_JOB_CONTINUOUS}
\pasteAttributeItem{PMIX_MAX_RESTARTS}

%%%%
\descr

Spawn a new job.
The assigned namespace of the spawned applications is returned in the \refarg{nspace} parameter.
A \code{NULL} value in that location indicates that the caller doesn't wish to have the namespace returned.
The \refarg{nspace} array must be at least of size one more than \refconst{PMIX_MAX_NSLEN}.
Behavior of individual resource managers may differ, but it is expected that failure of any application process to start will result in termination/cleanup of \emph{all} processes in the newly spawned job and return of an error code to the caller.

By default, the spawned processes will be PMIx ``connected'' to the parent process upon successful launch (see \refapi{PMIx_Connect} description for details).
Note that this only means that (a) the parent process will be given a copy of the new job's
information so it can query job-level info without incurring any communication penalties, (b) newly spawned child processes will receive a copy of the parent processes job-level info, and (c) both the parent process and members of the child job will receive notification of errors from processes in their combined assemblage.

%%%%%%%%%%%
\subsection{\code{PMIx_Spawn_nb}}
\declareapi{PMIx_Spawn_nb}

%%%%
\summary

Nonblocking version of the \refapi{PMIx_Spawn} routine.

%%%%
\format

\versionMarker{1.0}
\cspecificstart
\begin{codepar}
pmix_status_t
PMIx_Spawn_nb(const pmix_info_t job_info[], size_t ninfo,
              const pmix_app_t apps[], size_t napps,
              pmix_spawn_cbfunc_t cbfunc, void *cbdata)
\end{codepar}
\cspecificend

\begin{arglist}
\argin{job_info}{Array of info structures (array of handles)}
\argin{ninfo}{Number of elements in the \refarg{job_info} array (integer)}
\argin{apps}{Array of \refstruct{pmix_app_t} structures (array of handles)}
\argin{cbfunc}{Callback function \refapi{pmix_spawn_cbfunc_t} (function reference)}
\argin{cbdata}{Data to be passed to the callback function (memory reference)}
\end{arglist}

Returns \refconst{PMIX_SUCCESS} or a negative value corresponding to a PMIx error constant.

\priattr
The PMIx_Spawn_nb function in the \ac{PRI} does not parse or utilize the attributes directly. However, any provided attributes are passed to the host \ac{SMS} daemon for processing. Note that the \ac{PRI} automatically adds the following attributes to those provided before passing the request to the host daemon:

\pasteAttributeItem{PMIX_SPAWNED}
\pasteAttributeItem{PMIX_PARENT_ID}
\pasteAttributeItem{PMIX_REQUESTOR_IS_CLIENT}
\pasteAttributeItem{PMIX_REQUESTOR_IS_TOOL}

The \refattr{PMIX_SPAWNED} and \refattr{PMIX_PARENT_ID} attributes are passed to the child processes upon connection to a PMIx server.

\reqattr
\acp{RM} that implement support for \refapi{PMIx_Spawn} are required to support the following attributes:

\pasteAttributeItem{PMIX_WDIR}
\pasteAttributeItem{PMIX_SET_SESSION_CWD}
\pasteAttributeItem{PMIX_PREFIX}
\pasteAttributeItem{PMIX_HOST}
\pasteAttributeItem{PMIX_HOSTFILE}

\optattr
A complete implementation would include support for the following attributes:

\pasteAttributeItem{PMIX_ADD_HOSTFILE}
\pasteAttributeItem{PMIX_ADD_HOST}
\pasteAttributeItem{PMIX_PRELOAD_BIN}
\pasteAttributeItem{PMIX_PRELOAD_FILES}
\pasteAttributeItem{PMIX_PERSONALITY}
\pasteAttributeItem{PMIX_MAPPER}
\pasteAttributeItem{PMIX_DISPLAY_MAP}
\pasteAttributeItem{PMIX_PPR}
\pasteAttributeItem{PMIX_MAPBY}
\pasteAttributeItem{PMIX_RANKBY}
\pasteAttributeItem{PMIX_BINDTO}
\pasteAttributeItem{PMIX_NON_PMI}
\pasteAttributeItem{PMIX_STDIN_TGT}
\pasteAttributeItem{PMIX_FWD_STDIN}
\pasteAttributeItem{PMIX_FWD_STDOUT}
\pasteAttributeItem{PMIX_FWD_STDERR}
\pasteAttributeItem{PMIX_DEBUGGER_DAEMONS}
\pasteAttributeItem{PMIX_TAG_OUTPUT}
\pasteAttributeItem{PMIX_TIMESTAMP_OUTPUT}
\pasteAttributeItem{PMIX_MERGE_STDERR_STDOUT}
\pasteAttributeItem{PMIX_OUTPUT_TO_FILE}
\pasteAttributeItem{PMIX_INDEX_ARGV}
\pasteAttributeItem{PMIX_CPUS_PER_PROC}
\pasteAttributeItem{PMIX_NO_PROCS_ON_HEAD}
\pasteAttributeItem{PMIX_NO_OVERSUBSCRIBE}
\pasteAttributeItem{PMIX_REPORT_BINDINGS}
\pasteAttributeItem{PMIX_CPU_LIST}
\pasteAttributeItem{PMIX_JOB_RECOVERABLE}
\pasteAttributeItem{PMIX_JOB_CONTINUOUS}
\pasteAttributeItem{PMIX_MAX_RESTARTS}

%%%%
\descr

Nonblocking version of the \refapi{PMIx_Spawn} routine.


%%%%%%%%%%%%%%%%%%%%%%%%%%%%%%%%%%%%%%%%%%%%%%
%%%%%%%%%%%%%%%%%%%%%%%%%%%%%%%%%%%%%%%%%%%%%%
\section{Connecting and Disconnecting Processes}
\label{chap:api_proc_mgmt:connect}

This section defines functions to connect and disconnect separate \ac{PMIx} namespaces so that they may exchange information and otherwise communicate with each other.

%%%%%%%%%%%
\subsection{\code{PMIx_Connect}}
\declareapi{PMIx_Connect}

%%%%
\summary

Connect namespaces.

%%%%
\format

\versionMarker{1.0}
\cspecificstart
\begin{codepar}
pmix_status_t
PMIx_Connect(const pmix_proc_t procs[], size_t nprocs,
             const pmix_info_t info[], size_t ninfo)
\end{codepar}
\cspecificend

\begin{arglist}
\argin{procs}{Array of proc structures (array of handles)}
\argin{nprocs}{Number of elements in the \refarg{procs} array (integer)}
\argin{info}{Array of info structures (array of handles)}
\argin{ninfo}{Number of elements in the \refarg{info} array (integer)}
\end{arglist}

Returns \refconst{PMIX_SUCCESS} or a negative value corresponding to a PMIx error constant.

\priattr
The PMIx_Connect function in the \ac{PRI} does not parse or utilize the attributes directly. However, any provided attributes are passed to the host \ac{SMS} daemon for processing.

\optattr
A complete implementation would include support for the following attributes:

\pasteAttributeItem{PMIX_TIMEOUT}
\pasteAttributeItem{PMIX_COLLECTIVE_ALGO}
\pasteAttributeItem{PMIX_COLLECTIVE_ALGO_REQD}

%%%%
\descr

Record the specified processes as ``connected''.
This means that the resource manager should treat the failure of any process in the specified collection as a reportable event, and take appropriate action.
Note that different resource managers may respond to failures in different manners. The RM does not define a new identifier for the connected assemblage, nor does it define a new rank for each process within that group. In addition, the \ac{PMIx} server does not provide any tracking support for the assemblage. Thus, the caller is responsible for maintaining the membership list of the assemblage.

The function will return once all participating processes have called either \refapi{PMIx_Connect} or its non-blocking version.
The server is required to return any job-level info for the connecting processes that might not already have it (i.e., if the connect request involves \refarg{procs} from different namespaces, then each \refarg{proc} shall receive the job-level info from those namespaces other than their own). In addition, all members of the collection will receive notification of errors from processes in their combined assemblage. Processes that combine via \refapi{PMIx_Connect} must all depart the assemblage together – i.e., no member can depart the collective while leaving the remaining members in it.

A process can only engage in \emph{one} connect operation involving the identical set of processes at a time.
However, a process \emph{can} be simultaneously engaged in multiple connect operations, each involving a different set of processes.

As in the case of the fence operation, the info array can be used to pass user-level directives regarding the algorithm to be used for the collective operation involved in the ``connect'', timeout constraints, and other options available from the host RM.


%%%%%%%%%%%
\subsection{\code{PMIx_Connect_nb}}
\declareapi{PMIx_Connect_nb}

%%%%
\summary

Nonblocking \refapi{PMIx_Connect_nb} routine.

%%%%
\format

\versionMarker{1.0}
\cspecificstart
\begin{codepar}
pmix_status_t
PMIx_Connect_nb(const pmix_proc_t procs[], size_t nprocs,
                const pmix_info_t info[], size_t ninfo,
                pmix_op_cbfunc_t cbfunc, void *cbdata)
\end{codepar}
\cspecificend

\begin{arglist}
\argin{procs}{Array of proc structures (array of handles)}
\argin{nprocs}{Number of elements in the \refarg{procs} array (integer)}
\argin{info}{Array of info structures (array of handles)}
\argin{ninfo}{Number of element in the \refarg{info} array (integer)}
\argin{cbfunc}{Callback function \refapi{pmix_op_cbfunc_t} (function reference)}
\argin{cbdata}{Data to be passed to the callback function (memory reference)}
\end{arglist}

Returns \refconst{PMIX_SUCCESS} or a negative value corresponding to a PMIx error constant.

\priattr
The PMIx_Connect_nb function in the \ac{PRI} does not parse or utilize the attributes directly. However, any provided attributes are passed to the host \ac{SMS} daemon for processing.

\optattr
A complete implementation would include support for the following attributes:

\pasteAttributeItem{PMIX_TIMEOUT}
\pasteAttributeItem{PMIX_COLLECTIVE_ALGO}
\pasteAttributeItem{PMIX_COLLECTIVE_ALGO_REQD}

%%%%
\descr

Nonblocking version of \refapi{PMIx_Connect}. The callback function is called once all participating processes have called either \refapi{PMIx_Connect_nb} or its blocking version.


%%%%%%%%%%%
\subsection{\code{PMIx_Disconnect}}
\declareapi{PMIx_Disconnect}

%%%%
\summary

Disconnect a previously connected set of processes.

%%%%
\format

\versionMarker{1.0}
\cspecificstart
\begin{codepar}
pmix_status_t
PMIx_Disconnect(const pmix_proc_t procs[], size_t nprocs,
                const pmix_info_t info[], size_t ninfo);
\end{codepar}
\cspecificend

\begin{arglist}
\argin{procs}{Array of proc structures (array of handles)}
\argin{nprocs}{Number of elements in the \refarg{procs} array (integer)}
\argin{info}{Array of info structures (array of handles)}
\argin{ninfo}{Number of element in the \refarg{info} array (integer)}
\end{arglist}

Returns \refconst{PMIX_SUCCESS} or a negative value corresponding to a PMIx error constant.

\priattr
The PMIx_Disonnect function in the \ac{PRI} does not parse or utilize the attributes directly. However, any provided attributes are passed to the host \ac{SMS} daemon for processing.

\optattr
A complete implementation would include support for the following attributes:

\pasteAttributeItem{PMIX_TIMEOUT}

%%%%
\descr

Disconnect a previously connected set of processes.
An error will be returned if the specified set of \refarg{procs} was not previously ``connected''.
As with \refapi{PMIx_Connect}, a process may be involved in multiple simultaneous disconnect operations.
However, a process is not allowed to reconnect to a set of \refarg{procs} that has not fully completed disconnect (i.e., you have to fully disconnect before you can reconnect to the \emph{same} group of processes).
The \refarg{info} array is used as in \refapi{PMIx_Connect}.


%%%%%%%%%%%
\subsection{\code{PMIx_Disconnect_nb}}
\declareapi{PMIx_Disconnect_nb}

%%%%
\summary

Nonblocking \refapi{PMIx_Disconnect} routine.

%%%%
\format

\versionMarker{1.0}
\cspecificstart
\begin{codepar}
pmix_status_t
PMIx_Disconnect_nb(const pmix_proc_t ranges[], size_t nprocs,
                   const pmix_info_t info[], size_t ninfo,
                   pmix_op_cbfunc_t cbfunc, void *cbdata);
\end{codepar}
\cspecificend

\begin{arglist}
\argin{procs}{Array of proc structures (array of handles)}
\argin{nprocs}{Number of elements in the \refarg{procs} array (integer)}
\argin{info}{Array of info structures (array of handles)}
\argin{ninfo}{Number of element in the \refarg{info} array (integer)}
\argin{cbfunc}{Callback function \refapi{pmix_op_cbfunc_t} (function reference)}
\argin{cbdata}{Data to be passed to the callback function (memory reference)}
\end{arglist}

Returns \refconst{PMIX_SUCCESS} or a negative value corresponding to a PMIx error constant.

\priattr
The PMIx_Disonnect_nb function in the \ac{PRI} does not parse or utilize the attributes directly. However, any provided attributes are passed to the host \ac{SMS} daemon for processing.

\optattr
A complete implementation would include support for the following attributes:

\pasteAttributeItem{PMIX_TIMEOUT}

%%%%
\descr

Nonblocking \refapi{PMIx_Disconnect} routine.


%%%%%%%%%%%%%%%%%%%%%%%%%%%%%%%%%%%%%%%%%%%%%%
%%%%%%%%%%%%%%%%%%%%%%%%%%%%%%%%%%%%%%%%%%%%%%
\section{Query}
\label{chap:api_proc_mgmt:query}

As the level of interaction between applications and the host \ac{SMS} grows, so too does the need for the application to query the \ac{SMS} regarding its capabilities and state information. \ac{PMIx} provides a generalized query interface for this purpose, along with a set of standardized attribute keys to support a range of requests. This includes requests to determine the status of scheduling queues and active allocations, the scope of \ac{API} and attribute support offered by the \ac{SMS}, namespaces of active jobs, location and information about a job's processes, and information regarding available resources.

An example use-case for the \refapi{PMIx_Query_nb} \ac{API} is to ensure clean job completion. Time-shared systems frequently impose maximum run times when assigning jobs to resource allocations. To shut down gracefully, e.g., to write a checkpoint before termination, it is necessary for an application to periodically query the resource manager for the time remaining in its allocation. This is especially true on systems for which allocation times may be shortened or lengthened from the original time limit. Many resource managers provide \acp{API} to dynamically obtain this information, but each \ac{API} is specific to the resource manager.

\ac{PMIx} defines an attribute key (\refattr{PMIX_TIME_REMAINING}) that can be used with the \refapi{PMIx_Query_nb} interface to obtain the number of seconds remaining in the current job allocation. Note that one could alternatively use the \refapi{PMIx_Register_event_handler} \ac{API} to register for an event indicating incipient job termination, and then use the \refapi{PMIx_Job_control_nb} \ac{API} to request that the host \ac{SMS} generate an event a specified amount of time prior to reaching the maximum run time. \ac{PMIx} provides such alternate methods as a means of maximizing the probability of a host system supporting at least one method by which the application can obtain the desired service.

The following \acp{API} support query of various session and environment values.

%%%%%%%%%%%
\subsection{\code{PMIx_Resolve_peers}}
\declareapi{PMIx_Resolve_peers}

%%%%
\summary

Obtain the array of processes within the specified namespace that are executing on a given node.

%%%%
\format

\versionMarker{1.0}
\cspecificstart
\begin{codepar}
pmix_status_t
PMIx_Resolve_peers(const char *nodename, const char *nspace,
                   pmix_proc_t **procs, size_t *nprocs)
\end{codepar}
\cspecificend

\begin{arglist}
\argin{nodename}{Name of the node to query (string)}
\argin{nspace}{namespace (string)}
\argout{procs}{Array of process structures (array of handles)}
\argout{nprocs}{Number of elements in the \refarg{procs} array (integer)}
\end{arglist}

Returns \refconst{PMIX_SUCCESS} or a negative value corresponding to a PMIx error constant.

%%%%
\descr

Given a \refarg{nodename}, return the array of processes within the specified \refarg{nspace}
that are executing on that node.
If the \refarg{nspace} is \code{NULL}, then all processes on the node will be returned.
If the specified node does not currently host any processes, then the returned array will be \code{NULL}, and \refarg{nprocs} will be \code{0}.
The caller is responsible for releasing the \refarg{procs} array when done with it.
The \refapi{PMIX_PROC_FREE} macro is provided for this purpose.

%%%%%%%%%%%
\subsection{\code{PMIx_Resolve_nodes}}
\declareapi{PMIx_Resolve_nodes}

%%%%
\summary

Return a list of nodes hosting processes within the given namespace.

%%%%
\format

\versionMarker{1.0}
\cspecificstart
\begin{codepar}
pmix_status_t
PMIx_Resolve_nodes(const char *nspace, char **nodelist)
\end{codepar}
\cspecificend

\begin{arglist}
\argin{nspace}{Namespace (string)}
\argout{nodelist}{Comma-delimited list of nodenames (string)}
\end{arglist}

Returns \refconst{PMIX_SUCCESS} or a negative value corresponding to a PMIx error constant.

%%%%
\descr

Given a \refarg{nspace}, return the list of nodes hosting processes within that namespace.
The returned string will contain a comma-delimited list of nodenames.
The caller is responsible for releasing the string when done with it.


%%%%%%%%%%%
\subsection{\code{PMIx_Query_info_nb}}
\declareapi{PMIx_Query_info_nb}
\declareapi{pmix_info_cbfunc_t}

%%%%
\summary

Query information about the system in general.

%%%%
\format

\versionMarker{2.0}
\cspecificstart
\begin{codepar}
pmix_status_t
PMIx_Query_info_nb(pmix_query_t queries[], size_t nqueries,
                   pmix_info_cbfunc_t cbfunc, void *cbdata)

typedef void (*pmix_info_cbfunc_t)(pmix_status_t status,
                                   pmix_info_t *info, size_t ninfo,
                                   void *cbdata,
                                   pmix_release_cbfunc_t release_fn,
                                   void *release_cbdata)
\end{codepar}
\cspecificend

\begin{arglist}
\argin{queries}{Array of query structures (array of handles)}
\argin{nqueries}{Number of elements in the \refarg{queries} array (integer)}
\argin{cbfunc}{Callback function \refapi{pmix_info_cbfunc_t} (function reference)}
\argin{cbdata}{Data to be passed to the callback function (memory reference)}
\end{arglist}

\begin{constantdesc}
\item \refconst{PMIX_SUCCESS} All data has been returned
\item \refconst{PMIX_ERR_NOT_FOUND} None of the requested data was available
\item \refconst{PMIX_ERR_PARTIAL_SUCCESS} Some of the data has been returned
\item \refconst{PMIX_ERR_NOT_SUPPORTED} The host \ac{RM} does not support this function
\end{constantdesc}

\priattr
The PMIx_Query_info_nb function in the \ac{PRI} does not parse or utilize the attributes directly. However, any provided attributes are passed to the host \ac{SMS} daemon for processing.

\optattr
A complete implementation would include support for the following attributes:

\pasteAttributeItem{PMIX_QUERY_NAMESPACES}
\pasteAttributeItem{PMIX_QUERY_JOB_STATUS}
\pasteAttributeItem{PMIX_QUERY_QUEUE_LIST}
\pasteAttributeItem{PMIX_QUERY_QUEUE_STATUS}
\pasteAttributeItem{PMIX_QUERY_PROC_TABLE}
\pasteAttributeItem{PMIX_QUERY_LOCAL_PROC_TABLE}
\pasteAttributeItem{PMIX_QUERY_SPAWN_SUPPORT}
\pasteAttributeItem{PMIX_QUERY_DEBUG_SUPPORT}
\pasteAttributeItem{PMIX_QUERY_MEMORY_USAGE}
\pasteAttributeItem{PMIX_QUERY_LOCAL_ONLY}
\pasteAttributeItem{PMIX_QUERY_REPORT_AVG}
\pasteAttributeItem{PMIX_QUERY_REPORT_MINMAX}
\pasteAttributeItem{PMIX_QUERY_ALLOC_STATUS}
\pasteAttributeItem{PMIX_TIME_REMAINING}

%%%%
\descr

Query information about the system in general.
This can include a list of active namespaces, network topology, etc.
Also can be used to query node-specific info such as the list of peers executing on a given node.
We assume that the host \ac{RM} will exercise appropriate access control on the information.

NOTE: There is no blocking form of this API as the structures passed to query info differ from those for receiving the results.

The \refarg{status} argument to the callback function indicates if requested data was found or not.
An array of \refstruct{pmix_info_t} will contain each key that was provided and the corresponding value that was found. Requests for keys that are not found will return the key paired with a value of type \refconst{PMIX_UNDEF}.

\adviceuserstart
The desire to query a list of attributes supported by the implementation and/or the host environment has been expressed and noted. The \ac{PMIx} community is exploring the possibility and it will likely become available in a future release
\adviceuserend



%%%%%%%%%%%%%%%%%%%%%%%%%%%%%%%%%%%%%%%%%%%%%%%%%
