%%%%%%%%%%%%%%%%%%%%%%%%%%%%%%%%%%%%%%%%%%%%%%%%%
% Chapter: Events
%%%%%%%%%%%%%%%%%%%%%%%%%%%%%%%%%%%%%%%%%%%%%%%%%
\chapter{Event Notification}
\label{chap:api_event}

\ldots


%%%%%%%%%%%%%%%%%%%%%%%%%%%%%%%%%%%%%%%%%%%%%%
%%%%%%%%%%%%%%%%%%%%%%%%%%%%%%%%%%%%%%%%%%%%%%
\section{Logging}
\label{chap:api_event:logging}

\ldots

%%%%%%%%%%%
\subsection{\code{PMIx_Log_nb}}
\declareapi{PMIx_Log_nb}

%%%%
\summary

Log data to a data service.

%%%%
\format

\cspecificstart
\begin{codepar}
pmix_status_t
PMIx_Log_nb(const pmix_info_t data[], size_t ndata,
            const pmix_info_t directives[], size_t ndirs,
            pmix_op_cbfunc_t cbfunc, void *cbdata)
\end{codepar}
\cspecificend

\begin{arglist}
\argin{data}{Array of info structures (array of handles)}
\argin{ndata}{Number of elements in the \refarg{data} array (integer)}
\argin{directives}{Array of info structures (array of handles)}
\argin{ndirs}{Number of elements in the \refarg{directives} array (integer)}
\argin{cbfunc}{Callback function \refapi{pmix_op_cbfunc_t} (function reference)}
\argin{cbdata}{Data to be passed to the callback function (memory reference)}
\end{arglist}

Returns \refconst{PMIX_SUCCESS} or a negative value corresponding to a PMIx error constant.

%%%%
\descr

Log data to a ``central'' data service/store, subject to the services offered by the host resource manager.
The data to be logged is provided in the \refarg{data} array.
The (optional) \refarg{directives} can be used to request specific storage options and direct the choice of storage option.

The callback function will be executed when the log operation has been completed.
The \refarg{data} array must be maintained until the callback is provided.


%%%%%%%%%%%%%%%%%%%%%%%%%%%%%%%%%%%%%%%%%%%%%%
%%%%%%%%%%%%%%%%%%%%%%%%%%%%%%%%%%%%%%%%%%%%%%
\section{Notification and Management}
\label{chap:api_event:notify}

\ldots


%%%%%%%%%%%
\subsection{\code{PMIx_Register_event_handler}}
\declareapi{PMIx_Register_event_handler}

\cspecificstart
\begin{codepar}
/* Register an event handler to report events. Three types of events
 * can be reported:
 *
 * (a) those that occur within the client library, but are not
 *     reportable via the API itself (e.g., loss of connection to
 *     the server). These events typically occur during behind-the-scenes
 *     non-blocking operations.
 *
 * (b) job-related events such as the failure of another process in
 *     the job or in any connected job, impending failure of hardware
 *     within the job's usage footprint, etc.
 *
 * (c) system notifications that are made available by the local
 *     administrators
 *
 * By default, only events that directly affect the process and/or
 * any process to which it is connected (via the PMIx_Connect call)
 * will be reported. Options to modify that behavior can be provided
 * in the info array
 *
 * Both the client application and the resource manager can register
 * err handlers for specific events. PMIx client/server calls the registered
 * err handler upon receiving event notify notification (via PMIx_Notify_event)
 * from the other end (Resource Manager/Client application).
 *
 * Multiple err handlers can be registered for different events. PMIX returns
 * an integer reference to each register handler in the callback fn. The caller
 * must retain the reference in order to deregister the evhdlr.
 * Modification of the notification behavior can be accomplished by
 * deregistering the current evhdlr, and then registering it
 * using a new set of info values.
 *
 * See pmix_common.h for a description of the notification function */
void
PMIx_Register_event_handler(pmix_status_t codes[], size_t ncodes,
                            pmix_info_t info[], size_t ninfo,
                            pmix_notification_fn_t evhdlr,
                            pmix_evhdlr_reg_cbfunc_t cbfunc,
                            void *cbdata);
\end{codepar}
\cspecificend


%%%%%%%%%%%
\subsection{\code{PMIx_Deregister_event_handler}}
\declareapi{PMIx_Deregister_event_handler}

\cspecificstart
\begin{codepar}
/* Deregister an event handler
 * evhdlr_ref is the reference returned by PMIx from the call to
 * PMIx_Register_event_handler. If non-NULL, the provided cbfunc
 * will be called to confirm removal of the designated handler */
void
PMIx_Deregister_event_handler(size_t evhdlr_ref,
                              pmix_op_cbfunc_t cbfunc,
                              void *cbdata);
\end{codepar}
\cspecificend


%%%%%%%%%%%
\subsection{\code{PMIx_Notify_event}}
\declareapi{PMIx_Notify_event}

\cspecificstart
\begin{codepar}
/* Report an event to a process for notification via any
 * registered evhdlr. The evhdlr registration can be
 * called by both the server and the client application. On the
 * server side, the evhdlr is used to report events detected
 * by PMIx to the host server for handling. On the client side,
 * the evhdlr is used to notify the process of events
 * reported by the server - e.g., the failure of another process.
 *
 * This function allows the host server to direct the server
 * convenience library to notify all registered local procs of
 * an event. The event can be local, or anywhere in the cluster.
 * The status indicates the event being reported.
 *
 * The client application can also call this function to notify the
 * resource manager of an event it encountered. It can request the host
 * server to notify the indicated processes about the event.
 *
 * The array of procs identifies the processes that will be impacted
 * by the event. This could consist of a single process, or a number
 * of processes.
 *
 * The info array contains any further info the RM can and/or chooses
 * to provide.
 *
 * The callback function will be called upon completion of the
 * notify_event function's actions. Note that any messages will
 * have been queued, but may not have been transmitted by this
 * time. Note that the caller is required to maintain the input
 * data until the callback function has been executed!
*/
pmix_status_t
PMIx_Notify_event(pmix_status_t status,
                  const pmix_proc_t *source,
                  pmix_data_range_t range,
                  pmix_info_t info[], size_t ninfo,
                  pmix_op_cbfunc_t cbfunc, void *cbdata);
\end{codepar}
\cspecificend


%%%%%%%%%%%
\subsection{Event Notification Callback Function}
\declareapi{pmix_event_notification_cbfunc_fn_t}

The \refapi{pmix_event_notification_cbfunc_fn_t} \ldots

\cspecificstart
\begin{codepar}
/* define a callback by which an event handler can notify the PMIx library
 * that it has completed its response to the notification. The handler
 * is _required_ to execute this callback so the library can determine
 * if additional handlers need to be called. The handler shall return
 * PMIX_SUCCESS if no further action is required. The return status
 * of each event handler and any returned pmix_info_t structures
 * will be added to the array of pmix_info_t passed to any subsequent
 * event handlers to help guide their operation.
 *
 * If non-NULL, the provided callback function will be called to allow
 * the event handler to release the provided info array.
 */
typedef void (*pmix_event_notification_cbfunc_fn_t)(pmix_status_t status,
                                                    pmix_info_t *results, size_t nresults,
                                                    pmix_op_cbfunc_t cbfunc, void *thiscbdata,
                                                    void *notification_cbdata);
\end{codepar}
\cspecificend

%%%%
\descr

\ldots


%%%%%%%%%%%
\subsection{Notification Callback Function}
\declareapi{pmix_notification_fn_t}

The \refapi{pmix_notification_fn_t} \ldots

\cspecificstart
\begin{codepar}
/* define a callback function for the event handler. Upon receipt of an
 * event notification, PMIx will execute the specified notification
 * callback function, providing:
 *
 * evhdlr_registration_id - the returned registration number of
 *                          the event handler being called
 * status - the event that occurred
 * source - the nspace and rank of the process that generated
 *          the event. If the source is the resource manager,
 *          then the nspace will be empty and the rank will
 *          be PMIX_RANK_UNDEF
 * info - any additional info provided regarding the event.
 * ninfo - the number of info objects in the info array
 * results - any provided results from event handlers called
 *           prior to this one.
 * nresults - number of info objects in the results array
 * cbfunc - the function to be called upon completion of the handler
 * cbdata - pointer to be returned in the completion cbfunc
 *
 * Note that different resource managers may provide differing levels
 * of support for event notification to application processes. Thus, the
 * info array may be NULL or may contain detailed information of the event.
 * It is the responsibility of the application to parse any provided info array
 * for defined key-values if it so desires.
 *
 * Possible uses of the pmix_info_t object include:
 *
 * - for the RM to alert the process as to planned actions, such as
 *   to abort the session, in response to the reported event
 *
 * - provide a timeout for alternative action to occur, such as for
 *   the application to request an alternate response to the event
 *
 * For example, the RM might alert the application to the failure of
 * a node that resulted in termination of several processes, and indicate
 * that the overall session will be aborted unless the application
 * requests an alternative behavior in the next 5 seconds. The application
 * then has time to respond with a checkpoint request, or a request to
 * recover from the failure by obtaining replacement nodes and restarting
 * from some earlier checkpoint.
 *
 * Support for these options is left to the discretion of the host RM. Info
 * keys are included in the common definions above, but also may be augmented
 * on a per-RM basis.
 *
 * On the server side, the notification function is used to inform the host
 * server of a detected event in the PMIx subsystem and/or client
 */
typedef void (*pmix_notification_fn_t)(size_t evhdlr_registration_id,
                                       pmix_status_t status,
                                       const pmix_proc_t *source,
                                       pmix_info_t info[], size_t ninfo,
                                       pmix_info_t *results, size_t nresults,
                                       pmix_event_notification_cbfunc_fn_t cbfunc,
                                       void *cbdata);
\end{codepar}
\cspecificend

%%%%
\descr

\ldots



%%%%%%%%%%%
\subsection{Event Handler Registration Function}
\declareapi{pmix_evhdlr_reg_cbfunc_t}

The \refapi{pmix_evhdlr_reg_cbfunc_t} \ldots

\cspecificstart
\begin{codepar}
/* define a callback function for calls to PMIx_Register_evhdlr. The
 * status indicates if the request was successful or not, evhdlr_ref is
 * an integer reference assigned to the event handler by PMIx, this reference
 * must be used to deregister the err handler. A ptr to the original
 * cbdata is returned. */
typedef void (*pmix_evhdlr_reg_cbfunc_t)(pmix_status_t status,
                                         size_t evhdlr_ref,
                                         void *cbdata)
\end{codepar}
\cspecificend

%%%%
\descr

\ldots


%%%%%%%%%%%%%%%%%%%%%%%%%%%%%%%%%%%%%%%%%%%%%%%%%
