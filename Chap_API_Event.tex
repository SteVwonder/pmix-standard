%%%%%%%%%%%%%%%%%%%%%%%%%%%%%%%%%%%%%%%%%%%%%%%%%
% Chapter: Events
%%%%%%%%%%%%%%%%%%%%%%%%%%%%%%%%%%%%%%%%%%%%%%%%%
\chapter{Event Notification}
\label{chap:api_event}

This chapter defines the \ac{PMIx} event notification system.
These interfaces are designed to support the reporting of events to/from clients and servers, and between library layers within a single process.

%%%%%%%%%%%%%%%%%%%%%%%%%%%%%%%%%%%%%%%%%%%%%%
%%%%%%%%%%%%%%%%%%%%%%%%%%%%%%%%%%%%%%%%%%%%%%
\section{Notification and Management}
\label{chap:api_event:notify}

\ac{PMIx} event notification provides an asynchronous out-of-band mechanism for communicating events between application processes and/or elements of the \ac{SMS}. Its uses span a wide range that includes fault notification, coordination between multiple programming libraries within a single process, and workflow orchestration for non-synchronous programming models. Events can be divided into two distinct classes:

\begin{itemize}
\item \textit{Job-specific events} directly relate to a job executing within the session, such as a debugger attachment, process failure within a related job, or events generated by an application process. Events in this category are to be immediately delivered to the \ac{PMIx} server library for relay to the related local processes.

\item \textit{Environment events} indirectly relate to a job but do not specifically target the job itself. This category includes \ac{SMS}-generated events such as \ac{ECC} errors, temperature excursions, and other non-job conditions that might directly affect a session's resources, but would never include an event generated by an application process. Note that although these do potentially impact the session's jobs, they are not directly tied to those jobs. Thus, events in this category are to be delivered to the \ac{PMIx} server library only upon request.
\end{itemize}

Both \ac{SMS} elements and applications can register for events of either type.

\adviceimplstart
Race conditions can cause the registration to come after events of possible interest (e.g., a memory \ac{ECC} event that occurs after start of execution but prior to registration, or an application process generating an event prior to another process registering to receive it). \ac{SMS} vendors are \textit{requested} to cache environment events for some time to mitigate this situation, but are not \textit{required} to do so. However, \ac{PMIx} implementers are \textit{required} to cache all events received by the \ac{PMIx} server library and to deliver them to registering clients in the same order in which they were received
\adviceimplend

\adviceuserstart
Applications must be aware that they may not receive environment events that occur prior to registration, depending upon the capabilities of the host \ac{SMS}.
\adviceuserend

The generator of an event can specify the \textit{target range} for delivery of that event. Thus, the generator can choose to limit notification to processes on the local node, processes within the same job as the generator, processes within the same allocation, other threads within the same process, only the \ac{SMS} (i.e., not to any application processes), all application processes, or to a custom range based on specific process identifiers. Only processes within the given range that register for the provided event code will be notified. In addition, the generator can use attributes to direct that the event not be delivered to any default event handlers, or to any multi-code handler (as defined below).

Event notifications provide the process identifier of the source of the event plus the event code and any additional information provided by the generator. When an event notification is received by a process, the registered handlers are scanned for their event code(s), with matching handlers assembled into an \textit{event chain} for servicing. Note that users can also specify a \textit{source range} when registering an event (using the same range designators described above) to further limit when they are to be invoked. When assembled, PMIx event chains are ordered based on both the specificity of the event handler and user directives at time of handler registration. By default, handlers are grouped into three categories based on the number of event codes that can trigger the callback:
\begin{itemize}
\item \textit{single-code} handlers are serviced first as they are the most specific. These are handlers that are registered against one specific event code.

\item \textit{multi-code} handlers are serviced once all single-code handlers have completed. The handler will be included in the chain upon receipt of an event matching any of the provided codes.

\item \textit{default} handlers are serviced once all multi-code handlers have completed. These handlers are always included in the chain unless the generator specifically excludes them.
\end{itemize}

Users can specify the callback order of a handler within its category at the time of registration. Ordering can be specified either by providing the relevant returned event handler registration ID or using event handler names, if the user specified an event handler name when registering the corresponding event. Thus, users can specify that a given handler be executed before or after another handler should both handlers appear in an event chain (the ordering is ignored if the other handler isn't included). Note that ordering does not imply immediate relationships. For example, multiple handlers registered to be serviced after event handler \textit{A} will all be executed after \textit{A}, but are not guaranteed to be executed in any particular order amongst themselves.

In addition, one event handler can be declared as the \textit{first} handler to be executed in the chain. This handler will \textit{always} be called prior to any other handler, regardless of category, provided the incoming event matches both the specified range and event code. Only one handler can be so designated --- attempts to designate additional handlers as \textit{first} will return an error. Deregistration of the declared \textit{first} handler will re-open the position for subsequent assignment.

Similarly, one event handler can be declared as the \textit{last} handler to be executed in the chain. Note that this handler will not be called if the chain is terminated by an earlier handler. Only one handler can be designated as \textit{last} --- attempts to designate additional handlers as \textit{last} will return an error. Deregistration of the declared \textit{last} handler will re-open the position for subsequent assignment.

\adviceuserstart
Note that \textit{first} and \textit{last} handlers are qualified within their category. That is, default event handlers will always be invoked after a single code event handler, even when the later is qualified as \textit{last}.
\adviceuserend

Upon completing its work and prior to returning, each handler \textit{must} call the event handler completion function provided when it was invoked (including a status code plus any information to be passed to later handlers) so that the chain can continue being progressed. \ac{PMIx} automatically aggregates the status and any results of each handler (as provided in the completion callback) with status from all prior handlers so that each step in the chain has full knowledge of what preceded it. An event handler can terminate all further progress along the chain by passing the \refconst{PMIX_EVENT_ACTION_COMPLETE} status to the completion callback function.


%%%%%%%%%%%
\subsection{\code{PMIx_Register_event_handler}}
\declareapi{PMIx_Register_event_handler}

%%%%
\summary

Register an event handler

%%%%
\format

\versionMarker{2.0}
\cspecificstart
\begin{codepar}
void
PMIx_Register_event_handler(pmix_status_t codes[], size_t ncodes,
                            pmix_info_t info[], size_t ninfo,
                            pmix_notification_fn_t evhdlr,
                            pmix_evhdlr_reg_cbfunc_t cbfunc,
                            void *cbdata);
\end{codepar}
\cspecificend

\begin{arglist}
\argin{codes}{Array of status codes (array of \refattr{pmix_status_t})}
\argin{ncodes}{Number of elements in the \refarg{codes} array (\code{size_t})}
\argin{info}{Array of info structures (array of handles)}
\argin{ninfo}{Number of elements in the \refarg{info} array (\code{size_t})}
\argin{evhdlr}{Event handler to be called \refapi{pmix_notification_fn_t} (function reference)}
\argin{cbfunc}{Callback function \refapi{pmix_evhdlr_reg_cbfunc_t} (function reference)}
\argin{cbdata}{Data to be passed to the cbfunc callback function (memory reference)}
\end{arglist}

\priattr
The following attributes are supported in the \ac{PRI}:

\pasteAttributeItem{PMIX_EVENT_HDLR_NAME}
\pasteAttributeItem{PMIX_EVENT_HDLR_FIRST}
\pasteAttributeItem{PMIX_EVENT_HDLR_LAST}
\pasteAttributeItem{PMIX_EVENT_HDLR_FIRST_IN_CATEGORY}
\pasteAttributeItem{PMIX_EVENT_HDLR_LAST_IN_CATEGORY}
\pasteAttributeItem{PMIX_EVENT_HDLR_BEFORE}
\pasteAttributeItem{PMIX_EVENT_HDLR_AFTER}
\pasteAttributeItem{PMIX_EVENT_HDLR_PREPEND}
\pasteAttributeItem{PMIX_EVENT_HDLR_APPEND}
\pasteAttributeItem{PMIX_EVENT_CUSTOM_RANGE}
\pasteAttributeItem{PMIX_RANGE}
\pasteAttributeItem{PMIX_EVENT_RETURN_OBJECT}
\pasteAttributeItem{PMIX_EVENT_JOB_LEVEL}
\pasteAttributeItem{PMIX_EVENT_ENVIRO_LEVEL}

\reqattr
\acp{RM} that implement support for PMIx event notification are required to support the following attributes:

\pasteAttributeItem{PMIX_EVENT_JOB_LEVEL}
\pasteAttributeItem{PMIX_EVENT_ENVIRO_LEVEL}
\pasteAttributeItem{PMIX_EVENT_AFFECTED_PROC}
\pasteAttributeItem{PMIX_EVENT_AFFECTED_PROCS}

\optattr
A complete implementation that supports \ac{PMIx} event notification would offer notifications for environmental events impacting the job, for \ac{SMS} events relating to the job, and include support for the following attributes:

\pasteAttributeItem{PMIX_EVENT_TERMINATE_SESSION}
\pasteAttributeItem{PMIX_EVENT_TERMINATE_JOB}
\pasteAttributeItem{PMIX_EVENT_TERMINATE_NODE}
\pasteAttributeItem{PMIX_EVENT_TERMINATE_PROC}
\pasteAttributeItem{PMIX_EVENT_ACTION_TIMEOUT}
\pasteAttributeItem{PMIX_EVENT_SILENT_TERMINATION}


%%%%
\descr

Register an event handler to report events. By default, only events that directly affect the process and/or any process to which it is connected (via the \refapi{PMIx_Connect} call) will be reported. Options to modify that behavior can be provided in the info array, including:

\begin{itemize}

\item Both the client application and the host \ac{SMS} can register handlers for specific events. \ac{PMIx} client/server calls the registered event handler upon receiving event notify notification (via \refapi{PMIx_Notify_event}) from the other end (Resource Manager/Client application).

\item Multiple event handlers can be registered for different events. PMIX returns a \code{size_t} reference to each register handler in the callback function. The caller \textit{must} retain the reference in order to deregister the evhdlr. The notification behavior can be modified by deregistering the current evhdlr, and then registering again using a new set of info values.
\end{itemize}

Note that the codes being registered do \textit{not} need to be \ac{PMIx} error constants --- any integer value can be registered. This allows for registration of non-PMIx events such as those defined by a particular \ac{SMS} vendor or by an application itself.

\adviceuserstart
In order to avoid potential conflicts, users are advised to only define codes that lie outside the range of the \ac{PMIx} standard's error codes. Thus, \ac{SMS} vendors and application developers should constrain their definitions to positive values or negative values beyond the \refconst{PMIX_EXTERNAL_ERR_BASE} boundary.
\adviceuserend

Upon completion, the callback will receive a status based on the following table:

\begin{constantdesc}
\item \refconst{PMIX_SUCCESS} The event handler was successfully registered - the event handler identifier is returned in the callback.
\item \refconst{PMIX_ERR_BAD_PARAM} One or more of the directives provided in the \textit{info} array was unrecognized.
\item \refconst{PMIX_ERR_NOT_SUPPORTED} The \ac{PMIx} implementation does not support event notification, or the host \ac{SMS} does not support notification of the specified event code.
\end{constantdesc}


\adviceuserstart
As previously stated, upon completing its work, and prior to returning, each handler \textit{must} call the event handler completion function provided when it was invoked (including a status code plus any information to be passed to later handlers) so that the chain can continue being progressed. An event handler can terminate all further progress along the chain by passing the \refconst{PMIX_EVENT_ACTION_COMPLETE} status to the completion callback function. Note that the parameters passed to the event handler (e.g., the \refarg{info} and \refarg{results} arrays) will cease to be valid once the completion function has been called - thus, any information in the incoming parameters that will be referenced following the call to the completion function must be copied.
\adviceuserend

%%%%%%%%%%%
\subsection{\code{PMIx_Deregister_event_handler}}
\declareapi{PMIx_Deregister_event_handler}

%%%%
\summary

Deregister an event handler.

%%%%
\format

\versionMarker{2.0}
\cspecificstart
\begin{codepar}
void
PMIx_Deregister_event_handler(size_t evhdlr_ref,
                              pmix_op_cbfunc_t cbfunc,
                              void *cbdata);
\end{codepar}
\cspecificend

\begin{arglist}
\argin{evhdlr_ref}{Event handler ID returned by registration (\code{size_t})}
\argin{cbfunc}{Callback function to be executed upon completion of operation \refapi{pmix_op_cbfunc_t} (function reference)}
\argin{cbdata}{Data to be passed to the cbfunc callback function (memory reference)}
\end{arglist}

%%%%
\descr

Deregister an event handler. If non-NULL, the provided cbfunc will be called to confirm removal of the designated handler, including a status code as per the following:

\begin{constantdesc}
\item \refconst{PMIX_SUCCESS} The event handler was successfully deregistered.
\item \refconst{PMIX_ERR_BAD_PARAM} The provided \refarg{evhdlr_ref} was unrecognized.
\item \refconst{PMIX_ERR_NOT_SUPPORTED} The \ac{PMIx} implementation does not support event notification.
\end{constantdesc}


%%%%%%%%%%%
\subsection{\code{PMIx_Notify_event}}
\declareapi{PMIx_Notify_event}

%%%%
\summary

Report an event for notification via any
registered event handler.

%%%%
\format

\versionMarker{2.0}
\cspecificstart
\begin{codepar}
pmix_status_t
PMIx_Notify_event(pmix_status_t status,
                  const pmix_proc_t *source,
                  pmix_data_range_t range,
                  pmix_info_t info[], size_t ninfo,
                  pmix_op_cbfunc_t cbfunc, void *cbdata);
\end{codepar}
\cspecificend

\begin{arglist}
\argin{status}{Status code of the event (\refattr{pmix_status_t})}
\argin{source}{Pointer to a \refapi{pmix_proc_t} identifying the original reporter of the event (handle)}
\argin{range}{Range across which this notification shall be delivered (\refattr{pmix_data_range_t})}
\argin{info}{Array of \refapi{pmix_info_t} structures containing any further info provided by the originator of the event (array of handles)}
\argin{ninfo}{Number of elements in the \refarg{info} array (\code{size_t})}
\argin{cbfunc}{Callback function to be executed upon completion of operation \refapi{pmix_op_cbfunc_t} (function reference)}
\argin{cbdata}{Data to be passed to the cbfunc callback function (memory reference)}
\end{arglist}

\begin{constantdesc}
\item \refconst{PMIX_SUCCESS} The notification request is valid and is being processed. The callback function will be called when the process-local operation is complete and will provide the resulting status of that operation. Note that this does \textit{not} reflect the success or failure of delivering the event to any recipients.
\item \refconst{PMIX_ERR_BAD_PARAM} The request contains at least one incorrect entry that prevents it from being processed. The callback function will \textit{not} be called.
\item \refconst{PMIX_ERR_NOT_SUPPORTED} The \ac{PMIx} implementation does not support event notification, or in the case of a \ac{PMIx} server calling the API, the range extended beyond the local node and the host \ac{SMS} environment does not support event notification. The callback function will \textit{not} be called.
\end{constantdesc}

\priattr
The following attributes are supported in the \ac{PRI}:

\pasteAttributeItem{PMIX_EVENT_AFFECTED_PROC}
\pasteAttributeItem{PMIX_EVENT_AFFECTED_PROCS}
\pasteAttributeItem{PMIX_EVENT_NON_DEFAULT}
\pasteAttributeItem{PMIX_EVENT_CUSTOM_RANGE}

%%%%
\descr

Report an event for notification via any registered event handler. This function can be called by any \ac{PMIx} process, including application processes, \ac{PMIx} servers, and \ac{SMS} elements. The \ac{PMIx} server calls this \ac{API} to report events it detected itself so that the host \ac{SMS} daemon distribute and handle them, and to pass events given to it by its host down to any attached client processes for processing. Examples might include notification of the failure of another process, detection of an impending node failure due to rising temperatures, or an intent to preempt the application. Events may be locally generated or come from anywhere in the system.

Host \ac{SMS} daemons call the API to pass events down to its embedded \ac{PMIx} server both for transmittal to local client processes and for the server's own internal processing.

Client application processes can call this function to notify the \ac{SMS} and/or other application processes of an event it encountered. Note that processes are not constrained to report status values defined in the official \ac{PMIx} standard --- any integer value can be used. Thus, applications are free to define their own internal events and use the notification system for their own internal purposes.

\adviceuserstart
The callback function will be called upon completion of the
\code{notify_event} function's actions. At that time, any messages required for executing the operation (e.g., to send the notification to the local \ac{PMIx} server) will
have been queued, but may not yet have been transmitted. The caller is required to maintain the input
data until the callback function has been executed --- the sole purpose of the callback function is to indicate when the input data is no longer required.
\adviceuserend


%%%%%%%%%%%%%%%%%%%%%%%%%%%%%%%%%%%%%%%%%%%%%%%%%
